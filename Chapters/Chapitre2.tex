\chapter{Vers quelle gestion des compétences}
Essayons de trouver quel type de gestion des compétences serait le plus adapté pour atteindre les objectifs définis, puis définissons une méthodologie et un plan d'action pour sa mise en place. Mais, avant tout, il faut s'accordons-nous sur la définition de "compétence". 

\section{Définition générale de la compétence}
Regardons ce que la littérature dit à propos de la notion de compétence. Commençons par ce qui en est dit dans le livre de Guérin, Cadin, Pigeyre~\citep[pp.170-171]{gestionressourceshumaine2007}
\begin{quotation}
\textit{ On observe une grande variété dans les définitions adoptées, mais toutes retiennent, d'une manière ou d'une autre, les mêmes éléments essentiels:
\begin{itemize}
 \item La compétence prend sens par rapport à l'action: on ne peut parler de compétence que dans le cadre précis d'une situation de travail.
 \item Elle combine de façon dynamique les différents éléments qui la constituent (savoirs, savoir-faire pratiques, raisonnements) pour répondre à des exigences d'adaptation.
\end{itemize}}
\end{quotation}

Plus loin, nous retrouvons encore ce lien entre compétence et action.~\citep[pp.172]{gestionressourceshumaine2007}
\begin{quotation}
 \textit{"La compétence est une notion abstraite et hypothétique."}\footnote{Leplat J., ``Compétence et ergonomie'', Modèle en analyse du travail, Bruxelles, Mardaga, 1991, pp. 263-278}
 \textit{On ne peut en observer que les manifestations. [...], c'est à partir de la situation de travail et de la manière dont celle-ci est assumée qu'il est possible d'inférer la compétence.}
\end{quotation}

Et finalement, n'oublions pas le caractère socialement construit de la compétence. ~\citep[pp.249]{gestionressourceshumaine2007}
\begin{quotation}
\textit{[...] c'est le fait de reconnaître une compétence qui la fait exister. Autrement dit, la compétence n'existe pas sans le jugement d'autrui: nul ne saurait se déclarer compétent lui-même.}
\end{quotation}

Nous observons un point de vue assez similaire dans Aubret, Gilbert~\citep[pp.7]{evalcompetence} où là encore il ne s'accorde pas sur une définition unique de la compétence:

\begin{quotation}
\textit{La notion de compétence se présente souvent comme une notion insaisissable au regard de la diversité des usages. [...]
Le terme de compétence fait partie de ces mots à multiples facettes que personne n'a véritablement le pouvoir de réduire à une seule [définition] non équivoque et de l'imposer à tous.
Aussi, nous voyons apparaître dans la littérature sur les compétences de nombreuses définitions qui prennent ce mot, soit comme un terme, soit comme une notion, un concept ou un construit social. [...] R. Zemke (1995), [...], en arrive à la conclusion que la compétence, les compétences, les modèles de compétences et la formation axée sur la compétence sont des mots valises qui signifient seulement ce que l'auteur veut leur faire dire.}

\textit{Ce que disent les chercheurs et praticiens:}
\textit{\begin{itemize}
 \item Compétence: c'est la capacité à résoudre un problème dans un contexte donné (Michel \& Ledru, 1991);
 \item Les compétences sont des ensembles de connaissances, de capacités d'actions et de comportements structurés en fonction d'un but et dans un type de situations données (Gilbert \& Parlier, 1992);
 \item [...]
 \item La compétence est la prise d'initiative et de responsabilité de l'individu sur des situations professionnelles auxquelles il est confronté (Zarifian, 1999)
\end{itemize}}

\end{quotation}~\citep[pp.7-8]{evalcompetence}

Finalement l'article de Delobbe, Gilbert, Le Boudelaire~\citep[pp.30]{delobbe} résume assez bien la complexité de la situation.
\begin{quotation}
\textit{Les définitions de la compétence ont été marquées par des divergences idéologiques qui se sont traduites dans la façon concrète de formuler les référentiels. Entre le savoir-faire opérationnel validé de l'accord ACAP 2000 et la compétence définie comme l'intelligence des situations par Bortef, entre les approches ergonomiques et sociologique francophones dans lesquelles la compétence est directement ancrée dans les activités des opérateurs et l'approche psychologique surtout nord-américaine, il y a des nuances importantes. }~\citep[pp.30]{delobbe}
\end{quotation}


\paragraph{}Il ressort que la compétence n'a pas de définition unique: de nombreux chercheurs s'accordent pour dire que la compétence ne peut s'exprimer que dans l'action. Sans action, il n'y a pas de compétence. La compétence est le plus souvent un mélange de savoirs, savoir-faire et de capacités à raisoner. Notre définition dépendra donc du contexte dans lequel nous voulons l'utiliser. L'article Delobbe~\citep[pp.31]{delobbe} lie celle-ci aux caractéristiques de l'organisation qui va l'employer. Nous avons conclu au chapitre précédent qu'Odoo tend vers une structure adhocratique et que la transition vers un modèle individualisant doit s'opérer. Voyons donc quel modèle de gestion de compétence convient le mieux à cette situation.


\section{Recherche des modèles appropriés}
Quatre modèles sont définis dans l'article Dellobe~\citep[pp.39-49]{delobbe}: Le modèle de normalisation, le modèle de polyvalence, le modèle du talent individuel et le modèle de l'expertise. Nous allons maintenant mettre en perspective chaque modèle avec les besoins que nous avons définit plus haut pour déterminer lequel apporterait le plus à Odoo.
\begin{description}
  \item[Le modèle de normalisation]
  Ce modèle favorise l'application de normes qui sont acceptées et valorisées dans l'entreprise. Ce modèle est utile dans le cadre d'une société à croissance rapide ou à la suite de fusion-acquisition. Nous ne cherchons pas du tout à faire appliquer des normes au sein d'Odoo. Ce modèle ne semble donc pas d'application. 
  \item[Le modèle de la polyvalence]
  Un des élements recherché par la gestion des compétences est une meilleur organisation du travail au sein d'une même famille de postes. Hélas, ce modèle ne peut pas convenir. Il est surtout utile pour des métiers peu qualifiés et pour permettre une évolution d'une organisation rigide du travail vers une plus de flexibe. Ici ce n'est pas le cas, nous avons affaire à des travailleurs hautement qualifiés et la flexibilité est déjà présente. Ce modèle n'est pas optimal pour permettre un bon développement de l'équipe. 
  \item[Le modèle du talent individuel]
  Ce modèle convient bien à une structure adhocratique. \textit{"En termes de stratégie, cette approche convient aux entreprises [...] dans lesquel[le]s l'adaptabilité, la réactivité à saisir les opportunités, la capacité à gérer des situations neuves et complexes et à proposer des solutions innovantes sont décisives"}~\citep[pp.44]{delobbe} Cette situation décrit bien une certaine réalité du travail de consultant technique. Les équipes sont constituées en fonction des besoins du projet et des compétences de chacun. Ce modèle définit des compétences décontextualisées qui sont soit des aptitudes intellectuelles soit des aptitudes humaines. Le référentiel ne définit pas les tâches à réaliser ni la manière de les réaliser, ce qui correspond assez bien au contexte de travail chez Odoo. Il y a malgré tout un bémol à ce modèle: il définit le plus souvent des compétences très génériques et décontextualisées. Or, ici, nous avons affaire à une expertise qui est nécessaire pour mener à bien les tâches du poste de consultant. Cette expertise ne peut ni être générique, ni être décontextualisée. 
  \item[Le modèle de l'expertise]
  \textit{"La gestion des compétences a ici pour objectif premier d'affirmer et de développer l'expertise technique interne indispensable à la réalisation des missions de l'entreprise. [...] Les organisations qui visent la prestation d'un service à haute valeur ajoutée [...] basent leur avantage concurrentiel sur la détention de compétences et de connaissances internes pointues, rares, et difficilement imitables"}~\citep[pp.45]{delobbe} En tant qu'éditeur, les services de Odoo S.A. se doivent d'être à la pointe en ce qui concerne le logiciel Odoo. La gestion des compétences et des connaissances doit donc permettre de délivrer ce service de pointe et de le maintenir à niveau tout au long de l'évolution du logiciel. 
\end{description}


\paragraph{}Le modèle qui conviendrait donc au département de service d'Odoo serait un mélange du \textit{modèle du talent individuel} et du \textit{modèle de l'expertise}. La table \ref{model_comp} reprend les caractéristiques des deux modèles retenus. Comme on peut le voir, ces deux modèles possèdent des caractéristiques opposées, l'un se veut générique et décontextualisé, l'autre est très spécifique, adapté au contexte de chaque famille de métier. 

\paragraph{}Le modèle d'Odoo devra intégrer ces deux dimensions quelque peu contradictoires.
La performance du département de service passera bien entendu par le développement des compétences techniques et fonctionnelles, qui doivent sans cesse être renouvelées, et par celui des compétences humaines, nécessaires à la bonne marche des équipes de projet, à la relation client, ainsi qu'à la gestion des équipes grandissantes.



\begin{table}[H]
  \caption{Résumé des caractéristiques des référentiels de compétences selon leur modèle}
  \label{model_comp}

  \begin{center}
    \begin{tabular}{p{0.25\textwidth}p{0.34\textwidth}p{0.33\textwidth}}
      & \textbf{Talent individuel} & \textbf{Expertise} \\
      \hline
      \textbf{Formulation de la compétence} & Aptitutes intellectuelles et comportementales. Laisse une large marge de manoeuvre & Savoir et Savoir-faire techniques et fonctionnels. Des savoir-agir complexes\\
      \textbf{Maille du référentiel} & Assez large, générique et décontextualisé & Etroite, Spécifique\\
      \textbf{Acteurs impliqués}  & Département RH, Cabinet externe &  Les équipes concernées\\
  
    \end{tabular}
  \end{center}
\end{table}

\section{Définition du modèle de gestion de compétence chez Odoo}
Dans les sections précédentes, nous avons déterminé quels étaient les objectifs de la gestion des compétences et quels modèles de gestion allaient nous permettre d'atteindre ces objectifs. Nous allons maintenant synthétiser ces éléments et décrire les caractéristiques du modèle "idéal" de gestion de compétences pour Odoo dans le contexte actuel.\footnote{Compte tenu de la rapide croissance et du changement continu de stratégie, il est impossible de prédire la durée de la validité de cette analyse.} 

\subsection{Définition de la compétence adapté à Odoo}
Nous avons vu précédement que deux modèles assez différents étaient utiles pour atteindre les objectifs fixés. La compétence sera un savoir-agir complexe basée sur des savoirs et savoir-faire techniques et fonctionnels mais aussi des aptitudes comportementales et intellectuelles. 

\subsection{Porté du référentiel ou maille}
Pour le premier groupe de compétence, les savoir-agir complexes, la maille sera assez petite. Cette partie du référentiel sera différente pour chaque famille de rôle: consultant technique, consultant fonctionnel, développeur R\&D. Ce sont les postes actuels. Une définition des compétences techniques et fonctionnelles permettra une meilleur granularité au sein de chaque poste pour définir des profils. Dans le cadre de ce travail, on ne détaillera que les compétences nécessaires au rôle de consultant technique. Il faudra définir un référentiel différent pour chaque famille de postes, même si certains pourraient se recouper. Cette partie devra rendre la complexité de chaque poste. 

\paragraph{}Le problème avec une maille serrée est le nombre de référentiels à mettre à jour au fur et à mesure.
 Toutefois, dans le cas d'Odoo, avec 3 référentiels qui auraient beaucoup en commun, on couvre déjà 76 employés sur les 125. Bien entendu, à l'intérieur de chaque référentiel, il pourra y avoir de nombreux profils mais ceux-ci ne complexifieront pas beaucoup la maintenance. La seconde partie du référentiel sera par contre très large et décontextualisée. Il pourra être commun à tous les rôles et ne nécessitera que peu de maintenance. 

\subsection{Les acteurs impliqués dans la construction}
Pour la partie spécifique à chaque rôle, il faudra impliquer les membres des opérations directement. Ceux-ci seront les plus à même de décrire leur travail, les savoirs et les savoir-faire. Cette situation est pratique vu que l'initiative d'une meilleure gestion des compétences vient de la partie opérationnelle. Pour la partie générique, les choses se compliquent. Généralement, les acteurs impliqués ne sont pas conscients des aptitutes comportementales et intellectuelles qu'ils utilisent et qui sont nécessaires pour effectuer au mieux leur travail. Le référentiel ne pourra pas venir d'eux. Il vient le plus souvent de spécialistes et du département de ressources humaines mais il ne sera pas facile d'impliquer celui-ci. Cependant, il faut remarquer que cette partie est déjà présente dans le formulaire d'évaluation mis au point par le département de ressources humaines. Si l'on met de coté l'absence de référentiel s'y rapportant, il s'agit déjà un point positif.



\subsection{Les objectifs et les usages du référentiel}
Le référentiel aura plusieurs usages. Il permettra d'abord de formaliser la planification du besoin en compétences et d'aider à une meilleur répartition du travail, afin de favoriser le développement de chacun en fonction des besoins. Ensuite, il sera utilisé lors des évaluations qui serviront à faire l'état des lieux des compétences de chacun ainsi qu'à planifier leur développement au sein des équipes du PS. Finalement, cet état des lieux permettra d'objectiver une rémunération individualisée qui, pour le moment, semble arbitraire.



\section{Conclusion}
Dans le premier chapitre, nous avions mis en exergue les problèmes auxquels le département de services au sein d'Odoo faisait face ainsi que les objectifs à atteindre. Dans ce chapitre, nous avons défini les caractéristiques du modèle et du référentiel. Il nous reste à déterminer comment construire ce référentiel et ce qu'il va contenir.
 










 

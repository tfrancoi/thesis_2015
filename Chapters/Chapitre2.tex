\chapter{Vers quelle gestion des compétences}
Maintenant que nous avons fait le tour du problème et que nous avons définit les objectifs, il faut définir quelle type de gestion des compétences sera la plus adapté pour les atteindre et finalement définir la méthodologie et un plan d'action pour sa mise en place. Mais avant toute chose, il faut s'accorder sur la définition de compétence. 

\section{Définition de la compétence}
Avant toutes choses, regardons ce qui se dit à propos de la notion de compétence dans la littérature. Commençons parce ce qu'il en est dit dans le livre de Guérin, Cadin, Pigeyre\citep[pp.170-171]{gestionressourceshumaine2007}
\begin{quotation}
\textit{ On observer une grande variété dans les définitions adoptées, mais toutes retiennent, d'une manière ou d'une autre les mêmes éléments essentiels:
\begin{itemize}
 \item La compétence prend sens para rapport à l'action: on ne peut parler de compétence que dans le cadre précis d'une situation de travail;
 \item Elle combine de façon dynamique les différents éléments qui la constituent (savoirs, savoir-faire pratiques, raisonnements) pour répondre à des exigences d'adaptation.
\end{itemize}}
\end{quotation}

Plus loin, on retrouve encore ce lien entre compétence et action.\citep[pp.172]{gestionressourceshumaine2007}
\begin{quotation}
 \textit{"La compétences est une notion abstraite et hypothétique."}\footnote{Leplat J., ``Compétence et ergonomie'', Modèle en analyse du travail, Bruxelles, Mardaga, 1991, pp. 263-278}
 \textit{On ne peut en observer que les manifestations. [...], c'est à partir de la situation de travail et de la manière dont celle-ci est assumée qu'il est possible d'inférer la compétence.}
\end{quotation}

Et finalement il ne faut pas oublier le caractère socialement construit de la compétence. \citep[pp.249]{gestionressourceshumaine2007}
\begin{quotation}
\textit{[...] c'est le fait de reconnaître une compétence qui la fait exister. Autrement dit, la compétence n'existe pas sans le jugement d'autrui: nul ne saurait se déclarer compétent lui-même.}
\end{quotation}

Nous pouvons observer un point de vue assez similaire dans Aubret, Gilbert\citep[pp.7]{evalcompetence} où la encore il ne s'accorde pas sur une définition de la compétence:

\begin{quotation}
\textit{La notion de compétence se présente souvent comme une notion insaisissable au regard de la diversité des usages. [...]
Le terme de compétence fait partie de ces mots à multiples facettes que personne n'a véritablement le pouvoir de réduire à une seule non équivoque et de l'imposer à tous.
Aussi, nous voyons apparaître dans la littérature sur les compétences de nombreuses définitions qui prennent ce mot, soir comme un terme, soit comme une notion, un concept ou un construit social. [...] R. Zemke (1995), [...], en arrive à la conclusion que la compétence, les compétences, les modèles de compétences et la formation axée sur la compétence sont des mots valises qui signifient seulement ce que l'auteur veut leur faire dire.}

\textit{Ce que disent les chercheurs et praticiens:}
\textit{\begin{itemize}
 \item Compétence: c'est la capacité à résoudre un problème dans un contexte donné (Michel \& Ledru, 1991);
 \item Les compétences sont des ensembles de connaissances, de capacités d'actions et de comportement structurés en fonction d'un but et dans une type de situations données (Gilbert \& Parlier, 1992);
 \item [...]
 \item La compétences est la prise d'initiative et de responsabilité de l'individu sur des situations professionnelles auxquelles il est confronté (Zarifian, 1999)
\end{itemize}}

\end{quotation}

Finalement l'article de Delobbe, Gilbert, Le Boudelaire\citep[pp.30]{delobbe} résume assez bien la complexité de la situation.
\begin{quotation}
\textit{Les définitions de la compétence ont été marquées par des divergences idéologiques qui se sont traduites dans la facon concrète de formuler les référentiels. Entre le savoir-faire opérationnel validé de l'accord ACAP 2000 et la compétence définie comme l'intelligence des situations par Bortef, entre les approches ergonomiques et sociologique francophones dans lesquelles la compétence est directement ancrée dans les activités des opérateurs et l'approche psychologique surtout nord-américaine, il y a des nuances importantes. }\citep[pp.30]{delobbe}
\end{quotation}


\paragraph{}Il ressort que la compétence n'a pas de définition mais beaucoup s'accorde pour dire que la compétence ne peut s'exprimer que dans l'action. Sans action, il n'y a pas de compétence. La compétence est le plus souvent un mélange de savoir, savoir-faire et de capacité à raisoné. Pour construire notre méthode de gestion des compétences, il faudra donc tranché sur une défintion. Cette définition dépendra du contexte dans lequel nous voulons l'utiliser. L'article Delobbe\citep[pp.31]{delobbe} lie celle-ci aux caractéristique de l'organisation qui va l'employer. Nous avons conclus au chapitre précédent qu'Odoo tant vers une structure adhocratique et que la transition vers un modèle individualisant doit s'opérer. Voyons donc quel modèle de gestion de compétence convient le mieux à cette situation.


\section{Recherche des modèles appropriés}
Quatre modèles sont définit dans l'article\citep[pp.39-49]{delobbe}: Le modèle de normalisation, le modèle de polyvalence, le modèle du talent individuel et le modèle de l'expertise. Nous allons maintenant mettre en perspective chaque modèle avec les besoins que nous venons de définir pour déterminer lesquels apporterait le plus à Odoo.
\begin{description}
  \item[Le modèle de normalisation]
  Ce modèle favorise l'application de normes qui sont acceptées et valorisées dans l'entreprise. Ce modèle est utile dans le cadre d'une société à croissance rapide ou à la suite de fusion-acquisition. Nous ne cherchons pas du tout à faire appliquer des normes au sein d'Odoo. Ce modèle ne semble donc pas d'application. 
  \item[Le modèle de la polyvalence]
  Un des élements recherché par la gestion des compétences est une meilleur organisation du travail au sein d'une même famille de poste. Hélas, ce modèle ne peut pas convenir. Il est surtout utiles pour des métiers peux qualifier et pour permettre une transition rigide du travail vers plus de flexibilité. Ici ce n'est pas le cas, nous avons affaire à des travailleurs hautement qualifié et la flexibilité est déjà présente. Elle n'est pas optimales pour permettre bon développement de l'équipe. 
  \item[Le modèle du talent individuel]
  Ce modèle est convient bien à une structure adhocratique. "En termes de stratégie, cette approche convient aux entreprises [...] et dans lesquels l'adaptabilité, la réactivité à saisir les opportunités, la capacité à gérer des situations neuves et complexes et à proposer des solutions innovantes sont décisives"\citep[pp.44]{delobbe} Cette situation décrit bien une certaine réalité du travail de consultant technique. Les équipes sont constituées en fonction des besoins du projet et des compétences de chacun. Ce modèle définit des compétences décontextualisée qui sont soit des aptitues intellectuelles et humaines. Le référentiel ne définit pas les tâches à réaliser ni la manière de les réaliser. Ceci correspond assez bien au réalité du travail chez Odoo. Il y a malgré tout un bémol à ce modèle, il définit le plus souvent des compétences très génériques et décontextualisée, hors ici nous avons affaire à une expertise qui est nécessaire pour mener à bien les tâches du poste de consultant. Cette expertise ne peut pas être générique, ni décontextualisé. 
  \item[Le modèle de l'expertise]
  "La gestion des compétences a ici pour objectif premier d'affirmer et de développer l'expertise technique internet indispensable à la réalisation des missions de l'entreprise. [] dans les organisations qui visent la prestation dun service à haute valeur ajoutée. [..] basent leur avantage concurrentiel sur la détention de compétence et de connaissance internes pointues, rares, et difficilement imitables"\citep[pp.45]{delobbe} En tant qu'éditeur, les service de Odoo S.A. se doivent d'être à la pointe en ce qui concerne le logiciel Odoo. La gestion des compétences et des connaissances doit donc permettre de délivré ce service de pointe et de le maintenir à niveau tout au long de l'évolution du logiciel. 
\end{description}


\paragraph{}Le modèle qui conviendrait donc au département de service d'Odoo serait un mélange du \textit{modèle du talent individuel} et du \textit{modèle de l'expertise}. Le tableau \ref{model_comp} reprend les caractéristiques des deux modèles retenus. Comme on peut le voir, ces deux modèles possèdes des caractéristiques opposées, l'un se veut générique et décontextualisé, l'autre est très spécifiques mis dans le contexte de chaque famille de métier. 

\paragraph{}Le modèle pour Odoo devra intégré ces deux dimensions quelque peu contradictoire.
La performance du département de service passera bien entendu par le développement des compétences techniques et fonctionnelles qui doivent sans cesse être renouvelée mais aussi par celui de compétence plus humaines nécessaires à la bonne marche des équipes de projet, à la relation client ainsi qu'à la gestion des équipes grandissantes.



\begin{table}[H]
  \caption{Résumé des caractéristiques des référentiel de compétences selon leur modèle}
  \label{model_comp}

  \begin{center}
    \begin{tabular}{p{0.25\textwidth}p{0.34\textwidth}p{0.33\textwidth}}
      & \textbf{Talent individuel} & \textbf{Expertise} \\
      \hline
      \textbf{Formulation de la compétence} & Aptitutes intellectuelles et comportemental. Laisse une large marge de manoeuvre & Savoir et Savoir-faire technique et fonctionnels. Des savoir-agir complexe\\
      \textbf{Maille du référentiel} & Assez large, générique et décontextualisé & Etroite, Spécifique\\
      \textbf{Acteurs impliqués}  & Département HR, Cabinet externe &  Les équipes concernées\\
  
    \end{tabular}
  \end{center}
\end{table}

\section{Définition du modèle de gestion de compétence chez Odoo}
Dans les sections précédentes nous avons déterminés quels étaient les objectifs de la gestion des compétences, ensuite nous avons déterminés quels sont les modèles de gestion qui vont permettre d'atteindre ces objectifs. Nous allons maintenant synthétiser ces éléments et décrire les caractéristiques du modèle "idéal" de gestion de compétences pour Odoo dans le contexte actuel.\footnote{Compte tenu de la rapide croissance et du changement continu de stratégie, il est impossible de prédire la durée de la validité de cette analyse.} 

\subsection{Définition de la compétence}
Nous avons vu précédement que deux modèles assez différent étaient utiles pour atteindre les objectifs fixés. La compétence sera un savoir-agir complexe basée sur des savoir et savoir-faire techniques et fonctionnels mais aussi des aptitudes comportementales et intellectuelles. 

\subsection{Porté du référentiel ou maille}
Pour le premier groupe de compétence, les savoir-agir complexes, la maille sera assez petite. Cette partie du référentiel sera différente pour chaque famille de rôle: consultant technique, consultant fonctionnel, développeur R\&D. Ce sont les postes actuelles, une défintion des compétences techniques et fonctionnelles permettra une meilleur granularité au sein de chaque poste définir des profils. Dans le cadre de se travail on ne détaillera que les compétences nécessaire au rôle de consultant technique. Il faudra définir un référentiel différent pour chaque famille de poste, même si certaine partie pourrait se recouper. Cette partie devra rendre la complexité de chaque poste. 

\paragraph{}Le problème avec une maille serré est le nombre de référentiel à maintenir mais dans le cas d'odoo avec 3 référentiels qui auront déjà beaucoup de chose en commun on couvre 76 employés sur 125. Bien entendu à l'intérieur de chaque référentiel il pourra y avoir de nombreux profils mais ceux-ci ne complexifiront pas beaucoup la maintenance. La seconde partie du référentiel sera par contre très large et décontextualisé. Il pourra être commun à tous les rôles et ne nécessitera que peut de maintenance. 

\subsection{Les acteurs impliqués dans la construction}
Pour la partie spécifique à chaque rôle, il faudra impliqués les membres des opérations directement. Ceux-ci seront les plus à même de décrire leur travail, les savoirs et les savoir-faire. Cette situation est pratique vu que l'initiative d'une meilleur gestion des compétences vient de la partie opérationnelle. Pour la partie générique, les choses se compliquent. Généralement, les acteurs impliqués ne sont pas conscient des aptitutes comportementaux et intellectuelles qu'ils utilisent et qui sont nécessaires pour effectuer au mieux leur travail. Le référentiel ne pourra pas venir d'eux. Il vient le plus souvent de spécialiste et du département de ressource humaine et il ne sera pas facile d'impliqué celui-ci. Il faut quand-même remarquer que cette partie est déjà présent dans le formulaire d'évaluation mis au point par le département de ressources humaines, si on met de coté l'absence du référentiel qui s'y rapporte, c'est déjà une bonne choses.



\subsection{Les objectifs et les usages du référentiel}
Il permettra de formaliser la planification du besoin en compétence, d'aidé à une meilleur répartition du travail afin de favoriser le développement de chacun en fonction des besoins. Ensuite ce référentiel sera utiliser lors des évaluations qui servirons à faire l'état des lieux des compétences de chacun ainsi qu'a planifié le développement de celle-ci au sein des équipes du PS. Finalement cet état des lieux permettra d'objectiver une rémunération individualisé qui pour le moment semble arbitraire.



\section{Conclusion}
Dans le chapitre précédent, nous avions définit les problèmes auxquels le département de service au sein d'Odoo faisait face. Nous avons aussi définit les objectifs à atteindre. Dans ce chapitre nous avons définit les caractéristiques du modèle et du référentiel. Maintenant, il reste à savoir comment nous allons construire ce référentiel et ce qu'il va contenir.
 










 

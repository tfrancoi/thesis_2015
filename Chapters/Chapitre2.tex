\chapter{Vers quelle gestion des compétences }
Maintenant que nous avons fait le tour du problème, il faut définir les objectifs à atteindre en s'aidant de la structure de l'organisation. Ensuite il faut définir quelle type de gestion des compétences sera la plus adapté pour atteindre les objectifs et finalement définir la méthodologie et un plan d'action pour sa mise en place. Mais avant toute chose, il faut s'accorder sur la définition de compétence. 

\section{Définition de la compétence}
Avant toutes choses, regardons ce qui se dit à propos de la notion de compétence dans la littérature. Commençons parce ce qu'il en est dit dans le livre de Guérin, Cadin, Pigeyre\citep[pp.170-171]{gestionressourceshumaine2007}
\begin{quotation}
\textit{ On observer une grande variété dans les définitions adoptées, mais toutes retiennent, d'une manière ou d'une autre les mêmes éléments essentiels:
\begin{itemize}
 \item La compétence prend sens para rapport à l'action: on ne peut parler de compétence que dans le cadre précis d'une situation de travail;
 \item Elle combine de façon dynamique les différents éléments qui la constituent (savoirs, savoir-faire pratiques, raisonnements) pour répondre à des exigences d'adaptation.
\end{itemize}}
\end{quotation}

Plus loin, on retrouve encore ce lien entre compétence et action.\citep[pp.172]{gestionressourceshumaine2007}
\begin{quotation}
 \textit{"La compétences est une notion abstraite et hypothétique."}\footnote{Leplat J., ``Compétence et ergonomie'', Modèle en analyse du travail, Bruxelles, Mardaga, 1991, pp. 263-278}
 \textit{On ne peut en observer que les manifestations. [...], c'est à partir de la situation de travail et de la manière dont celle-ci est assumée qu'il est possible d'inférer la compétence.}
\end{quotation}

Et finalement il ne faut pas oublier le caractère socialement construit de la compétence. \citep[pp.249]{gestionressourceshumaine2007}
\begin{quotation}
\textit{[...] c'est le fait de reconnaître une compétence qui la fait exister. Autrement dit, la compétence n'existe pas sans le jugement d'autrui: nul ne saurait se déclarer compétent lui-même.}
\end{quotation}

Nous pouvons observer un point de vue assez similaire dans Aubret, Gilbert\citep[pp.7]{evalcompetence} où la encore il ne s'accorde pas sur une définition de la compétence:

\begin{quotation}
\textit{La notion de compétence se présente souvent comme une notion insaisissable au regard de la diversité des usages. [...]
Le terme de compétence fait partie de ces mots à multiples facettes que personne n'a véritablement le pouvoir de réduire à une seule non équivoque et de l'imposer à tous.
Aussi, nous voyons apparaître dans la littérature sur les compétences de nombreuses définitions qui prennent ce mot, soir comme un terme, soit comme une notion, un concept ou un construit social. [...] R. Zemke (1995), [...], en arrive à la conclusion que la compétence, les compétences, les modèles de compétences et la formation axée sur la compétence sont des mots valises qui signifient seulement ce que l'auteur veut leur faire dire.}

\textit{Ce que disent les chercheurs et praticiens:}
\textit{\begin{itemize}
 \item Compétence: c'est la capacité à résoudre un problème dans un contexte donné (Michel \& Ledru, 1991);
 \item Les compétences sont des ensembles de connaissances, de capacités d'actions et de comportement structurés en fonction d'un but et dans une type de situations données (Gilbert \& Parlier, 1992);
 \item [...]
 \item La compétences est la prise d'initiative et de responsabilité de l'individu sur des situations professionnelles auxquelles il est confronté (Zarifian, 1999)
\end{itemize}}

\end{quotation}

Finalement l'article de Delobbe, Gilbert, Le Boudelaire\citep[pp.30]{delobbe} résume assez bien la complexité de la situation et résume les différentes approche dans le tableau \ref{def_comp}.
\begin{quotation}
\textit{Les définitions de la compétence ont été marquées par des divergences idéologiques qui se sont traduites dans la facon concrète de formuler les référentiels. Entre le savoir-faire opérationnel validé de l'accord ACAP 2000 et la compétence définie comme l'intelligence des situations par Bortef, entre les approches ergonomiques et sociologique francophones dans lesquelles la compétence est directement ancrée dans les activités des opérateurs et l'approche psychologique surtout nord-américaine, il y a des nuances importantes. }
\end{quotation}


\begin{table}
  \caption{Définitions et approches opérationnelles de la compétence\citep[pp.31]{delobbe}}
  \label{def_comp}

  \begin{center}
    \begin{tabular}{p{0.25\textwidth}|p{0.35\textwidth}|p{0.4\textwidth}}
       & \textbf{Contextualisation forte: compétences specifiques} & \textbf{Contextualisation faible: compétenes génériques}\\
       \hline
      \textbf{Prescription forte: contrôle externe des comportements}  & Savoir-faire élémentaires, gestes professionnels prédéfinis et prescrits (Accords ACAP 2000) & Normes de comportements et savoir-être partagés, traduisant l'adéquation aux valeurs plus ou moins explicite de l'entreprise (Schippman et al., 2000\\
      \hline
      \textbf{Prescription faible: autonomie accrue des travailleurs} & Savoir-agir complexes et autonomes en situation incertaine (Le Boterf, 1997; Zarifian, 2001)  & Connaissances, aptitudes, capacités et toutes caractéristiques psycologiques individuelles associées à un niveau élevé de performance (Boyatzis, 1982; McClelland, 1973; Spencer et Spencer, 1993)\\
    \end{tabular}
  \end{center}
\end{table}

\paragraph{}Il ressort que la compétence n'a pas de définition mais beaucoup s'accorde pour dire que la compétence ne peut s'exprimer que dans l'action. Sans action, il n'y a pas de compétence. La compétence est le plus souvent un mélange de savoir, savoir-faire et de capacité à raisoné. Pour construire notre méthode de gestion des compétences, il faudra donc tranché sur une défintion. Cette définition dépendra du contexte dans lequel nous voulons l'utiliser. Le tableau \ref{def_comp} lie celle-ci aux caractéristique de l'organisation qui va l'employer. C'est pour cela qu'avant de définir notre notion de compétence, nous allons nous atteler à définir la structure organisationelle d'Odoo.  

\section{Étude de la configuration organisationelle d'Odoo}
\subsection{La configuration en 2010}
Il est interressant de revenir à la configuration d'Odoo en 2010, juste après la première levée de fond. Comme on peut le voir dans le tableau \ref{nb_employe}, au début de l'année 2010 il y avait 18 employé et à la fin de l'année il y en avait déjà 34. Ce nombre reste faible comparé au 125 employés en belgique actuellement. A cette époque le sommet hierarchique était composé d'un CEO-fondateur\footnote{Chief Executive Officer}, CSO\footnote{Chief Sales Officer}, COO\footnote{Chief Operating Officer} et d'un CTO\footnote{Chief Technical Officer}. Il y avait déjà trois département, tous présent sur le même site. Le département de recherche et développement gérer par le CTO, le département de vente gérer par le CSO et le département de service en théorie gérer par le COO. En dehors du sommet hiérchique, il n'y avait pas de responsable d'équipe. Les employés sont depuis le début très qualifié: des ingénieurs en informatique en R\&D et au département de service et des ingénieurs des gestions au département de service et à la vente. Au sein de chaque département, le travail était intercheangable entre chaque membre d'un département. En R\&D et au PS\footnote{Abréviation pour les département de service: Professional Services}, chacun travaillait sur son projet et il est difficile de changer l'assignation en cours de route, mais toute nouvelle tâche était suceptible d'être assigné à quiconque. Le CEO passait presque quotidiennement voir qui faisait quoi pour donner quelque ajustement, débloquer une situation. Toutes les décisions était prise par le comité de direction composé des quatres membres exécutifs mais bien entendu le CEO avait toujours le dernier mots. C'est à cette époque le beau-frère de celui-ci fut envoyé pour ouvrir un bureau aux états-unis.  

\newpage
\paragraph{}Ci-dessous un résumé la situation au niveau de la gestion ressources humaines chez Odoo en 2010.

\begin{description}
    \item[Planification] Il nous manque des informtations pour pouvoir juger de la planification au niveau RH. Il semble que dans un contexte de croissance, le recrutement était ouvert pout les trois département.
    \item[Sélection] La sélection se faisait via une interview avec le directeur concerné ou alors directement avec le CEO. Dans un premier temps les interview était tout à fait informelle. Pour les postes techniques des exercices de programmation on été mis en place à la fin de 2010
    \item[Formation] Une semaine de formation était donné à tout les employés à leur arrivées. Une semaine supplémentaire est donné au profils techniques. Ces deux semaines de formation était dispensé car elle étaient vendues et prestées pour nos partenaires. Rien de spécifique à chaque poste n'existe. Pour cela il fallait se former sur le temps. Généralement, la méthode de formation consistait à jetté les nouveaux employés tout habillé dans la piscine.
    \item[Evaluation] La notion d'évaluation n'était pas du tout formaliser, elle se faisait à le demande de l'employé principalement. A cette époque les seules personnent qui furent congédiée appartenaient au département de vente lorsque ceux-ci avaient de mauvais chiffres. En générale sans prester le moindre préavis. Les personnes qui démissionnait ne prestaient aussi que très rarement leur préavis. La démission étant perçu parfois comme une trahisons de la part de la direction.
    \item[Rémunération] La rémunération dépendait de l'humeur du CEO lors de l'entretien de sélection. Ensuite, les augmentations se faisait plus à la demande de l'employé lorsque celui-ci considérait qu'il en méritait une. Il n'y avait aucune règle pour les augmentation.
    \item[Promotion] A cette époque il n'existait aucune possibilité de promotion, seul la croissance donnait l'espoir d'avoir une promotion un jour. Cela n'empêchait par contre en rien de voir sa rémunération augmentée. 
\end{description}

 

\paragraph{}Si regarde cette configuration et le modèle de GRH au configuration en lumière de la théorie de Pichault et Nizet \citep[pp. 48-49]{pichault}.
La configuration était à cette époque principalement entrepreneuriale: L'autorité du fondeur est grande. Une grande division du travail vertical mais faible au niveau horizontal, en tout cas intra département. La coordinatation du travail se fait via la supervision directe du CEO et du directeur. On peut aussi noter l'implication des membres de la famille. Malgré tout, il y a déjà des signes d'une configuration adhocratique: Employé très qualifié, une organisation par projet principalement en R\&D et au PS. 

\paragraph{}Par contre le modèle de GRH est clairement arbitraire\citep[pp. 115-119]{pichault}. Congédie sur le champs. Nous n'en avons pas parler mais à cette époque le petit nombre d'employé favorisait l'esprit maison avec des nombreux verre organisé. En R\&D, il n'était pas rare de se réveiller chez un collègue les lendemains de veille avec le CTO dans le canapé d'à coté. La sélection et les évaluations était sur un mode intuitif.  La formation se fait sur le tas et les promotions était presque inexistantes. 

\subsection{La configuration à l'heure actuelle}
\paragraph{} Le modèle de GRH arbitraire fonctionnait assez bien dans une petite société de 20 employés dans un configuration entrepreneuriale. Mais cette configuration à bien évoluée en cinq année alors que la société passait de 18 employés à 125 employés. 

\paragraph{} Comme évoqué déjà au chapitre 1, les départements se sont structurés en équipe. Le COO a été remplacé par un directeur du PS. Un département Marketing et financié se sont rajoutés à l'ensemble. L'autonomie de chaque équipe à grandie, même si il y a toujours quelque coup de sonde et parfois un contournement de la ligne hiérarchique de la part du CEO. Par contre les décisions stratégique reste toujours entre les mains du comité de direction, emputé de son COO, mais avec deux nouveaux membres: le responsable marketing et le responsable financier. La communication entre les équipes se sont structurées : système de tickets. Au département de services, ils y a deux type de profils: Fonctionnel et Technique. Les équipes formé des deux profils se forme et se déforme au fils des projets. Ses équipes sont assez autonome. Des tensions sont apparues entre le département de ventes et le département de service, le premier ayant des objectifs de chiffre d'affaire et le second des objectifs de qualité de service et de rentabilité.

\paragraph{} Avec la croissance du nombre de client, le support à prit une place stratégique au sein du PS. Mais le support n'est pas gérer par une équipe dédié, celui-ci tourne entre les personnes. Cela présente deux avantages, le support étant perçu comme une tâche ingrate, on évite un turnover important qu'il pourrait y avoir dans le cas d'une équipe dédié. Le support touche à tous les aspects opérationnel d'Odoo, il a donc un grand pouvoir formateur dont tout le monde se doit de bénéficier. Mais cette configuration pose aussi des problèmes, celui de la standardisation de la qualité et des procédures. Des processus plus standardisés apparaissent aussi au niveau de la vente avec l'appui du logiciel Odoo. 

\paragraph{} Nous pouvons observer que la configuration entrepreneuriale à laisser place à une configuration adhocratique\citep[pp. 53-54]{pichault} avec une forte décentralisation du pouvoir pour les questions opérationnelles, mais toujours une fortes centralisation pour les décisions stratégique. Il y a aussi une petite tendance à la bureaucracie pour les tâches plus répétitives comme le support, la création de contracts. Maintenant faisons le point de l'évolution de la politique des ressources humaines. 

\begin{description}
  \item[Planification] La planification des ressources se fait toujours département par département. A l'heure actuelle, un gel total du recrutement est opéré sans tenir compte des besoins d'aucun département dans le but d'atteindre le seuil de rentabilité. Les départs sont maintenant mieux gérer ceux-ci sont arrangé pour que l'employé puisse confier ses responsabilité à un collègue.
  \item[Sélection] La sélection s'opère maintenant sur base de test adapté pour chaque poste. Les tests sont évalué avant l'entretien avec le futur responsable. 
  \item[Formation] Concernant la formation, rien n'a bouger. Seuls les deux semaines de formations sont offertes et ensuite chaque équipe coache ses nouveaux venus. 
  \item[Evaluation] L'évaluation a été décrite très largement dans le chapitre 1, c'est un des points noires de la gestion de ressources humaines. Il y a eu un début de formalisme mais qui n'a pas apporté grand chose. Il y malgré tout une définition des objectifs de chacun, qui pour le moment ne sont pas suivit en tout cas au PS. 
  \item[Rémunération] Au niveau de la rémunération, une grille salariale a été introduite. Elle permet au nouveau arrivant d'être sur un pied d'égalité. Mais pour ce qui est des augmentation cela reste très intuitif et très peu objectif.
  \item[Promotion] Il y a eu des promotions mais très peu et la croissance de la société reste toujours le meilleur espoir de promotion. Mais ce n'est pas automatique. La R\&D est restée très longtemps avec un seul responsable le CTO malgré ses 40 membres. Ce n'est que très récement que celle-ci c'est doté de responsable.  
\end{description}

\paragraph{} On retrouve toujours beaucoup de caractéristique du modèle arbitraire. On retrouve malgré tout déjà quelque élément du modèle individualisant. Chez les commerciaux la rémunération est variable. La sélection ce fait sur base de compétence vérifiées par des tests lors de l'embauche. L'évaluation détermine des objectifs qui devrait être suivis et évalués à la fin de la période.


\paragraph{} Odoo présente quelque signe d'un modèle individualisant. La gestion des compétences doit permettre d'achevé la transition d'un modèle arbitraire qui ne convient plus à la taille de l'entreprise vers un modèle individualisant plus adapté à la nouvelle structure adhocratique. Les aspects les plus important que devra permettre cette gestion des compétences, sont la planification, la formation et l'évaluation. Il faudra pouvoir planifié de manière plus structuré les compétences nécessaire au sein de chaque équipe. Mais pour pouvoir planifié les besoins, il faudra faire un état des lieux de l'existant et ensuite décidé de la manière d'acquérir compétences manquante, soit par le recrutement soit par la formation. La formation devra se faire en fonction des besoins planifiés. Les évaluations devront permettre d'établir l'état des lieux des compétences existantes mais aussi de poussé les employés a apprendre les compétences manquantes au sein de leur équipe. 

\paragraph{} Une fois cette gestion mis en place, il sera plus facile d'objectivé la rémunération basée sur les compétences de chacun. Et finalement, il sera aussi plus facile d'envisager une mobilité horizontale et verticale. 

\paragraph{} Maintenant que nous avons clarifié ce que nous attendions de la gestion des compétences chez Odoo et plus particulièrement au PS. Nous pouvons revenir à la définition des compétences que nous voulons gérer et à la manière dont nous allons les gérer. 

 

\chapter{Vers quelle gestion des compétences }
Maintenant que nous avons fait le tour du problème, il faut définir les objectifs à atteindre en s'aidant de la structure de l'organisation. Ensuite il faut définir quelle type de gestion des compétences sera la plus adapté pour atteindre les objectifs et finalement définir la méthodologie et un plan d'action pour sa mise en place. Mais avant toute chose, il faut s'accorder sur la définition de compétence. 

\section{Définition de la compétence}
Avant toutes choses, regardons ce qui se dit à propos de la notion de compétence dans la littérature. Commençons parce ce qu'il en est dit dans le livre de Guérin, Cadin, Pigeyre\citep[pp.170-171]{gestionressourceshumaine2007}
\begin{quotation}
\textit{ On observer une grande variété dans les définitions adoptées, mais toutes retiennent, d'une manière ou d'une autre les mêmes éléments essentiels:
\begin{itemize}
 \item La compétence prend sens para rapport à l'action: on ne peut parler de compétence que dans le cadre précis d'une situation de travail;
 \item Elle combine de façon dynamique les différents éléments qui la constituent (savoirs, savoir-faire pratiques, raisonnements) pour répondre à des exigences d'adaptation.
\end{itemize}}
\end{quotation}

Plus loin, on retrouve encore ce lien entre compétence et action.\citep[pp.172]{gestionressourceshumaine2007}
\begin{quotation}
 \textit{"La compétences est une notion abstraite et hypothétique."}\footnote{Leplat J., ``Compétence et ergonomie'', Modèle en analyse du travail, Bruxelles, Mardaga, 1991, pp. 263-278}
 \textit{On ne peut en observer que les manifestations. [...], c'est à partir de la situation de travail et de la manière dont celle-ci est assumée qu'il est possible d'inférer la compétence.}
\end{quotation}

Et finalement il ne faut pas oublier le caractère socialement construit de la compétence. \citep[pp.249]{gestionressourceshumaine2007}
\begin{quotation}
\textit{[...] c'est le fait de reconnaître une compétence qui la fait exister. Autrement dit, la compétence n'existe pas sans le jugement d'autrui: nul ne saurait se déclarer compétent lui-même.}
\end{quotation}

Nous pouvons observer un point de vue assez similaire dans Aubret, Gilbert\citep[pp.7]{evalcompetence} où la encore il ne s'accorde pas sur une définition de la compétence:

\begin{quotation}
\textit{La notion de compétence se présente souvent comme une notion insaisissable au regard de la diversité des usages. [...]
Le terme de compétence fait partie de ces mots à multiples facettes que personne n'a véritablement le pouvoir de réduire à une seule non équivoque et de l'imposer à tous.
Aussi, nous voyons apparaître dans la littérature sur les compétences de nombreuses définitions qui prennent ce mot, soir comme un terme, soit comme une notion, un concept ou un construit social. [...] R. Zemke (1995), [...], en arrive à la conclusion que la compétence, les compétences, les modèles de compétences et la formation axée sur la compétence sont des mots valises qui signifient seulement ce que l'auteur veut leur faire dire.}

\textit{Ce que disent les chercheurs et praticiens:}
\textit{\begin{itemize}
 \item Compétence: c'est la capacité à résoudre un problème dans un contexte donné (Michel \& Ledru, 1991);
 \item Les compétences sont des ensembles de connaissances, de capacités d'actions et de comportement structurés en fonction d'un but et dans une type de situations données (Gilbert \& Parlier, 1992);
 \item [...]
 \item La compétences est la prise d'initiative et de responsabilité de l'individu sur des situations professionnelles auxquelles il est confronté (Zarifian, 1999)
\end{itemize}}

\end{quotation}

Finalement l'article de Delobbe, Gilbert, Le Boudelaire\citep[pp.30]{delobbe} résume assez bien la complexité de la situation et résume les différentes approche dans le tableau \ref{def_comp}.
\begin{quotation}
\textit{Les définitions de la compétence ont été marquées par des divergences idéologiques qui se sont traduites dans la facon concrète de formuler les référentiels. Entre le savoir-faire opérationnel validé de l'accord ACAP 2000 et la compétence définie comme l'intelligence des situations par Bortef, entre les approches ergonomiques et sociologique francophones dans lesquelles la compétence est directement ancrée dans les activités des opérateurs et l'approche psychologique surtout nord-américaine, il y a des nuances importantes. }
\end{quotation}


\begin{table}
  \caption{Définitions et approches opérationnelles de la compétence\citep[pp.31]{delobbe}}
  \label{def_comp}

  \begin{center}
    \begin{tabular}{p{0.25\textwidth}|p{0.35\textwidth}|p{0.4\textwidth}}
       & \textbf{Contextualisation forte: compétences specifiques} & \textbf{Contextualisation faible: compétenes génériques}\\
       \hline
      \textbf{Prescription forte: contrôle externe des comportements}  & Savoir-faire élémentaires, gestes professionnels prédéfinis et prescrits (Accords ACAP 2000) & Normes de comportements et savoir-être partagés, traduisant l'adéquation aux valeurs plus ou moins explicite de l'entreprise (Schippman et al., 2000\\
      \hline
      \textbf{Prescription faible: autonomie accrue des travailleurs} & Savoir-agir complexes et autonomes en situation incertaine (Le Boterf, 1997; Zarifian, 2001)  & Connaissances, aptitudes, capacités et toutes caractéristiques psycologiques individuelles associées à un niveau élevé de performance (Boyatzis, 1982; McClelland, 1973; Spencer et Spencer, 1993)\\
    \end{tabular}
  \end{center}
\end{table}

\paragraph{}Il ressort que la compétence n'a pas de définition mais beaucoup s'accorde pour dire que la compétence ne peut s'exprimer que dans l'action. Sans action, il n'y a pas de compétence. La compétence est le plus souvent un mélange de savoir, savoir-faire et de capacité à raisoné. Pour construire notre méthode de gestion des compétences, il faudra donc tranché sur une défintion. Cette définition dépendra du contexte dans lequel nous voulons l'utiliser. Le tableau \ref{def_comp} lie celle-ci aux caractéristique de l'organisation qui va l'employer. C'est pour cela qu'avant de définir notre notion de compétence, nous allons nous atteler à définir la structure organisationelle d'Odoo.  

\section{Structue organisationelle d'Odoo}
\subsection{De la structure entrepreneurial à une configuration adhocratique}
Il est interressant de revenir à la configuration d'Odoo en 2010, juste après la première levée de fond. Comme on peut le voir dans le tableau \ref{nb_employe}, au début de l'année 2010 il y avait 18 employé et à la fin de l'année il y en avait déjà 34. Ce nombre reste faible comparé au 125 employés en belgique actuellement. A cette époque le sommet hierarchique était composé d'un CEO, CSO, COO et d'un CTO. Il y avait déjà trois département, tous présent sur le même site. Le département de recherche et développement, le département de vente et le département de service. En dehors du sommet hiérchique, il n'y avait pas de responsable d'équipe. Les employés sont depuis le début très qualifié: des ingénieurs en informatique en R\&D et au département de service et des ingénieurs des gestions au département de service et à la vente. Au sein de chaque département, le travail est intercheangable entre chaque membre d'un département. La R\&D


Si on compare au caractétistique du de la configuration entrepreneuriale de Pichault et Nizet \citep[pp. 48-49]{pichault} 

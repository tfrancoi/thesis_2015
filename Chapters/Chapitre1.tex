\chapter{État des lieux et définition des objectifs}
\paragraph{}Odoo S.A, située en brabant wallon, est une société d’édition d"applications de gestion d'enterprise. Depuis sa fondation en 2005, elle vie une croissance hors pair : de 2  à 285 employés en 2015. Depuis 2010, le nombre d'employé en Belgique est passé de 18 à 125\footnote{Cette information est tirée du logiciel de gestion des ressource humaines interne consulté le 8 juillet}. Cette croissance est d'abord dû a l’aspect innovant des logiciels édités, ils sont Open Source, sous licence AGPL\footnote{\enquote{GNU Affero General Public License (ou AGPL, signifiant licence publique générale Affero) est une licence libre dérivée de la Licence publique générale GNU avec une partie supplémentaire couvrant les logiciels utilisés sur le réseau. Elle a été écrite par Affero pour autoriser les droits garantis par la GPL à couvrir les interactions avec des produits à travers un réseau comme Internet, ce que la GPL ne fait pas.}\cite{agpl}}, intégrés entre eux à l'image des ERP\footnote{Enterprise Resource Planning\cite{wikierp}} et permettent, contrairement à la compétition, une adaptation rapide et compétitive aux besoins des clients. Ensuite à une approche commerciale centré sur les PME sans contrainte géographie. Ces atouts ont permis deux levé de fonds en 2010 et en 2014. La première a permis de passé de 18 employés à 34 employés et la seconde de 64 à 111 employés.  

\paragraph{}Les applications ne portant pas de couts de licence, Odoo génère ses revenues des services d’implémentation de ses applications auprès de ses clients, des services de maintenance, de support et les migrations vers des versions supérieurs. Les applications sont rendu disponible sur une platforme mutualisée ou SAAS\footnote{Software as a Service}. Odoo dévlivre aussi du support et de la formation à son réseau de partenaire qui s'occupe du marché des grandes entreprises. 


\paragraph{}Les produits et activités cités ci-dessus sont assurés par trois activités principales. Le développement du logiciel et de la platforme mutualisée par le département de recherche et développpement\footnote{Par la suite, il sera abrégé par R\&D}, le département de vente et marketing assure la stratégie commerciale et la ventes de services et finalement le département de service  ou PS\footnote{Abréviation de Professional Services} assure les services aux clients: services avant-vente, services d’analyse des besoins et de faisabilité, de mise en oeuvre à travers des projets, de maintenance, de support. 

\paragraph{}Au PS, il y a deux équipes. D'une part l'équipe fonctionnelle, composé de consultants fonctionnels pour la plupart ingénieur de gestion et d'un responsable, elle des missions d'analyse, d'implémentation, de gestion de projet et de support. D'autre part l'équipe technique composée de consultants techniques, tous avec une solide formation en informatique et un responsable d'équipe, elle s'occupe de l'analyse de faisabilité, de l'implémentation des adaptation du logiciel aux besoins clients, du support technique et de la formation. Depuis 2010, le département de service est passée de 5 personnes à 36 personnes à l'heure actuelle. La croissance fut chaotique avec de nombreux départs, les deux équipes furent crées en court de route, la séparation entre poste fonctionel et technique également et finalement la délégation par le directeur du département d'une partie de ces responsabilités à deux "team leader" date de moins d'un an. Mais en dehors de ca rien n'a été fait pour accompagner la croissance de ce département. En tant que responsable d’équipe technique au PS , j’ai le privilège de pouvoir accéder à une partie des informations permettant de faire un grand nombre de constats ainsi que de pouvoir formuler des solutions et de les mettre en oeuvre c'est pourquoi le présent travail se focalisera principalement sur ce département.


\paragraph*{}La croissance a engendré de nombreux problème et défits. L'intégration des nouveaux employés dans la culture de l'entreprise, leur formation. Il faut aussi veuiller à garder une qualité de service constante malgré la disparité des individus qui grandit. Proposer un chemin d'évolution dans l'entreprise malgré le nombre de postes très restreint au sein de l'entreprise et finalement une meilleur organisation du travail qui est pour le moment basé principalement sur la disponibilité, afin d'assurer un résultat optimal et le développement de chacun. Pour répondre à ces nombreux défits nous avons choisit de nous aider de la gestion des compétences.

\section{Objectifs visés par la gestion des compétences}
D'après le livre\citep{gestionressourceshumaine2007} et l'article \citep{delobbe}, la gestion des compétences a plusieurs objectifs dont voici une liste non exhaustive.
\begin{itemize}
    \item \enquote{Elle a pour effet d'assurer l'adhésion des salariés à des normes de comportements considérées comme acceptables et valorisées au sein d'une organisation}\citep[p.40]{delobbe}
    \item Elle permet d'adapter les comportements des salariés aux attentes des clients dans une logique de service et de qualité. \citep[182]{gestionressourceshumaine2007}
    \item \enquote{Gérer la polyvalence et la flexibilité... Elle vise à dépasser une allocation rigide des effectifs à des postes de travail délimités}\citep[p.41]{delobbe}
    \item Elle favorise une redistribution des tâches et des responsabilités en cas de réduction d'effectifs \citep[182]{gestionressourceshumaine2007}
    \item  \enquote{L'objectif de la gestion des compétences est donc d'identifier, de sélectionner, de promouvoir et de récompenser les plus talentueux..}\citep[p.43]{delobbe}
    \item  \enquote{La gestion des compétences a ici pour objectifs premier d'affirmer et de développer l'expertise technique interne indispensable à la réalisation des missions de l'entreprise.} \citep[p.45]{delobbe}

\end{itemize}

\paragraph{}La gestion des compétences peut être utilisé à tous le niveau des ressources humaines: Les évaluations et la rémunération du personnel, la gestion des performances, la classification des emplois, l'allocation des effectifs, la sélection des candidats, la formation des employés et la gestion des carrières. \citep[p.32]{delobbe}. Normalement, la gestion des compétences devrait s'inscrire dans une démarche stratégique. Malheureusement cette mise en place de la gestion des compétences ne venant pas du sommet hiérarchique, il ne sera pas possible d'inscrire cette démarche dans la stratégie de l'entreprise. Un autre problème est le manque de stratégie à moyen et long terme.  La gestion des compétences s'inscrira donc dans la limite du département de service et commencera avec l'équipe des consultants techniques. 

\paragraph{}Avant de ce lancer tête baissé dans la défintion de la gestion des compétences qui va être mis en place chez Odoo, arrêtons nous un instance. La conclusion de l'article\citep{delobbe} suggère que le type de gestion des compétences est corréler avec le type de la structure organisationelle de l'entreprise qui la met en place. Attardons nous donc un instant sur la structure organisationelle de l'entrerprise et de son évolution ainsi que l'évolution du modèle de gestion des ressources humaines. Une fois cet état des lieux fait, nous pourrons déterminer quelle gestion des compétences est la plus appropriée.
        



\section{Structure organisationnelle et politique de gestion des ressources humaines au sein d'Odoo}
\subsection{La configuration en 2010}
Il est interressant de revenir à la configuration d'Odoo en 2010, juste après la première levée de fond. Comme on peut le voir dans le tableau \ref{nb_employe}, au début de l'année 2010 il y avait 18 employé et à la fin de l'année il y en avait déjà 34. Ce nombre reste faible comparé au 125 employés en belgique actuellement. A cette époque le sommet hierarchique était composé d'un CEO-fondateur, CSO, COO et d'un CTO\footnote{Respectivement Chief Executive Officer, Chief Sales Officer, Chief Operating Officer, Chief Technical Officer}. Il y avait déjà trois département, tous présent sur le même site. Le département de recherche et développement gérer par le CTO, le département de vente gérer par le CSO et le département de service en théorie gérer par le COO. En dehors du sommet hiérchique, il n'y avait pas de responsable d'équipe. Les employés sont depuis le début très qualifié: des ingénieurs en informatique en R\&D et au département de service et des ingénieurs des gestions au département de service et à la vente. Au sein de chaque département, le travail était intercheangable entre chaque membre d'un département. En R\&D et au PS\footnote{Abréviation pour les département de service: Professional Services}, chacun travaillait sur son projet et il est difficile de changer l'assignation en cours de route, mais toute nouvelle tâche était suceptible d'être assigné à quiconque. Le CEO passait presque quotidiennement voir qui faisait quoi pour donner quelque ajustement, débloquer une situation. Toutes les décisions était prise par le comité de direction composé des quatres membres exécutifs mais bien entendu le CEO avait toujours le dernier mot. C'est à cette époque le beau-frère de celui-ci fut envoyé pour ouvrir un bureau aux états-unis. Au sein du PS l'organisation du travail se faisait de manière très simple on choisissait la personne la plus compétente parmis les personnes disponible, sachant que personnes n'était jamais vraiment disponible on choisissait simplement la plus disponible. Le paragraphe suivant résume la situation au niveau de la gestion ressources humaines chez Odoo en 2010.

\begin{description}
    \item[Planification] Il nous manque des informtations pour pouvoir juger de la planification au niveau RH. Il semble que dans un contexte de croissance, le recrutement était ouvert pout les trois département.
    \item[Sélection] La sélection se faisait via une interview avec le directeur concerné ou alors directement avec le CEO. Dans un premier temps les interview était tout à fait informelle. Pour les postes techniques des exercices de programmation on été mis en place à la fin de 2010
    \item[Formation] Une semaine de formation était donné à tout les employés à leur arrivées. Une semaine supplémentaire est donné au profils techniques. Ces deux semaines de formation était dispensé car elle étaient vendues et prestées pour nos partenaires. Rien de spécifique à chaque poste n'existe. Pour cela il fallait se former sur le temps. Généralement, la méthode de formation consistait à jetté les nouveaux employés tout habillé dans la piscine.
    \item[Evaluation] La notion d'évaluation n'était pas du tout formaliser, elle se faisait à le demande de l'employé principalement. A cette époque les seules personnent qui furent congédiée appartenaient au département de vente lorsque ceux-ci avaient de mauvais chiffres. En générale sans prester le moindre préavis. Les personnes qui démissionnait ne prestaient aussi que très rarement leur préavis. La démission étant perçu parfois comme une trahisons de la part de la direction.
    \item[Rémunération] La rémunération dépendait de l'humeur du CEO lors de l'entretien de sélection. Ensuite, les augmentations se faisait plus à la demande de l'employé lorsque celui-ci considérait qu'il en méritait une. Il n'y avait aucune règle pour les augmentation.
    \item[Promotion] A cette époque il n'existait aucune possibilité de promotion, seul la croissance donnait l'espoir d'avoir une promotion un jour. Cela n'empêchait par contre en rien de voir sa rémunération augmentée. 
\end{description}

 

\paragraph{}Si regarde cette configuration et le modèle de GRH au configuration en lumière de la théorie de Pichault et Nizet \citep[pp. 48-49]{pichault}.
La configuration était à cette époque principalement entrepreneuriale: L'autorité du fondeur est grande. Une grande division du travail vertical mais faible au niveau horizontal, en tout cas intra département. La coordinatation du travail se fait via la supervision directe du CEO et du directeur. On peut aussi noter l'implication des membres de la famille. Malgré tout, il y a déjà des signes d'une configuration adhocratique: Employé très qualifié, une organisation par projet principalement en R\&D et au PS. 

\paragraph{}Par contre le modèle de GRH est clairement arbitraire\citep[pp. 115-119]{pichault}. Congédie sur le champs. Nous n'en avons pas parler mais à cette époque le petit nombre d'employé favorisait l'esprit maison avec des nombreux verre organisé. En R\&D, il n'était pas rare de se réveiller chez un collègue les lendemains de veille avec le CTO dans le canapé d'à coté. La sélection et les évaluations était sur un mode intuitif.  La formation se fait sur le tas et les promotions était presque inexistantes. 

\subsection{La configuration à l'heure actuelle}
\paragraph{} Le modèle de GRH arbitraire fonctionnait assez bien dans une petite société de 20 employés dans un configuration entrepreneuriale. Mais cette configuration à bien évoluée en cinq année alors que la société passait de 18 employés à 125 employés. 

\paragraph{} Comme évoqué déjà au premier chapitre, les départements se sont structurés en équipe. Le COO a été remplacé par un directeur du PS. Deux départements, marketing et financié, se sont rajoutés à l'ensemble. L'autonomie de chaque équipe à grandie, même si il y a toujours quelque coup de sonde et parfois un contournement de la ligne hiérarchique de la part du CEO. Par contre les décisions stratégique reste toujours entre les mains du comité de direction, emputé de son COO, mais avec deux nouveaux membres: le responsable marketing et le responsable financier. La communication entre les équipes se sont structurées : système de tickets. Au département de services, ils y a deux type de profils: Fonctionnel et Technique. Les équipes formé des deux profils se forme et se déforme au fils des projets. Ses équipes sont assez autonome. Des tensions sont apparues entre le département de ventes et le département de service, le premier ayant des objectifs de chiffre d'affaire et le second des objectifs de qualité de service et de rentabilité.

\paragraph{} Avec la croissance du nombre de client, le support à prit une place stratégique au sein du PS. Mais le support n'est pas gérer par une équipe dédié, celui-ci tourne entre les personnes. Cela présente deux avantages, le support étant perçu comme une tâche ingrate, on évite un turnover important qu'il pourrait y avoir dans le cas d'une équipe dédié. Le support touche à tous les aspects opérationnel d'Odoo, il a donc un grand pouvoir formateur dont tout le monde se doit de bénéficier. Mais cette configuration pose aussi des problèmes, celui de la standardisation de la qualité et des procédures. Des processus plus standardisés apparaissent aussi au niveau de la vente avec l'appui du logiciel Odoo. 

\paragraph{} Au niveau de l'organisation du travail au sein du PS, les choses n'ont pas beaucoup changées, on choisit toujours la personne la plus compétentes parmis les personnes disponibles, c'est-à-dire, celle qui est un peu moins occupé que les autres et la notion de personne compétente reste toujours très floue. 

\paragraph{} Nous pouvons observer que la configuration entrepreneuriale à laisser place à une configuration adhocratique\citep[pp. 53-54]{pichault} avec une forte décentralisation du pouvoir pour les questions opérationnelles, mais toujours une fortes centralisation pour les décisions stratégique. Il y a aussi une petite tendance à la bureaucracie pour les tâches plus répétitives comme le support, la création de contracts. 

\paragraph{}Il y a cependant encore de nombreux problème au niveau de l'organisation du travail au sein du PS. Il y a d'abord le problème de la qualité. Des activités comme le support ou la nécessite de fournir un niveau constant de qualité, en tout cas un niveau minimal. Ensuite, il y a un problème de spécialisation. Lorsqu'un nouveau type de tâche se présente la personne qui est disponible s'en charge. Par la suite, si d'autre tâche de la même nature se présente, la tendance est de l'assigné toujours à la même personne car elle connait déjà le problème. Ce comportement génère une situation problèmatique car il n'existe pas de backup si la personne en charge tombe malade ou démissionne. 
Au final l'assignation des tâches est principalement basée sur la disponibilité des ressources. Cela pose un problème au niveau de la formation de l'équipe. Les compétences de chacun sont développées au hasard, sans tenir compte des besoins ou de la volonté de l'employé. Il n'y a politique cohérente de développement des membres de l'équipe. C'est le moment pour faire le point de l'évolution de la politique des ressources humaines. 

\begin{description}
  \item[Planification] La planification des ressources se fait toujours département par département. A l'heure actuelle, un gel total du recrutement est opéré sans tenir compte des besoins d'aucun département dans le but d'atteindre le seuil de rentabilité. Les départs sont maintenant mieux gérer ceux-ci sont arrangé pour que l'employé puisse confier ses responsabilité à un collègue.
  \item[Sélection] Le processus de recrutement vise l’acquisition des profils génériques : des individus possédant des savoir académiques appropriés, une capacité intellectuelle d’apprentissage continue ainsi que un degré d’autonomie potentiel élevé. Les tests adapté pour chaque famille de poste sont évalués avant l'entretien avec le futur responsable.  Aucune palette de compétence spécifique ou de spécialiste n'est recherchée. Les compétences spécifiques s'aquiéreront avec l'expérience. 
  
  \item[Formation] Concernant la formation, rien n'a bouger. Seuls les deux semaines de formations sont offertes et ensuite chaque équipe coache ses nouveaux venus. Elle se fait sur le tas car le sommet hiérarchique est convaincu que la formation externe contient très peu de valeur et que l’exposition immédiate et fréquente aux problématiques des clients est un accélérateur pour l’obtention et le renforcement des compétences. Un système système d’accompagnement personnalisée a été mis en place pour chaque nouveau employé, mais seulement au sein de l'équipe fonctionnelle. 
  \item[Evaluation] L'évaluation reste est un des points noires de la gestion de ressources humaines. L'évaluation consiste simplement en un entretien d'évaluation. Celui-ci est l'occasion pour le responsable d'avoir une discussion sur l'année écoulée avec chaque membre de son équipe et surtout l'occasion pour l'employé de demander une augmentation si celle-ci c'est bien passé. Mais rien ne permet d'objectivé sur l'entretien c'est bien passé ou pas.  
\paragraph{}Le formulaire actuel d'évaluation (en Annexe 2) contient d'abord les objectifs de l'année précédente et tout à la fin des objectifs pour l'année à venir. La seconde partie de l'évaluation contient une série de point à évaluer, est-ce des compétences, des savoirs, des savoirs-faire ou des éléments pour mesurer la performance ? Le glossaire\footnote{A la première page du formulaie on peut lire \enquote{Each item is defined in the glossary (page 7).}. Le formulaire ne compte que 6 pages} n'est jamais communiqué ni à l'évaluateur ni à l'évalué. Les différents point doivent être noté. Mais la référence à utilisé dans la notation n'est pas définie. Ensuite viennent la conformité aux valeurs de l'entreprise, qui sont au nombre de huit. Encore une fois il faut les évaluaer selon une échelle sans référentiel. Ensuite vient l'appréciation global encore une fois sans aucune référence de base. Finalement, deux questions permettent d'aborder avec l'employé son aspiration professionnel. Ce formulaire rassemble de nombreux aspects de la gestion des ressources humaines, mais sans leurs donner de cadre. Il a été conçu de manière très générique pour convenir à tout les départements sans qu'aucun ne puisse s'y reconnaître.

\paragraph{}Dans le formulaire, des objectifs sont définis. Hélas, à l'heure actuelle, dans le département de service, il n'y a aucune politique cohérente pour définir les objectifs de chacun. Les objectifs qui sont définit sont que très rarement suivit. Le reste du questionnaire est souvent évité sauf si l'employé l'a remplit et qu'il souhaite discuter de point précis. La faute au manque de définition des éléments à évaluer et l'absence de méthode et de référence pour évaluer ces éléments. Tout cela donne la sensation d'une politique d'évaluation et de rémunération complètement arbitraire. Il y a aussi un manque important d'information et formation pour les personnes évaluent.
  \item[Rémunération] Le système est basé sur les précédents, il n'y toujours pas de politique de rémunération. Les salaires des anciens et leur évolution sert comme base comparative pour les nouveaux et leur évolution. Seule la liste des salaires actuelles est maintenue.
  
  \item[Promotion] Il y a eu des promotions mais très peu et la croissance de la société reste toujours le meilleur espoir de promotion. Mais ce n'est pas automatique. La R\&D est restée très longtemps avec un seul responsable le CTO malgré ses 40 membres. Ce n'est que très récement que celle-ci c'est doté de responsable.  
\end{description}

\paragraph{} On retrouve toujours beaucoup de caractéristique du modèle arbitraire. On retrouve malgré tout déjà quelque élément du modèle individualisant. Chez les commerciaux la rémunération est variable. La sélection ce fait sur base de compétence vérifiées par des tests lors de l'embauche. L'évaluation détermine des objectifs qui devrait être suivis et évalués à la fin de la période.


\paragraph{} Odoo présente quelque signe d'un modèle individualisant. La gestion des compétences doit permettre d'achevé la transition d'un modèle arbitraire qui ne convient plus à la taille de l'entreprise vers un modèle individualisant plus adapté à la nouvelle structure adhocratique. Les aspects les plus important que devra permettre cette gestion des compétences, sont la planification, la formation et l'évaluation. Il faudra pouvoir planifié de manière plus structuré les compétences nécessaire au sein de chaque équipe. Mais pour pouvoir planifié les besoins, il faudra faire un état des lieux de l'existant et ensuite décidé de la manière d'acquérir compétences manquante par la formation. Les évaluations devront permettre d'établir l'état des lieux des compétences existantes mais aussi de poussé les employés a apprendre les compétences manquantes au sein de leur équipe. Une fois cette gestion mis en place, il sera plus facile d'objectivé la rémunération basée sur les compétences de chacun. Et finalement, il sera aussi plus facile d'envisager une mobilité horizontale et verticale. Maintenant que nous savons exactement ce que nous attendons de la mise en place de la gestion des compétences, nous allons nous attarder dans le chapitre suivant à trouver le modèle qui convient le mieux à la situation et au objectif du PS au sein d'Odoo.








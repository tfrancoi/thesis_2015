\chapter{Définition du problème}
\paragraph*{}Je travail pour la société Odoo S.A., une jeune société du brabant wallon qui a été fondé en 2005. Une première levée de fond en 2010 à permis à l'entreprise de passé de 18 employés à 34 employés. Et une seconde en 2014 de 64 employés 111 employés. Il y a actuellement 125 personnes travaillant pour Odoo en Beglique et 285 au total dans le monde\footnote{Cette information est tirée du logiciel de gestion des ressource humaines interne consulté le 8 juillet}. Cette société édite des applications de gestion d'entreprise avec la particularité d'être toutes intégrée entre elles. Il s'agit d'un ERP\footnote{Enterprise Resource Planning\cite{wikierp}}. 


\paragraph*{}La société possède 3 activités principales, le développement du logiciel en tant que tel par la R\&D, le département de vente et le département de service. Le logiciel (Odoo) étant libre, sous licence AGPL\footnote{\enquote{GNU Affero General Public License (ou AGPL, signifiant licence publique générale Affero) est une licence libre dérivée de la Licence publique générale GNU avec une partie supplémentaire couvrant les logiciels utilisés sur le réseau. Elle a été écrite par Affero pour autoriser les droits garantis par la GPL à couvrir les interactions avec des produits à travers un réseau comme Internet, ce que la GPL ne fait pas.}\cite{agpl}} jusqu'en 2015\cite{odoonewlicense}, il n'y a pas de coup de license associé au logiciel. Les revenus viennent des services offerts autour du logiciel.Ceux-ci sont assurés par le département de service "Professional Services (PS)", là où je travail. 

\paragraph*{}Dans ce département, il y a deux équipes, l'équipe fonctionnelle, composé de consultants fonctionnels pour la pluspart ingénieur de gestion et d'un responsable et l'équipe technique composée de consultants techniques, tous avec une solide formation en informatique et un responsable d'équipe, moi-même. Depuis mon arrivées en 2010, le département de service est passée de 5 personnes à 36 personnes à l'heure actuelle. La croissance fut chaotique avec de nombreux départ, les deux équipes furent créent en court de route, la séparation entre poste fonctionelle et technique également et finalement la délégation par le directeur du département d'une partie de ces responsabilités à deux "team leader" date de moins d'un an. Mais en dehors de ca rien n'a été fait pour accompagner la croissance de ce département. Ni d'aucun autre d'ailleurs mais dans ce travail, nous ne nous intéresserons qu'a ce département pour d'une part circonscrire ce travail à un problème de taille modeste et pour l'accès priviligié à l'information que je possède concernant cette équipe. 


\paragraph*{}Cette croissance engendre de nombreux défits. L'intégration des nouveaux employés dans la culture de l'entreprise, leur formation. Il faut aussi veuiller à garder une qualité de service constante malgré la disparité des individus qui grandit. Proposer un chemin d'évolution dans l'entreprise malgré le nombre de postes très restreint au sein de l'entreprise et finalement assurer un maximum de flexibilité des ressources malgré la diversité des missions à accomplir. 

\section{La politique RH actuelle}
\subsection{Politique d'évaluation et de rémunération} 
\setlength{\epigraphwidth}{0.7\textwidth}
\epigraph{Personnel Appraisal: nom commun: donné par quelqu'un qui ne veut pas le donner à quelqu'un qui ne veut pas le recevoir}{\textit{Bowman J.S, 1999}}

A l'heure actuelle, la polique RH en terme d'évaluation et de rémunération est assez simple. Tout les ans, le manager de l'employé organise avec celui-ci un entretien d'évaluation. Mais l'entretien d'évaluation est simplement l'occasion pour le responsable d'équipe d'avoir une discussion sur l'année écoulée avec chaque membre de son équipe et surtout l'occasion pour l'employé de demander une augmentation si celle-ci c'est bien passé. Mais il n'y a aucun lien de proportionalité entre le résultat de l'évaluation et l'augmentation, pour peu qu'il y ai un résultat quantifiable.

\paragraph{}Le formulaire actuel d'évaluation (en Annexe 2) contient d'abord les objectifs de l'année précédente et tout à la fin des objectifs pour l'année à venir. La seconde partie de l'évaluation contient une série de point à évaluer, est-ce des compétences, des savoirs, des savoirs-faire ou des éléments pour mesurer la performance ? Le glossaire\footnote{A la première page du formulaie on peut lire \enquote{Each item is defined in the glossary (page 7).}. Le formulaire ne compte que 6 pages} n'est jamais communiqué ni à l'évaluateur ni à l'évalué. C'est laissé à la libre interprétation de l'évaluateur. Les différents point doivent être noté: bas, moyen ou haut. Mais cette notation fait référence à quoi ? Aucun n'élément dans le formulaire n'apporte cette réponse. Ensuite viennent la conformité aux valeurs de l'entreprise, qui sont au nombre de huit. Encore une fois il faut les évaluaer selon l'échelle: bas, moyen, haut. Et encore une fois aucune référence n'est donné. Ensuite vient l'appréciation global de 1 à 5 encore une fois sans aucune référence de base. Finalement, deux questions permettent d'aborder avec l'employé son aspiration professionnel. Ce formulaire rassemble de nombreux aspects de la gestion des ressources humaines, mais sans leurs donner de cadre. Il a été conçu de manière très générique pour convenir à tout les départements mais finalement personne ne s'y reconnait.

\paragraph{}Comme nous l'avons vu dans le formulaire, des objectifs sont définis. Pour le département de la vente c'est très simple, les objectifs sont les volumes de ventes quoi réaliser l'employé chaque trimestre. Mais dans un département de service, les choses sont beaucoup plus compliquée. Surtout lorsqu'il n'est pas encore possible de mesuré la rentabilité des services et parce que cette rentabilité dépends de nombreux facteur qui ne sont pas contrôllable par l'employé. Par exemple dans mon équipe, nous avons des tâches de développement à effectué. Un nombre de jours bien définit à été vendu pour cette tâches. Si cette activité n'est pas rentable, est-ce la faute du consultant qui est trop lent ? La faute de la personne qui à estimé le temps pour cette tâche et qui s'est trompé ? La faute au commercial qui n'a pas respecté l'estimation ou qui a fait trop de ristourne ? Si la rentabilité doit entré en ligne de compte dans les objectifs, ce n'est clairement pas le seul qui doit entré en ligne de compte. 

\paragraph{}A l'heure actuelle, dans le département de service, il n'y a aucune politique cohérente pour définir les objectifs de chacun. Les objectifs qui sont définit sont que très rarement suivit. Le reste du questionnaire est souvent évité sauf si l'employé l'a remplit et qu'il souhaite discuter de point précis. La faute au manque de définition des éléments à évaluer et l'absence de méthode et de référence pour évaluer ces éléments. Tout cela donne la sensation d'une politique d'évaluation et de rémunération complètement arbitraire. Il y a aussi un manque important d'information et formation pour les personnes évaluent.



 


\subsection{Organisation du travail et flexibilité}
\paragraph*{} Dans l'équipe technique, malgré la diversité des tâches tout le monde est suceptible d'effectué toutes les tâches. En tout cas en théorie. En pratique, les tâches les plus complexes sont assignées aux employés les plus expérimenter. Les tâches nécessitant des connaissance spécifique sont assignées aux personnes qui ont déjà réalisé des tâches similaires. Le support technique implique tout les membres de l'équipe qui prennent à tour de rôle chaque semaine sa responsabilité. Cette organisation permet d'avoir un socle commun à tout les membres et une légère spécialisation de chacun. Elle permet une bonne flexibilité et une répartition du travail optimal. Jamais personne ne se retrouve surchargé pendant que quelqu'un d'autre n'a rien à faire. 

\paragraph{}Mais cette organisation pourrait être améliorer. Il y a d'abord le problème de la qualité. Des activités comme le support ou la nécessite de fournir un niveau constant de qualité, en tout cas un niveau minimal. Ensuite, il y a un problème de spécialisation. Lorsqu'un nouveau type de tâche se présente la personne qui est disponible s'en charge. Par la suite, si d'autre tâche de la même nature se présente, la tendance est de l'assigné toujours à la même personne car elle connait déjà le problème. Ce comportement génère une situation problèmatique car il n'existe pas de backup si la personne en charge tombe malade ou démissionne. 
Finalement l'assignation des tâches est principalement basée sur la disponibilité des ressources. Il n'y a donc aucune politique cohérente de développement des membres de l'équipe. 



\section{Objectifs}
Au vu des constats tirés dans la section précédente, il y a beaucoup de chose qui pourraient être améliorée. La gestion des compétences pourrait nous y aider. D'après le livre\citep{gestionressourceshumaine2007} et l'article \citep{delobbe}, la gestion des compétences a plusieurs objectifs dont voici une liste non exhaustive.
\begin{itemize}
    \item \enquote{Elle a pour effet d'assurer l'adhésion des salariés à des normes de comportements considérées comme acceptables et valorisées au sein d'une organisation}\citep[p.40]{delobbe}
    \item Elle permet d'adapter les comportements des salariés aux attentes des clients dans une logique de service et de qualité. \citep[182]{gestionressourceshumaine2007}
    \item \enquote{Gérer la polyvalence et la flexibilité... Elle vise à dépasser une allocation rigide des effectifs à des postes de travail délimités}\citep[p.41]{delobbe}
    \item Elle favorise une redistribution des tâches et des responsabilités en cas de réduction d'effectifs \citep[182]{gestionressourceshumaine2007}
    \item  \enquote{L'objectif de la gestion des compétences est donc d'identifier, de sélectionner, de promouvoir et de récompenser les plus talentueux..}\citep[p.43]{delobbe}
    \item  \enquote{La gestion des compétences a ici pour objectifs premier d'affirmer et de développer l'expertise technique interne indispensable à la réalisation des missions de l'entreprise.} \citep[p.45]{delobbe}

\end{itemize}
           
Mais une seule méthode de gestion des compétences ne peut répondre à tous ces objectifs à la fois. Il faudra cibler les  plus importants. Et comme le suggère, la conclusion de l'article\citep{delobbe}, le type de gestion des compétences est corréler avec le type de la structure organisationelle. Dans la suite de ce document, il faudra définir quels sont les objectifs prioritaire en s'aidant de la structure organisationnelle d'Odoo. 

\paragraph{}La gestion des compétences peut être utilisé à tous le niveau des ressources humaines: Les évaluations et la rémunération du personnel, la gestion des performances, la classification des emplois, l'allocation des effectifs, la sélection des candidats, la formation des employés et la gestion des carrières. \citep[p.32]{delobbe}. Dans un monde idéal la gestion des compétences devrait s'inscrire dans une démarche stratégique. 

\paragraph{}\enquote{Si, comme le suggèrent les modèles théoriques, les compétences constituent un avantage concurrentiel, alors la mise en place d'une gestion des compétences devrait s'inscrire directement dans la démarche stratégique de l'entreprise.}\citep[P. 188]{gestionressourceshumaine2007}.

\paragraph{}Malheureusement cette mise en place de la gestion des compétences ne venant pas du sommet de la hiérarchie, il ne sera pas possible d'inscrire totalement cette démarche dans la stratégie de l'entreprise. Une autre contrainte étant que la stratégie de l'entreprise change assez fréquement car un modèle vraiment rentable n'a pas encore été trouvé. La gestion des compétences s'inscrira donc dans la limite du département de service et commencera avec l'équipe des consultants techniques. 


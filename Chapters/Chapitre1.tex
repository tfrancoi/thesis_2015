\chapter{État des lieux et définition des objectifs}
\paragraph{}Odoo S.A, située en Brabant Wallon, est une société d’édition d'applications de gestion d'enterprises. Depuis sa fondation en 2005, elle jouit d'une croissance hors pair : elle est passée de 2 à 285 employés en dix ans. En cinq ans, le nombre d'employés en Belgique est passé de 18 à 125\footnote{Cette information est tirée du logiciel de gestion des ressource humaines interne consulté le 8 juillet}. Cette croissance est d'abord dûe à l’aspect innovant des logiciels édités: ils sont Open Source, sous licence AGPL\footnote{\enquote{GNU Affero General Public License (ou AGPL, signifiant licence publique générale Affero) est une licence libre dérivée de la licence publique générale GNU avec une partie supplémentaire couvrant les logiciels utilisés sur le réseau. Elle a été écrite par Affero pour autoriser les droits garantis par la GPL à couvrir les interactions avec des produits à travers un réseau comme Internet, ce que la GPL ne fait pas.}\cite{agpl}}, intégrés entre eux à l'image des ERP\footnote{Enterprise Resource Planning\cite{wikierp}} et permettent, contrairement à la concurrence, une adaptation rapide et compétitive aux besoins des clients. Elle est également encouragée par une approche commerciale centrée sur les PME sans contrainte géographique. Ces atouts ont permis deux levées de fonds, en 2010 et en 2014, desquelles ont découlé deux augmentations de personnel (de 18 à 34 employés, puis de 64 à 111 employés). 

\paragraph{}Les applications ne comportant pas de couts de licence, Odoo génère ses revenus des services d’implémentation des applications auprès des clients, ainsi que des services de maintenance, de support et de migrations vers des versions supérieures. Les applications sont rendues disponibles sur une platforme mutualisée ou SAAS\footnote{Software as a Service}. Odoo délivre aussi du support et de la formation à un réseau de partenaires qui s'occupe du marché des grandes entreprises. 


\paragraph{}Les produits et activités cités ci-dessus sont assurés par trois départements principaux: Le développement du logiciel et de la platforme mutualisée par le département de recherche et développpement\footnote{Par la suite, il sera abrégé par R\&D}; le département de vente et marketing qui assure la stratégie commerciale et la vente de services; et finalement le département de services ou PS\footnote{Abréviation de Professional Services} qui assure les services aux clients (services avant-vente, services d’analyse des besoins et de faisabilité, de mise en oeuvre à travers des projets, de maintenance, de support). 

\paragraph{}Le PS comprend deux équipes. D'une part, il y a l'équipe fonctionnelle, composée de consultants fonctionnels pour la plupart ingénieurs de gestion et d'un responsable, qui s'occupe des missions d'analyse, d'implémentation, de gestion de projet et de support. D'autre part, il y a l'équipe technique, composée de consultants techniques et d'un responsable d'équipe,tous avec une solide formation en informatique, qui s'occupe de l'analyse de faisabilité, de l'implémentation des adaptations du logiciel aux besoins des clients, du support technique et de la formation. Depuis 2010, le département de services est passé de 5 personnes à 36 personnes. La croissance a été chaotique avec de nombreux départs; les deux équipes ont été crées en cours de route; la séparation entre poste fonctionel et technique également; et finalement la délégation par le directeur du département d'une partie de ses responsabilités à deux "team leaders" date de moins d'un an. Mais en dehors de cela, rien n'a été fait pour accompagner la croissance de ce département. En tant que responsable d’équipe technique au PS , j’ai le privilège de pouvoir accéder à une partie des informations permettant de faire des constats ainsi que de formuler des solutions et de les mettre en oeuvre. Pour cette raison, le présent travail se focalisera principalement sur ce département.


\paragraph*{}Une croissance aussi rapide engendre de nombreux problèmes et défits dont l'intégration des nouveaux employés dans la culture de l'entreprise et leur formation. Il faut veiller à garder une qualité de service constante malgré les disparités grandissantes entre les individus, offrir à ceux-ci des possibilités d'évolution dans l'entreprise malgré le nombre de postes très restreint dont elle dispose et finalement envisager une meilleure organisation du travail (actuellement basée principalement sur la disponibilité), afin d'assurer un résultat optimal et le développement de chacun. Pour répondre à ces nombreux défits nous avons choisi de nous aider de la gestion des compétences.

\section{Objectifs visés par la gestion des compétences}
D'après le livre de Guérin\citep{gestionressourceshumaine2007} et l'article de Delobbe\citep{delobbe}, la gestion des compétences a plusieurs objectifs dont voici une liste non exhaustive.
\begin{itemize}
    \item \enquote{Elle a pour effet d'assurer l'adhésion des salariés à des normes de comportements considérées comme acceptables et valorisées au sein d'une organisation}\citep[p.40]{delobbe}
    \item Elle permet d'adapter les comportements des salariés aux attentes des clients dans une logique de service et de qualité. \citep[182]{gestionressourceshumaine2007}
    \item \enquote{Gérer la polyvalence et la flexibilité... Elle vise à dépasser une allocation rigide des effectifs à des postes de travail délimités}\citep[p.41]{delobbe}
    \item Elle favorise une redistribution des tâches et des responsabilités en cas de réduction d'effectifs \citep[182]{gestionressourceshumaine2007}
    \item  \enquote{L'objectif de la gestion des compétences est donc d'identifier, de sélectionner, de promouvoir et de récompenser les plus talentueux..}\citep[p.43]{delobbe}
    \item  \enquote{La gestion des compétences a ici pour objectif premier d'affirmer et de développer l'expertise technique interne indispensable à la réalisation des missions de l'entreprise.} \citep[p.45]{delobbe}

\end{itemize}

\paragraph{}La gestion des compétences peut être utilisée à tous les niveaux des ressources humaines: les évaluations et la rémunération du personnel; la gestion des performances; la classification des emplois; l'allocation des effectifs; la sélection des candidats; la formation des employés et la gestion des carrières. \citep[p.32]{delobbe}. Normalement, la gestion des compétences devrait s'inscrire dans une démarche stratégique. Malheureusement, cette mise en place de la gestion des compétences ne venant pas du sommet hiérarchique, il ne sera pas possible de l'inscrire dans la stratégie de l'entreprise. Un autre problème est le manque de stratégie de la société à moyen et long terme.  La gestion des compétences s'inscrira donc dans la limite du département de services et commencera avec l'équipe des consultants techniques. 

\paragraph{}Avant de se lancer tête baissée dans la définition de la gestion des compétences qui va être mise en place chez Odoo, arrêtons-nous un instant. La conclusion de l'article\citep{delobbe} suggère un lien entre le modèle de gestion des compétences à appliquer et le type de structure organisationelle de l'entreprise qui la met en place. Attardons-nous donc un instant sur la structure organisationnelle d'Odoo, sur son évolution et sur l'évolution du modèle de gestion des ressources humaines. Une fois cet état des lieux fait, nous pourrons déterminer quelle gestion des compétences est la plus appropriée.
        



\section{Structure organisationnelle et politique de gestion des ressources humaines au sein d'Odoo}
\subsection{La configuration en 2010}
Il est intérressant de revenir à la configuration d'Odoo en 2010, juste après la première levée de fond. Comme on peut le voir dans le tableau \ref{nb_employe} en annexe, au début de l'année 2010, il y avait 18 employés et à la fin de l'année, il y en avait déjà 34. Ce nombre reste faible comparé au 125 travailleurs employés en belgique actuellement. À cette époque, le sommet hierarchique était composé d'un CEO-fondateur, d'un CSO, d'un COO et d'un CTO\footnote{Respectivement Chief Executive Officer, Chief Sales Officer, Chief Operating Officer, Chief Technical Officer}. Il y avait déjà trois départements, présents sur le même site: le département de recherche et développement géré par le CTO; le département de vente géré par le CSO; et le département de services, en théorie géré par le COO. En dehors du sommet hiérchique, il n'y avait pas de responsable d'équipe. Les employés sont depuis le début très qualifiés: des ingénieurs en informatique (en R\&D et au département de services), et des ingénieurs des gestions (au département de services et à la vente). Au sein de chaque département, le travail était intercheangable entre chaque membre d'un département. En R\&D et au PS\footnote{Abréviation pour les départements de service: Professional Services}, chacun travaillait sur son projet et il était difficile de changer l'assignation en cours de route, mais toute nouvelle tâche était suceptible d'être assignée à quiconque. Le CEO passait presque quotidiennement voir qui faisait quoi, pour faire des ajustements ou débloquer une situation. Toutes les décisions étaient prises par le comité de direction composé des quatres membres exécutifs mais bien entendu le CEO avait toujours le dernier mot. C'est à cette époque que le beau-frère de celui-ci fut envoyé aux États-Unis pour ouvrir un nouveau bureau. Au sein du PS, l'organisation du travail se faisait de manière très simple: on choisissait la personne la plus compétente parmis les personnes disponibles, et sachant que personne n'était jamais vraiment disponible, on choisissait simplement la plus disponible. Le paragraphe suivant résume la situation au niveau de la gestion ressources humaines chez Odoo en 2010:

\begin{description}
    \item[Planification] Il nous manque des informations pour pouvoir évaluer la planification au niveau RH. Il semble que dans un contexte de croissance, le recrutement était ouvert pour les trois départements.
    \item[Sélection] La sélection se faisait via une interview avec le directeur concerné ou alors directement avec le CEO. Dans un premier temps, les interviews étaient informelles. Pour les postes techniques, des exercices de programmation ont été mis en place à la fin de l'année 2010.
    \item[Formation] Tous les employés recevaient une semaine de formation à leur arrivée. Une semaine supplémentaire était octroyée aux profils techniques. Ces deux semaines de formation étaient données car elles étaient vendues et prestées pour nos partenaires. Les formations n'étaient pas spécifiques en fonction des différents postes. Il fallait se former sur le tas. De manière générale, la méthode de formation consistait à jeter les nouveaux employés tout habillés dans la piscine.
    \item[Evaluation] La notion d'évaluation n'était pas du tout formalisée. Elle se faisait principalement à la demande de l'employé. À cette époque, les seules personnes congédiées étaient celles du département de vente lorsqu'elles avaient de mauvais chiffres, et elles ne prestaient généralement pas leur préavis. Les personnes qui démissionnaient ne prestaient également que très rarement leur préavis. La démission étant parfois perçue comme une trahison de la part de la direction.
    \item[Rémunération] La rémunération dépendait de l'humeur du CEO lors de l'entretien de sélection. Les augmentations se faisaient ensuite essentiellement à la demande de l'employé: il n'y avait donc aucune règle en la matière. 
    \item[Promotion] Il n'existait, par ailleurs, aucune possibilité de promotion: seule la croissance donnait l'espoir d'en avoir une un jour. Cela n'empêchait par contre en rien de voir sa rémunération augmentée. 
\end{description}

 

\paragraph{}Si l'on met en parallèle les particularités des ressources humaines d'Odoo en 2010 et la théorie de Pichault et Nizet \citep[pp. 48-49]{pichault}, on peut tirer comme conclusion que la configuration était principalement entrepreneuriale. L'autorité du fondeur était élevée. Il y avait une grande division du travail au niveau vertical mais faible au niveau horizontal, en tout cas au sein d'un même département. La coordinatation du travail se faisait via les supervisions directes du CEO et du directeur. On pouvait aussi noter l'implication des membres de la famille. Cependant, on pouvait déjà percevoir des signes d'une configuration adhocratique: employé très qualifié, organisation principalement par projet en R\&D et au PS. 

\paragraph{}Par contre, le modèle de GRH\footnote{Gestion des ressources humaines} était clairement arbitraire\citep[pp. 115-119]{pichault}, preuve en est le licenciement instantané des employés à congédier. Par ailleurs, le petit nombre d'employé favorisait "l'esprit maison"" avec l'organisaton de nombreux verres entre collègues. En R\&D, il n'était pas rare de se réveiller chez un collègue les lendemains de veille avec le CTO dans le canapé d'à coté. La sélection et les évaluations se faisaient intuitivement.  La formation se faisait sur le tas et les promotions étaient presque inexistantes. 

\subsection{La configuration à l'heure actuelle}
\paragraph{} Ce modèle de GRH arbitraire fonctionnait assez bien avec la configuration entrepreneuriale d'une petite société de 20 employés. La configuration a cependant bien évolué ces cinq dernières années, durant lesquelles l'entreprise est passée de 18 employés à 125 employés. 

\paragraph{} Comme évoqué antérieurement, les départements se sont progressivement structurés en équipes. Le COO a été remplacé par le directeur du PS. Deux départements, marketing et financier, se sont greffés à l'ensemble. L'autonomie de chaque équipe a grandi, avec, parfois, un contournement de la ligne hiérarchique de la part du CEO. Par contre, les décisions stratégiques restent toujours entre les mains du comité de direction, emputé de son COO, mais comprenant deux nouveaux membres: le responsable marketing et le responsable financier. Les communications entre les équipes se sont structurées via un système de tickets. Au département de services, il y a deux types de profils: fonctionnel et technique. Les équipes constituées des deux profils se forment et se déforment au fil des projets. Elles sont assez autonomes. Des tensions sont apparues entre le département de ventes et le département de service, le premier ayant des objectifs de chiffre d'affaire et le second des objectifs de qualité de service et de rentabilité.

\paragraph{} Avec la croissance du nombre de clients, le support des utilisateurs a pris une place stratégique au sein du PS, mais il n'est pas géré par une équipe en particulier. Les employés s'en occupent à tour de rôle. Cela présente deux avantages. Premièrement, le support étant perçu comme une tâche ingrate, on évite la frustration que pourraient ressentir les personnes si elles y étaient allouées sans interruption. Deuxièmement, le support touche à tous les aspects opérationnel d'Odoo: il a donc un grand pouvoir formateur dont tout le monde doit bénéficier. Cependant, cette configuration pose problème au niveau de la standardisation de la qualité et des procédures. Des processus plus standardisés apparaissent par ailleurs au niveau de la vente avec l'appui du logiciel Odoo. 

\paragraph{} Au niveau de l'organisation du travail au sein du PS, les choses n'ont pas beaucoup changé, on choisit toujours la personne la plus compétente parmi les personnes disponibles, c'est-à-dire, celle qui est un peu moins occupé que les autres et la notion de personne compétente reste très floue. 

\paragraph{} Nous pouvons observer que la configuration entrepreneuriale a cédé la place à une configuration adhocratique\citep[pp. 53-54]{pichault} avec une forte décentralisation du pouvoir pour les questions opérationnelles, mais toujours une forte centralisation pour les décisions stratégiques. Il y a aussi une petite tendance à la bureaucracie pour les tâches plus répétitives comme le support ou la création de contrats. 

\paragraph{}Il y a cependant encore de nombreux problèmes au niveau de l'organisation du travail au sein du PS. Il y a, d'abord, celui de la qualité: une activité telle que le support nécessite de fournir un niveau relativement constant de qualité, qui n'est actuellement pas garanti. Ensuite, la spécialisation pose problème. Lorsqu'un nouveau type de tâche se présente, la personne qui est disponible s'en charge. Par la suite, si d'autres tâches de la même nature surviennent, la tendance est de l'assigner toujours à la même personne car elle connait déjà le problème. Ce comportement génère une situation problèmatique car il n'existe pas de backup si la personne en charge tombe malade ou démissionne. 
Finalement, l'assignation des tâches est principalement basée sur la disponibilité des ressources. Cela pose un problème au niveau de la formation de l'équipe. Les compétences de chacun sont développées au hasard, sans tenir compte des besoins ou de la volonté de l'employé. Il n'y a pas de politique cohérente de développement.

\paragraph{}Faisons maintenant le point sur l'évolution de la politique des ressources humaines depuis 2010. 

\begin{description}
  \item[Planification] La planification des ressources se fait toujours département par département. À l'heure actuelle, un gel total du recrutement est opéré sans tenir compte des besoins des départements, dans le but d'atteindre un seuil de rentabilité. Les départs sont maintenant mieux gérés, de façon à ce que l'employé qui part puisse confier ses responsabilités à un collègue.
  \item[Sélection] Le processus de recrutement vise l’acquisition de profils génériques : des individus possédant des savoir académiques appropriés, une capacité intellectuelle d’apprentissage continue ainsi qu'un potentiel d’autonomie élevé. Les tests adaptés pour chaque famille de postes sont évalués, avant l'entretien, avec le responsable du service. Les compétences spécifiques ne sont pas recherchées: elles s'acquerront avec l'expérience. 
  
  \item[Formation] Concernant la formation, rien n'a bougé. Après les deux semaines de formation, chaque équipe coache ses nouveaux venus. Ce modèle a été maintenu car le sommet hiérarchique est convaincu que la formation externe contient très peu de valeur et que l’exposition immédiate et fréquente aux problématiques des clients est un accélérateur pour l’obtention et le renforcement des compétences. Un système d’accompagnement personnalisé a donc été mis en place pour chaque nouvel employé au sein de l'équipe fonctionnelle. 
  \item[Evaluation] L'évaluation reste est un des points noirs de la gestion de ressources humaines. Elle consiste simplement en un entretien d'évaluation. Celui-ci est l'occasion pour le responsable d'avoir une discussion sur l'année écoulée avec chaque membre de son équipe, mais c'est surtout l'occasion pour l'employé de demander une augmentation si l'évaluation s'est bien passée. Rien ne permet objectivement de savoir si l'évaluation est bonne ou mauvaise. 
  
\paragraph{}Le formulaire d'évaluation actuel (Annexe B) comporte plusieurs subdivisions. La première et la dernière abordent les objectifs de l'année écoulée et de celle à venir. La seconde partie contient une série de points à coter. Il s'agit d'un mélange de compétences, de savoirs, de savoirs-faire qui seraient pertinants pour mesurer la performance. Cependant, le glossaire\footnote{A la première page du formulaie on peut lire \enquote{Each item is defined in the glossary (page 7).}. Le formulaire ne compte que 6 pages} n'est jamais communiqué, ni à l'évaluateur, ni à l'évalué. Les différents points doivent être notés, mais l'échelle de référence à utiliser dans la notation n'est pas définie. La troisième subdivision traite de la conformité du comportement de la personne aux valeurs de l'entreprise, qui sont au nombre de huit. Elle comporte également une notation de l'appréciation globale. Ici encore, l'évaluation doit se faire selon une échelle non définie. Finalement, deux questions permettent d'aborder avec l'employé ses aspirations professionnelles. Ce formulaire rassemble de nombreux aspects de la gestion des ressources humaines, mais sans leur donner de cadre. Il a été conçu de manière très générique pour convenir à tous les départements sans qu'aucun ne puisse s'y reconnaître.

\paragraph{}Dans le formulaire, des objectifs globaux sont définis. Malheureusement, à l'heure actuelle, dans le département de service, il n'y a aucune politique cohérente pour définir des objectifs personnels. D'ailleurs, les objectifs qui sont définis ne sont que très rarement suivis. Dans la pratique, le reste du questionnaire n'est souvent pas abordé, sauf si l'employé l'a rempli et qu'il souhaite discuter de points précis. La faute en incombe au manque de définition des éléments à évaluer et à l'absence de méthode et de références pour évaluer ces éléments. Tout cela donne la sensation d'une politique d'évaluation et de rémunération complètement arbitraire. Il y a également un manque important d'information et formation pour les personnes qui évaluent.
  \item[Rémunération] Le système est basé sur les précédents, il n'y toujours pas de politique de rémunération. Les salaires des anciens et leurs évolutions servent comme base comparative pour les salaires des nouveaux venus et leurs évolutions. Seule la liste des salaires actuels est maintenue.
  
  \item[Promotion] Il y a eu très peu de promotions, et la croissance de la société reste toujours le meilleur espoir pour en obtenir une. Ellles ne sont toutefois pas automatiques. La R\&D est restée très longtemps avec un seul responsable, le CTO, malgré ses 40 membres. Ce n'est que très récemment qu'elle en a eu davantage.  
\end{description}

\paragraph{} On retrouve toujours beaucoup de caractéristiques du modèle arbitraire, tout en rencontrant déjà quelques éléments du modèle individualisant. Chez les commerciaux, la rémunération est variable. La sélection se fait sur base de compétences vérifiées par des tests lors de l'embauche. L'évaluation détermine des objectifs qui devraient être suivis et évalués à la fin de la période.


\paragraph{} La gestion des compétences chez Odoo doit donc permettre d'achever le passage d'un modèle arbitraire, qui ne convient plus à la taille de l'entreprise, à un modèle individualisant, plus adapté à la nouvelle structure adhocratique. Les aspects les plus importants que cette gestion des compétences devra permettre sont la planification, la formation et l'évaluation. L'objectif est de pouvoir planifier de manière plus structurée les compétences nécessaires au sein de chaque équipe. Pour pouvoir planifier les besoins, nous devrons faire un état des lieux de ce qui existe et ensuite décider comment acquérir les compétences manquantes via la formation. Les évaluations devront permettre d'établir l'état des lieux des compétences existantes mais aussi de pousser les employés à acquérir celles manquant au sein de leur équipe. Une fois cette gestion mise en place, il sera plus facile d'objectiver une rémunération basée sur les compétences de chacun. Il sera également plus facile d'envisager une mobilité horizontale et verticale.

\paragraph{} Maintenant que les attentes par rapport à la mise en place d'une gestion des compétences ont été définies, chercons le modèle qui convient le mieux à la situation et aux objectifs du PS au sein d'Odoo.







\chapter{Ébauche du référentiel de compétences techniques}

\section{Processus à analyser}
\begin{description}
  \item[Projet]
  Gestion de projet du point de vue de l'équipes techniques, de l'analyse technique à la maintenance en passant par les développement et les tests. L'un des processus les plus complexe à gérer mais aussi l'un des mieux maitriser et mieux documenter
  \item[Support Technique]
  Le processus est en soit assez simple mais relève de cas le plus souvent complexe et inédit. Il nécessite beaucoup d'ingéniosité mais aussi beaucoup de tact.
  \item[Formation]
  Le contenu de la formation est simple et relativement stable. Mais ce processus nécessite des compétences pédagogiques importantes
  \item[Consultance]
  C'est un processus fourre-tout qui nécessite une bonne dose d'improvisation parfois et un bon contact avec le client
  \item[Développement Saas] 
  Ce processus est le plus simple et est assez bien maitrisé. Mais les possibilités limités de l'environnement nécessite pas mal de créativité.
  \item[Avant-vente]
  Malgré ces aspects l'avant-vente est un des processus les plus avancés, il nécessite généralement de bien maitrisé tout les autres mais il faut en plus maitrisé des apsects humains et tenir compte de considération économique.
  \item[Administration de l'infrastructure]
  Processus clairement définit mais qui nécessite des compétences très différent de tout les autres. 
\end{description}


\section{Analyse du processus: Projet}
\begin{description}
    \item[Analyse Technique, estimation et faisabilité] 
    \item[Développement Backend]
    \item[Création des vues]
    \item[Développement Front end]
    \item[Web Design]
    \item[Test fonctionnel]
    \item[Test automatique backend] -
    \item[Test automatique frontend] -
    \item[Revue de la qualité du code]
    \item[Intégration]
    \item[Documentation Technique]
    \item[Documentation Fonctionnelle]
    \item[Support] -
    \item[Version Migration]
    \item[Data Migration]
    \item[Odoo Deployement]
    \item[Administration Postgresql]
    \item[Performance optimisation] - 
    \item[Méthode agile]       -
    \item[Gestion des sources] -
\end{description}

\section{Description de compétence}


\section{Exemple de socle de base}

\section{Exemple de profil cohérent}

\chapter{Élaboration d'un référentiel de compétence}
L'élaboration d'un référentiel de compétences pousse à prendre une série de décisions. Cette élaboration peut être vue comme un processus qui va figer dans le marbre la définition et l'organisation du travail mais il n'en est rien: un référentiel est un outil qui évolue dans le temps et qui est conçu pour apréhender le futur. "S'il s'ancre dans le travail d'aujourd'hui, il vise essentiellement le travail de demain."\citep[pp.19]{refcompetence} Nous ne resterons donc pas pétrifié par l'ampleur de la tâche et par le nombre de mauvaises directions qu'il est possible de prendre. Après tout, puisqu'il s'agit de construire un réfétentiel de compétences pour Odoo, sa construction doit se faire suivant l'esprit de l'entreprise\footnote{Nous faisons ici référence aux deux dernières valeurs présentes dans le formulaire d'évaluation en annexe: "I want to move forward" et "I prefer to make things evolve than to not makes mistakes"} 

\paragraph{}L'introduction du livre "Élaborer des référentiels de compétences"\citep{refcompetence} propose une méthodologie en neuf étapes pour la mise en place et l'adoption du référentiel et son usage dans l'entreprise. Le processus est représenté à l'annexe C. 

\paragraph{} Les deux premières étapes du processus: "Se doter d'une définition de la compétence" et "Clarifier la finalité" ont été explicitées dans le chapitre précédent. Il est intérressant de noter que nous avons suivi une approche légèrement différente: suite à la lecture de l'article\citep{delobbe}, nous sommes parti de la finalité pour se doter de la définition appropriée de la compétence. Les sept étapes suivantes seront élaborées dans ce chapitre.

\paragraph{} Nous nous attarderons sur les étapes "soumettre à validation" et "organiser l'approbation par les acteurs" car ces étapes pourraient présenter des problèmes assez différents de ce à quoi l'on s'attend lors d'une mise en place "classique" par le département des ressources humaines ou par le sommet hiérarchique. 
Ces étapes sont nécessaires pour asseoir la légimité du référentiel. Le faire accepter par les employés opérationels pose généralement problème au département des ressources humaines et à la hiérarchie. Cependant, dans le cas présent, des éléments assez bas dans la hiérarchie définissent le référentiel. En outre, comme nous l'avons déjà expliqué dans le chapitre précédent, une partie du référentiel devra être construit avec les membres de l'équipe du PS. Il faudra bien sûr le faire valider par ceux-ci, ce qui ne devrait pas trop poser de problèmes, mais il faudrait, dans l'idéal, le faire accepter par les ressources humaines et le sommet de la hiérarchie. C'est une problématique hautement politique et il est fort probable que ce ne soit pas le cas dans un premier temps. Il faut donc se poser la question de savoir si la gestion des compétences qui sera mise en place va permettre d'atteindre les objectifs fixés sans leur soutien. Si c'est le cas, comme nous le pensons, une fois mise en place et fonctionelle, elle sera beaucoup plus facile à promouvoir au sommet de la hiérarchie. Nous allons maintenant décrire ce que nous allons faire lors de chacune des étapes. 

\section{Préciser le format, Recueillir les données et Traiter les données}
Le référentiel contiendra une liste de compétences, mais sous quelle forme et comment générer cette liste ? Dans le chapitre précédent, nous avons déterminé que deux types bien distincts de compétences nous intérressaient : les références génériques et transversalles, qui pourront être utilisées par tous; et les compétences fonctionnelles et techniques, fortement contextualisées et uniques à chaque équipe. Pour les compétences génériques, la littérature est assez abondante, nous pourrons nous baser sur un glossaire existant, comme par exemple \citep{exemple_ref}, en l'adaptant au format choisi. Il est également possible de partir du travail déjà effectué par le département des ressources humaines. Les compétences techniques et fonctionelles sont quant-à-elles spécifiques à l'équipe du PS. Il faudra donc les construire à partir de rien et nous allons surtout nous concentrer sur cette partie. 

\paragraph{}Si l'on considère que les compétences sont ancrées dans l'action et qu'elles se manifestent dans l'action, les tâches et les processus sont un bon point de départ pour en établir la liste.  Dans l'annexe D, nous listons les processus qu'il faudra décortiquer en tâches. À partir de chaque tâche, nous listons les compétences nécessaires. Pour faire ressortir les compétences comportementales, il est interressant de noter avec qui intéragit la personne qui effectue la tâche, comme évoqué dans \citep[pp. 185]{refcompetence}. Le processus de projet est découpé en tâches et chaque tâche est analysée dans l'annexe D. Pour receuillir toutes les informations sur les processus, les tâches et les compétences requis, le concours des équipes sera nécessaire. Une fois ces données récupérées, il faudra les faire traiter par une personne qui connait bien le metier. Il pourra ainsi éliminer les doublons et proposer une classification de chaque compétence. En parallèle, chaque compétence devra se doter d'une définition des différents niveaux de maitrise. Enfin, certaines compétences vont dépendre d'autres compétences: par exemple, la capacité à juger la qualité du code produit par un collègue nécessite de maitriser les domaines impliqués dans son programme; pour pouvoir donner une formation technique, il faudra connaitre parfaitement son contenu, etc.. 

\paragraph{} Il est important de s'arrêter un instant sur les niveaux de maitrise de chaque compétence. Il existe de nombreuses possibilités pour les définir. L'une d'entre elles consiste à déterminer un certain nombres de niveaux (par exemple trois). Chaque description de compétence contiendra dès lors l'explication de ces niveaux dans le cadre de la compétence. Avec cette méthodologie, on risque de se retrouver avec des niveaux peu cohérents au sein des différentes compétences (un trois pour une compétence ne correspondra pas à un trois dans une autre). Mais, si nous partons du principe que la compétence ne peut s'observer que dans l'action, il pourrait être interressant de définir des niveaux génériques, comme par exemple: 
\begin{enumerate}
  \item Aucun: N'a jamais réalisé avec succès de tâches qui requièrent cette compétence.
  \item Débutant: A réalisé une tâche requérant cette compétence, mais soit elle était facile, soit elle a nécessité le coaching d'un collègue. 
  \item Confirmé: A réalisé plusieurs tâches avec succès requérant cette compétence de manière autonome
  \item Maitrise: Idem que le niveau 2, mais est en plus capable d'expliquer l'utilisation de cette compétence, de l'approfondir de manière autonome et, éventuellement, de donner formation sur le sujet.
\end{enumerate}
\paragraph{}Bien entendu, cette échelle n'est qu'une ébauche et ne peut probablement s'appliquer qu'aux compétences techniques et fonctionnelles. Toutefois, elle présente l'intérêt de pouvoir s'appliquer à toutes les compétences et d'être facile à observer, même si la composante "avec succès" peut être sujette à de nombreuses interprétations. Pour l'équipe technique, cette évaluation a l'avantage d'être centrée sur les tâches et d'être facile à appliquer, à trois conditions: que toutes les tâches effectuées par un employé soient répertoriées; que l'inventaire des compétences requises soit fait de manière systèmatique ; et que cette équipe travaille systématiquement par assignation de tâches. Ces tâches étant déjà définies, il ne reste "plus qu'à" définir les compétences requises pour chacune d'entre elles. 

\paragraph{} Nous avons donc, maintenant, une méthodologie pour lister les compétences, les décrire, en définir le niveau de maitrise et l'évaluer. Des exemples peuvent être trouvés en Annexe D. Deux éléments pourraient être rajoutés dans le référentiel: le socle de base et une liste de profils. Le socle de base consiste en une liste de "compétences minimales requises" sans lesquelles il est impossible de faire le travail de consultant technique. Ce socle servira de base pour déterminer la formation de tout nouvel arrivant dans l'équipe. Les profils cohérents consituent la suite du développement de l'individu une fois le socle de base maitrisé. Il est évident qu'un seul individu ne peut maitriser tous les aspects maitrisés par l'équipe. Les profils sont un ensemble cohérent de compétences qui permettent d'effectuer un ensemble de tâches. Les employés seront donc encouragés à développer leurs compétences vers un profil particulier afin de maximiser leur efficacité sur le type de tâche qui correspond au profil. Bien entendu, rien n'empêche l'employé d'essayer de correspondre à plusieurs profils, une fois le premier maitrisé.  


\paragraph{}Le référentiel n'est pas un outil qui permette de supporter les processus RH en lui-même. Il est davantage un outil de base duquel dérive toutes sortes d'outils utiles à la sélection, à l'oraganisation du travail, à la formation, à l'évaluation, à la rémunération et à la gestion de carrière.\citep[pp.29]{refcompetence}. Pour chaque objectif fixé, il faudra construire un outil approprié. 

\begin{description}
  \item[Planification des besoins] 
  Les besoins ne sont malheureusement pas conduits par la stratégie à long terme. Ils sont régis par les tâches journalières, par les besoins des clients et par la stratégie à court terme. Le référentiel listera toutes les compétences nécessaires aux besoins journaliers. Il suffira de faire l'inventaire du niveau de compétence nécessaire et du nombre de personnes qui doivent le maitriser. Il y a une autre source de besoins: comme nous sommes face à des personnes hautement qualifiées, il faut qu'elles aient l'opportunité de se développer au sein de l'entreprise. Il faut donc garder à l'esprit que les besoins individuels doivent rester compatibles avec ceux de l'entreprise. 
%TODO exemple de matrice

  \item[Organisation du travail, Développement et formation]
  Pour pouvoir rationnaliser la répartition du travail, il faudra trois outils. Le premier devra permettre de lister les compétences requises et le niveau à atteindre pour effectuer la tâche. Le second sera l'inventaire des compétences disponibles pour chacun. Le dernier devra lister les objectifs de chacun en terme de compétence. Ces trois outils permettront d'assigner la tâche aux personnes qui, ensemble, possèdent toutes les compétences requises pour l'effectuer, tout en tenant compte des objectifs de chacun. En outre, si le contexte et la difficulté de la tâche le permettent, la tâche sera assignée à des personnes qui auront l'opportunité de développer leurs compétences via son accomplissement. %TODO exemple 
  
  \item[Evaluation]
  Dans le référentiel, nous avons listé toutes les compétences nécessaires pour les besoins actuels et à venir. Nous avons défini plus haut la matrice des besoins. L'évaluation doit permettre de faire l'inventaire, avec l'employé, des compétences acquises et de leur degré d'acquisition. Le formulaire d'évaluation doit donc aussi permettre de lister les compétences à acquerir ou à perfectionner. Le formulaire d'évaluation devra lister les compétences et leurs niveaux respectifs, de façon à ce qu'il ne soit pas possible d'en oublier lors de l'entretien d'évaluation, mais il devra permettre de passer des compétences, sans que cela ne pose problème, car personne ne peut tout maitriser. Les objectifs en terme d'acquisition de compétences devront pousser à la construction de profils cohérents et, une fois un profil suffisant maitrisé, l'employé pourra s'étendre à d'autres en fonction des besoins de l'entreprise et de ses désirs. 
  
\paragraph{}Les niveaux d'acquisition de chaque compétence ont été définis dans le référentiel, mais rien n'indique dans le référentiel comment déterminer ces niveaux. Cette détermination devrait faire l'objet d'un chapitre entier, si pas d'un travail annexe. Elle diffère très fortement entre les compétences techniques et fonctionelles et entre les compétences génériques dites transversales. Pour ces dernières, la littérature est abondante et nous n'aurons pas le temps de nous y attarder dans ce travail. Pour les compétences techniques et fonctionnelles, nous pouvons explorer une piste. Il serait possible d'utiliser une méthode dérivée de la méthode des incidents critiques.\citep[pp.272]{gestionressourceshumaine2002}. Si l'on part du principe qu'une compétence ne se manifeste qu'à travers les actions\citep[pp.171]{gestionressourceshumaine2007} et du fait que, pour chaque tâche, nous avons établi les compétences et leur degré de maitrise nécessaire au bon accomplissement de la tâche, il serait en théorie possible d'inférer les compétences acquises sur base des tâches effectuées durant la période évaluée. Avec cette méthode, quid des compétences acquises et démontréeS lors de la période précédente? mais pas lors de la période en cours? Les compétences s'accumulent-elles d'années en années ? Y-a-t-il une date de péremption ? Cela est probable, car Odoo se situe dans un contexte qui évolue très vite. Certaines compétences seront suplantées par d'autres plus importantes, d'autres nécessiteront un entretien régulier. 
    \item[Rémunération]
    Un des objectifs est d'objectiver la rémunération individualisée. Il faudra être très prudent dans la mise en place d'une rémunération basée sur les compétences, car, en fonction des indicateurs choisis, l'un ou l'autre comportement sera encouragé. Doit-on encourager le développement de profils cohérents et de plus en plus spécialisés? ou doit-on encourager la polyvalence? Il est certain qu'il faut encourager le développement continu des employés vers des compétences clés ou qui seront clés dans le futur. Nous n'avons malheureusement pas l'occasion dans ce travail de statuer sur la question. Il faut savoir que nous n'avons pas énormément de latitude au sein du PS pour structurer la rémunération, mais nous avons la possibilité de décider du degré d'augmentation de chacun dans le budget qui a été imparti au département. Il serait possible d'imaginer un classement basé sur le développement des compétences de chacun en faisant l'hypothèse que ce qui est acquis n'est pas remis en question. Un point est octroyé au passage d'un niveau à un autre de compétence et à l'acquisiation de nouvelles compétences. Il est possible de donner des points lorsqu'on a développé les compétences requises pour un profil cohérent. Dans ce cas, les profils devront être définis avec soin et faire l'objet de l'approbation de toute l'équipe. Un dernier élément à prendre en compte pour la construction de la grille d'augmentation sont les aptitudes existantes, avec lesquelles il faut rester un maximum cohérent. Avant d'établir une nouvelle politique de rémunération, il faudra avoir fait un état des lieux précis des compétences et de la rémunération de chacun.
    
    \end{description}
    
    Le référentiel et les outils qui en découlent ont une nature dynamique. Odoo étant un éditeur de logiciel de gestion, il serait envisageable de concevoir un support informatisé et dynamique pour ce référentiel dans une base de donnée basée sur notre framework. Cette base de donnée permettrait de s'affranchir d'une représentation particulière et pourrait être construite avec la contribution de tous (par exemple, lorsque le besoin d'une nouvelle compétence se fait sentir, l'équipe l'ajouterait au référentiel). Elle permettrait de concevoir les différents outils que nous avons cité, basés sur le référentiel. En outre, à chaque fois que l'on ajouterait une compétence, celle-ci se retrouverait automatiquement dans le formulaire d'évaluation, dans l'inventaire des compétences, etc. L'intérêt de concevoir son propre outil est de pouvoir commencer très simplement et d'évoluer avec les besoins. 



\section{Piloter la mise en usage}
La mise en pratique ne posera pas de problème dans un premier temps puisque les utilisateurs seront les concepteurs. Néanmoins, il sera utile de mettre en place des métriques des utilisations. Comparer les objectifs fixés pour le PS et leurs réalisations. Mesurer le développement de chacun, mesurer le succès de l'organisation du travail: est-ce que les tâches assignées ont été remplies correctement et est-ce qu'elles ont permis le développement des compétences de la personne? Ces données pourront être utilisées pour convaincre le reste de la société du bien-fondé de la démarche. 

\section{Assurer la maintenance du système}
 Le réferentiel sera en mouvement permanent car son contenu est anticipatif et repose sur l’appréciation actuelle d’un futur moyen-terme et d’un environnement complexe et changeant. Hélas, en l'absence d'indications déterminées par le sommet hiérarchique, issues du plan stratégique, la nature anticipative devra être compensée par la dimension consultative de la construction et de la maintenance du référentiel : l’ensemble de l’équipe technique sera sollicité pour peaufiner et maintenir le référentiel. L'outil informatique accompagné d'un processus d'entretien devra permettre à chacun de contribuer à entretenir le processus. Dans l'idéal, il faudra pouvoir conserver le contexte et la date des modifications.
 
 
 \section{Conclusion}
 Dans ce chapitre, nous avons survolé rapidement la méthodologie pour la mise en place d'un référentiel et des outils qui en dérivent. La méthodologie nécessitera sûrement des adaptations lorsqu'elle se confrontera à la réalité, mais elle laisse entrevoir la possibilité d'atteindre les objectifs fixés pour la gestion des compétences au sein d'Odoo, à savoir : une rationalisation de l'organisation du travail, un meilleur pilotage des formations et du développement des employés, une plus grande pertinence donnée à l'évaluation, et une meilleure objectivation de la rémunération. Pour cela, il faudra implémenter la méthodologie définie dans ce chapitre.




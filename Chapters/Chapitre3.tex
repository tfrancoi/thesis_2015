\chapter{Élaboration d'un référentiel de compétence}
\section{Définition des tâches}

\begin{description}
    \item[Support] 
    \item[Formation basique]
    \item[Formation avancée]
    \item[Formation web]
    \item[Développement Backend]
    \item[Développement Front end]
    
\end{description}

\subsection{Les tâches impliquée dans les projets clients}
\begin{description}
    \item[Analyse Technique, estimation et faisabilité] 
    \item[Développement Backend]
    \item[Développement Front end]
    \item[Web Design]
    \item[Test fonctionnel]
    \item[Test automatique backend]
    \item[Test automatique frontend]
    \item[Revue de la qualité du code]
    \item[Intégration]
    \item[Documentation Technique]
    \item[Documentation Fonctionnelle]
    \item[Support]
    \item[Version Migration]
    \item[Data Migration]
    \item[Odoo Deployement]
    \item[Administration Postgresql]
    \item[Performance optimisation]
    \item[Méthode agile] 
    \item[Gestion des sources] -
\end{description}


\begin{description}
  \item[Senior Backend Developper] 
  Bonne connaissance du python
  Connaitre le SQL
  Il doit très bien connaitre le framework odoo. 
  Pouvoir traduire un besoin fonctionnel en une spécification technique.
  Définir le modèle de donnée adéquat pour modéliser une spécification fonctionnelle.
  être capabable de modifier des module existant
  être capable d'étendre des modules existant
  De créer des nouveaux module modules
  Avoir déjà participer à des projets dont la gestion était basée sur les méthodes agilent

  Connaitre les outils de débuggage python.
  Connaitre les outils de profiling python et postgresql
  Savoir comment concevoir et implémenter des tests automatiser pour le backend 
  Connaître bien git pour merger des branches, gérer les tags, revenir en arrière etc..
  Connaissance fonctionelle des modules principaux: CRM, Achat, Vente, Warehouse, Comptabilité
  
  être capable revoir la qualité du code et de donner un feedback.
  Connaitre les guideline de développement odoo. 
  être capable d'écrire des script RPC
  
  \item[Senior Frontend Developper] 
  Connaitre de manière approfondie le javascript
  Connaitre les outils de débuggage javascript
  Connaitre le HTML/CSS 
  être capable de créer de nouveau widget
  être capable d'étendre les widget existant
  Connaitre l'implémentation de la vue liste, form et search
  être capable d'écrire des test frontend avec phantomJS
  
\end{description}








\section{Gestion des ressources humaines} \cite{gestionressourceshumaine2007}


Rémunération basée sur l'individu et ses compétences: Poste est trop restrictif p241
``La notion de compétence est ainsi apparue comme répondant à ces besoins'' p241




Chapitre sur l'évaluation



1 Lister les compétences nécessaires
2 Définir les compétences 
3 Méthode d'évaluation des compétences
4 Formation pour augmenté les compétences
5 Prévisionelles quelle compétence vont être nécessaire dans le futur
4 Valorisation salariales en fonction des compétences et de leur évaluation

\section{Gestion des ressources humaines: Le développement de la gestion des compétences}


Pour pouvoir mesurer la compétence, il faut analyser le travail. Découper celui-ci en activités secondaire qui mène au résultat.
Mais pour des activités complexe ce n'est pas si facile de découper celle-ci en sous-tâches, parfois l'expert lui-même n'est pas capable de décrire son raisonement.

Attention il ne faut pas oublé le contexte dans lequel la compétence s'exerce. 

\paragraph{Méthode de repérage et de codification des compétences}



=> Listes de compétencs en fonction des activités de l'emploi

L'élaboration d'un référentiel de compétences soulèvent des problèmes. 
=> Problème de définition des compétences 
=> Problème de l'usage du référentiel de compétence: besoin du formateur, évaluateur et recruteur est différent
=> Problème de jugement : La reconnaissance de la compétence n'est dû qu'au jugement public. On ne peut pas se déclarer compétence sois-même. 



\paragraph{Les compétences un outil}
- Pour rationaliser le travail de l'équipe
- Repenser la contribution du salarié et de sa performance: ce n'est plus implement le diplôme. 
- Conformer le comportement des employés à de nouvelle norme d'action 
- Permet de définir des normes de coopération et d'échange. 

\paragraph{C'est une réponse à de nombreux problèmes}
Mécontentement des nouveaux entrants en termes d'évolution professionnelle
Pallier à l'insufissance de polyvalence

=> La gestion des compétences serait paré de toutes les vertus et la solutions de toutes sortes de problèmes

\begin{itemize}
    \item Favorise une redistribution des tâches et des responsabilités
    \item Elle accroit la mobilité professionnelle
\end{itemize}




La mise en place de la gestion des compétences devraient s'inscrire directement dans la démarch stratégique de l'entrerpise, alors que ca reste souvent déconnecté. 

Voir de la gestion des compétences à la gestion par les compétences. 


\section{Grille d'analyse de la gestion des compétences}

Page 192
\begin{enumerate}
    \item Quel est la stratégie de l'entreprise visée par le projet ?
    \item Quel problème l'entreprise essaye de régler par la gestion des compétences ?
    \item Lien entre compétence et stragégie 
    \item Comment la compétence est-elle définie
    \item Quel champ des ressources humaines: Recrutement, classification, rémunération, formaton, gestion des carières, etc.)
    \item Quels changement introduit la gestion des coméptences par rapport à l'existant
    \item Quels sont les outils utilisés
    \item Quels moyens sont-ils prévus  
\end{enumerate}
%TODO
% Demander à Johan les objectifs à atteindre à travers la gestion de compétence
% Demander à mon equipe de décrire les situations de travail et les compétences nécessaire. 


\section{Objectifs}
\begin{itemize}
    \item donner les outils pour avoir la possibilité d'objectivé les évaluations
    \item Pouvoir circonscrire le domaine de l'évaluation des salariés. 
    \item Organiser les formations en conséquence. 
    \item Meilleur organisation du travail. 
    \item Permettre un gestion de carrière. 
\end{itemize}

\section{Rémunération basée sur les compétences} \cite{gestionressourceshumaine2007}
Depuis les années 1990, et dans certain contexte d'entreprise. La rémunération se base sur les individus et ses compétences et non plus sur le poste. 

La notion de compétence est apparue comme répondant à ces besoins. 

Faut mettre en place un système de validation et de développement des compétences

Si l'individu est rémunérer en fonction de ces compétences, il revient à l'entreprise d'utiliser celle-ci au mieux. 

La compétence renvoie à un ensemble de connaissance et de capacités propres à l'individu et relatives à une situation de travail donnée. 

Les compétences sont une évolution du modèle taylorien, réquisent pour les nouvelles formes d'organisation du travail. Elle donne du sens au travail de l'individu et rend un part d'autonomie. Ils veulent être reconnu comme détenteurs d'une capacité d'action intelligente. 

La compétence s'impose commme un attribut de l'individu, elle est purement individuelle et attachée à une réalité de travail précis. 

=> Elle laisse croire que l'évaluation des compétences peut se faire sans débats ni conflits alors que c'est inhérent à tout ce qui touche les enjeux salariaux. 

L'approche de la rémunration par les compétences tant vers un plus grande individualisation de la rémunération.
Chez odoo, la rémunération est déjà individualisée, mais elle ressort plus de l'arbitraire qu'autre chose, même si on peut voir une corrélation entre le salaire, le diplôme et les années d'expérience. 

Il y a toujours eu tendance à différencier un maximum les emplois et donc les salaires, mais aujourd'hui, la notion de poste tant à disparaitre et donc cette différenciation aussi. C'est le cas chez Odoo en R\&D il y a des développeur, et c'est tout, dans le département service. On distingue les consultants fonctionnels des consultants techniques mais c'est tout. La notions de compétence permet de revenir à une meilleur différenciation. 

La classification par les compétences nécessite de pouvoir évaluer les compétences. Les compétences ne peuvent pas être saisient mais inférée à partir d'une situation de travail. Comment décrire les compétences et les évaluer. Décrire la situation de travail ? Ca ne suffit pas.


Compétences Horizontales : Pour les emplois d'un même niveau hiérarchique
Compétences Verticales: Pour l'encadrement
Compétences en profondeur : approfondissement de la maitrise de l'emploi actuel, pour les expert et pour les postes avec des emplois à fort contenu cognitif. 


Mesuré la variété de compétences, la spécialisation dans une compétence. 


Quels compétences veut-on rémunérer, à acquérir ou déjà acquises.

Est-ce qu'on se soucie de la performance réelle de l'entreprise, dans ce cas il ne faut que tenir compte des compétences effectivement utilisées dans les activité professionnelles. 



\section{Développement des compétences } \cite{gestionressourceshumaine2002}
La formation permet si on planifie le besoin en ressource humaine, de comblé les manques avec le personnels existant. 
Si on choisit un mode de formation.
Besoin pour les tâches répétitive, besoin à court termes et les besoins à long termes. 
Qui va suivre la formation
Qui va donner la formation 
Quelle méthode
Niveau d'apprentissage souhaité.
Le lieu de la formation (interne ou externe)

Pour les formations en interne, il faut former les formateurs et leur donner le temps de dispenser ces formations. 
Plus la méthode fait appel à la participation de l'employé plus l'information transmise sera retenue.

En conclusion on va faire du coaching en fonction des compétences: prendres des tâches qui requiert des compétences qu'on veut perfectionner sous la supervision d'un coach qui maitrise mieux cette compétence. 


\section{Evaluation des performance}
L'analyse des postes
La sélection et le placement 
La rémunération


Qui évaluer 
Qu'est-ce qu'on va évaluer
Comment évaluer (les méthodes) : Critère multiples où unique (par exemple le volume de vente chez les sales) 

Si il y a des critères multiple il faut déterminer le combinaison des critère
Score globale = f(c1,c2,...,cn)

Les sources d'information
Supérieur, Subordonné, Pairs, collègues, client, l'auto évaluation 

Ni les gestionnaire ni les employés n'ont confiance dans le système d'évaluation annuelle

Type d'évaluation utiliser chez Odoo: L'échelle d'évaluation conventionnelle.
Etant donné le contrôle totale qu'exerce d'évaluateur sur leur utilisation, ces formulaires peuvent donner lieu à des erreurs
d'indulgence, de sévérité, de tendance centrale ou à l'effet halo. 

%TODO (Définir toutes ces erreurs)

Gestion par objectif: Répandue pour les gestionnaires. Il est important de récompenser les personnes pour leur réalisation.Mais ce n'est pas toujours possible de traduire l'ensemble des activités d'un poste en objectifs. 
Source de motivation si ceux-ci participe à la définition des objectifs. 

Définir les objectifs, le temps pour les réaliser, ensuite les évaluers et finalement définir de nouveaux objectifs. 


\paragraph{}{Les facteurs qui expliques les échecs des évaluations de rendement}
\begin{enumerate}
    \item les facteurs contextuels et la motivation de l'évaluateur
    \item Fausses attentes et les postulats érronés concernant ce processus
\end{enumerate}

\paragraph{Les éléments à considérer pour améliorer le processus de rendement}
\begin{itemize}
    \item Evaluer une tâche à la fois
    \item peut observer le comportement sur base régulière
    \item Eviter l'emploi de moyenne qui peut avoir un sens différent pour chaque évaluateur
    \item Pas évaluer un grand groupe d'individu
    \item Dimension à évaluer sont faciles à comprendre, clairement définies et importantes
    \item Former les évaluateur
    
\end{itemize}


\section{L'appréciation} \cite{gestionressourceshumaine2007}


\begin{enumerate}
    \item Repondre aux besoins de feedback des employés
    \item Responsabiliser l'encadrement
    \item Faciliter la gestion du personnel: Evaluation des potentiels et gestion des carrières
    \item Favoriser la communication
    \item Servir de référence aux propositions d'augmentation de salaires
    \item Fournir des données pour la formation
    \item Améliorer la productivité, Rationaliser les décisions de gestion du personnel, valoriser les hommes
\end{enumerate} 

Sur quoi souhaitons centrer l'appréciation: sur la performance actuelle ou sur le potentiel ? 

Quoi évaluer: La personne, les résultats, les comportements, les compétences ou le potentiel. 

Evaluer les compétences soulève moins de réticences que les résultats que par exemple les résultats. 
Les résultats étant très fort dépendant des autres collègues, voir des clients, des machines etc.. 

Mais les compétences ne sont pas que des caractéristique individuels mais les compétences sont en relation avec le milieux dans lequel elle s'exerce. 

La maitrise de la compétence n'est pas me garant de la performance. 

Comment évaluer: Choisir le moment ou ca se passe, une grille d'évaluation et la validé.
Informer et former les appréciateurs.
Faire par vague
Documenter et archiver.

Il faut définir un référentiel: L'évaluation est un outils qui sert aussi à véhiculer le changement. 

L'entretien d'appréciation contient du fonctions: juger et développer 
L'appréciateur doit formuler un jugement. 
Mais il doit aussi donner des pistes pour améliorer les performances, dans le cadre d'un amélioration continue.

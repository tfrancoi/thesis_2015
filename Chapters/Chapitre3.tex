\chapter{Élaboration d'un référentiel de compétence}
L'élaboration d'un référentiel de compétence amène à prendre toutes une série de décisions. Cette élaboration peut être vu comme un processus qui va figé dans le marbre la définition et l'organisation du travail mais il n'en est rien, un référentiel est un outil qui évolue dans le temps et qui est conçu pour apréhender le futur. "S'il s'ancre dans le travail d'aujourd'hui, il vise essentiellement le travail de demain."\citep[pp.19]{refcompetence} Sachant cela nous ne resterons pas pétrifié par l'ampleur de la tâche et le nombre de mauvaises direction qu'il est possible de prendre. Après tout si il faut construire un réfétentiel de compétence pour Odoo, sa construction dans ce faire dans l'esprit de l'entreprise\footnote{Nous faisons ici référence aux deux dernières valeurs présentent dans le formulaire d'évaluation en annexe: "I want to move forward" et "I prefer to make things evolve than to not makes mistakes"} 

\paragraph{}L'introduction du livre "Élaborer des référentiels de compétences"\citep{refcompetence} propose une méthodologie en neuf étapes pour la mise en place et l'adoption du référentiel et son usage dans l'entreprise. Le processus est représenté à l'annexe C. 

\paragraph{} Les deux premières étapes du processus: "Se doter d'une définition de la compétence" et "Clarifier la finalité" ont été explicité dans le chapitre précédent. Il est interressant de noté que nous avons suivit une approche légèrement différente. Suite à l'article\citep{delobbe}, nous sommes parti de la finalité pour se doter de la définition appropriée de la compétence. Les étapes suivantes seront élaboré dans la suite de ce chapitre. Il est interressant de nous attarder un peu sur les étapes "soumettre à validation" et "organiser l'approbation par les acteurs" car ces étapes risquent de présenter des problèmes assez différents de ce à quoi on pourrait s'attendre dans une mise en place "classique" par le département des ressources humaines ou par le sommet hiérarchique.

\paragraph{} Ces étapes sont nécessaire pour assoire la légimité du référentiel. Le problème se pose généralement pour le département des ressources humaines et pour la hiérarchie de le faire accepter par les employés opérationels. Dans le cas présent, nous avons des éléments assez bas dans la hiérarchie qui définisse le référentiel. Et comme nous l'avons déjà expliqué dans le chapitre précédent, une partie du référentiel devra être construit avec les membres de l'équipe du PS. Il faudra bien sûr les faire valider par ceux-ci. Ceci ne devrait pas trop poser de problème mais il faudrait dans l'idéal le faire accepter par les ressources humaines et le sommet de la hiérarchie. C'est une problématique hautement politique et il est fort probable que non dans un premier temps. Il faut pluôt se poser la question de savoir si la gestion des compétences qui sera mis en place va permettre d'atteindre les objectifs fixés sans leur soutien. Il est fort probable que oui et une fois mis en place et fonctionelle, celle-ci sera beaucoup plus facile à promouvoir au sommet de la hiérarchie. Sachant cela, n'ayons pas peur d'avancer, nous allons donc maintenant décrire ce que nous allons faire lors de chacune des étapes. 

\section{Préciser le format, Recueillir les données et Traiter les données}
Le référentiel contiendra bien entendu une liste de compétences mais sous quelle forme et comment générer cette liste ? Dans le chapitre précédent nous avons déterminer que nous aurions à faire avec deux types bien disctint de compétences : les références génériques et transversalles qui pourront être utilisé par toutes les équipes et des compétences fonctionnelles et techniques qui seront fortement contextualisées est uniques à chaque équipe. Pour les compétences génériques, la littérature est assez abondatante, nous pourrons nous basé sur quelquechose d'existant comme ici\citep{exemple_ref} en l'adaptant au format choisit. Il est aussi possible de partir du travail déjà effectué par le département des ressources humaines. Les compétences techniques et fonctionelles sont quant-à-elles spécifique à l'équipe du PS. Il faudra donc les construires à partir de rien et nous allons surtout nous concentrer sur cette partie. 

\paragraph{}Si on considère que les compétences sont ancrées dans l'action et que celles-ci se manifestent dans l'action, les tâches et les processus semblent un bon point de départ pour établir une liste de compétence.  Dans l'annexe D, nous listons les processus qu'il faudra décortiquer en tâche. A partir de chaque tâche, nous listons les compétences nécessaires. Pour faire ressortir les compétences comportementales, il est interressant de noté avec qui intéragi la personne qui effectue la tâche comme évoqué dans \citep[pp. 185]{refcompetence}. Le processus de projet est découpé en tâches et chaque tâche est analysée dans l'annexe D. Pour receuillir toutes les informations sur les processus, les tâches et les compétences requies, le councours des équipes sera nécessaire. Une fois, ces données récupérées, il faudra les traités par une personne qui connait bien le metier. Il pourra ainsi éliminé les doublons, proposer une classification de chaque compétence. En parallèle chaque compétence devra se doter d'une définition de différent niveau de maitrise. Finalement certaines compétences vont dépendre d'autre compétence par exemple la capacité à revoir la qualité du code produit par un collègue nécessite de maitriser les domaines impliqués dans son programmes, pour pouvoir donner une formation technique, il faudra connaitre parfaitement son contenu, etc.. 

\paragraph{} Il est important de s'arrêter un instant sur le niveau de maitrise de chaque compétence. Il existe ici de nombreuses possibilités, l'une d'entre elle consiste à déterminer un certain nombre de niveau (par exemple trois) et dans chaque description de compétence contiendra l'explication de ces niveaux dans le cadre de la compétence. Avec cette méthodologie on risque très fort de se retrouver avec des niveaux peut cohérent entre les compétences où un trois pour une compétence ne correspondra pas à un trois dans une autre. Si nous repartons du principe que la compétence ne peut s'observer que dans l'action, il pourrait être interressant de définir plusieurs niveau générique comme par exemple: 
\begin{enumerate}
  \item Aucun: N'a jamais réaliser de tâches avec succès qui requiert cette compétence.
  \item Débutant: A réalisé une tâche requérant cette compétence qui était soit facile où qui a nécessité le coaching d'un collègue. 
  \item Confirmé: A réalisé plusieurs tâches avec succès requérant cette compétence de manière autonome
  \item Maitrise: Idem que le niveau 2, mais en plus est capable d'expliqué l'utilisation de cette compétence, de l'approfondir de manière autonome et éventuellement de donner formation.
\end{enumerate}
\paragraph{}Bien entendu cette échelle n'est qu'une ébauche et ne peut probablement ne s'appliqué qu'au compétence techniques et fonctionnelles, mais elle à l'intérêt de pouvoir s'appliquer à toutes les compétences et d'être facile à observer même si la composante "avec succès" peut être sujet à de nombreuse interprétation. Pour l'équipe technique, cette évaluation à l'intérêt d'être centré sur les tâches et d'être aisée si toutes les tâches effectué par un employé sont répertoriées et que l'inventaire des compétences requises est faites de manière systèmatique et justement cette équipe travail déjà systématiquement par assignation de tâches. Ces tâches sont déjà archivées, il ne reste "plus qu'à" définir les compétences requises pour chacune d'entre elle. 

\paragraph{} Nous avons donc maintenant, une méthodologie pour lister les compétences, les décrires, définit un niveau de maitrise pour chacune d'entre elle ainsi qu'un moyen de les évaluer. Des exemples peuvent être trouver en Annexe D. Deux éléments pourrait être rajouté dans le référentiel: le socle de base et une liste de profils. Le socle de base consiste en une liste de compétence minimal requise sans lequel il est impossible de pouvoir faire le travail de consultant technique. Ce socle servira de base pour déterminer la formation de tout les nouveaux arrivant dans l'équipe techniques. Les profils cohérent consituent la suite du développement de l'individu une fois le socle de base maitrisé. Il est évident qu'un seul individu ne peut pas maitriser tout les aspects maitrisée par l'équipe et les profils sont un ensemble cohérent de compétence qui permet d'effectué un certain ensemble de tâche. Les employés seront donc encourager de développer leur compétence vers un profils particulier afin de maximiser leur efficacité sur le type de tâche qui correspond au profils. Bien entendu rien n'empêche à l'employé de vouloir correspondre à plusieurs profils une fois le premier maitrisé.  



\paragraph{}Le référentiel n'est pas un outil qui permet de supporter les processus RH en sois. Le référentiel est plutôt un outil de base duquel dérive toutes sortes d'outils utiles à la sélection, à l'oraganisation du travail, à la formation, à l'évaluation, à la rémunération et à la gestion de carrière.\citep[pp.29]{refcompetence}. Pour chaque objectif que fixés, il faudra construire un outil approprié. 

\begin{description}
  \item[Planification des besoins] 
  Les besoins ne sont hélas pas conduit par la stratégie à long terme, ils sont régis par les tâches journalière et les besoins des clients et la stratégie à cours terme. Le référentiel listera toutes les compétences nécessaires au journalier. Il suffira de faire l'inventaire du niveau de compétence nécessaire et le nombre de personnes qui doivent les posséder. Il y a une autre source de besoin, comme nous sommes face à des personnes hautement qualifié, il faut qu'ils aient l'opportunité de se développer au sein de l'entreprise. Donc même si tout les besoins serton comblé il faudra toujours garder à l'esprit qu'en plus des besoins de l'entreprise il faut garder une trace des besoins de chacuns qui doivent bien entendu rester compatible avec ceux de l'entreprise. 
%TODO exemple de matrice

  \item[Organisation du travail, Développement et formation]
  Pour pouvoir rationnalisé la répartition du travail, il faudra trois outils. Le premier devra permettre de lister les compétences requises et leur niveau pour effectuer la tâche. Le second sera l'inventaire des compétences disponible pour chacun. Et le dernier devra lister les objectifs de chacun en terme de compétence. Ces trois outils permettrons d'assigner aux personnes qui ensemble possède toutes les compétences requis pour effectué là tâche, le plus souvent il n'y aura qu'une personne, mais il faudra aussi tenir compte des objectifs de chacun et si le contexte et la difficulté de la tâche le permet, assigné des personnes qui auront l'opportunité de développer leur compétence dans l'accomplissement de la tâche. %TODO exemple 
  
  \item[Evaluation]
  Dans le référentiel nous avons listé toutes les compétences nécessaires à l'heure actuelle et à venir. Nous avons définis plus haut la matrice des besoins. L'évaluation doit permettre de faire l'inventaire avec l'employé des compétences acquises et leur degré d'acquisition. Le formulaire d'évaluation donc aussi permettre de lister les compétences à acquerir ou à perfectionner. Le formulaire d'évaluation se devra de lister les compétences et leur niveau respectifs pour qu'il ne soit pas possible d'en oublier lors de l'entretien d'évaluation mais il devra permettre des laisser des compétences vides sans que cela ne pose problème car personne ne peut maitriser tout les aspects. Les objectifs en termes d'acquisition de compétence devront poussés à la construction de profils cohérents et une fois un profils suffisant maitrisé, l'employé pourra s'étendre à d'autres en fonction des besoins de l'entreprise et de ses désirs. 
  
\paragraph{}Les niveaux d'acquisition de chaque compétence ont été définit dans le référentiel mais rien n'indique dans le référentiel comment déterminer ce niveau. Cette évaluation devrait faire l'objet d'un chapitre entier si pas d'un travail annexe, il diffère très fortement entre les compétences techniques et fonctionelles et entre les compétences génériques dites transversales. Pour ces dernière, la littérature est abondante et nous n'avons hélas pas le temps de nous y pencher dans ce travail. Pour les compétences techniques et fonctionnelles, nous pouvons explorer une piste. Il serait possible d'utiliser une méthode dérivée de la méthode des incidents critiques.\citep[pp.272]{gestionressourceshumaine2002}. Si on part du principe qu'une compétence ne se manifeste qu'à travers les actions\citep[pp.171]{gestionressourceshumaine2007} et du fait que pour chaque tâche nous avons établit les compétences et leur degré de maitrise nécessaire au bon accomplissement de la tâche, il serait en théorie possible d'inférer les compétences acquises sur base des tâches effectuées durant la période évaluées. Avec cette méthode, quid des compétences acquises et démontré lors de la période précédente mais pas lors de cette période. Les compétences s'accumulent-elles d'années en années ? Y-a-t-il une date de péremption ? Probablement car Odoo se situe dans un contexte qui évolue très vite. Certaine compétence seront suplanté par d'autre plus importantes, d'autres nécessiteront un entretien régulier. 
    \item[Rémunération]
    Un des objectifs est d'objectiver la rémunération individualisée. Il faudra être très prudent dans la mise en place d'une rémunération basée sur les compétences car en fonction des indicateurs choisit ceux-ci vous encourager l'un ou l'autre comportement. Veut-on encourager le développement de profils cohérent et de plus en plus spécialisé. Veut-on encourager la polyvalence. Il est certain qu'il faut encourager le développement continu des employés vers les compétences qui sont clés et qui seront clés pour le futur. Nous n'avons hélas pas l'occasion dans ce travail de statuer sur la question. Il faut savoir que nous n'avons pas énormément de latitude au sein du PS pour structuré la rémunération mais nous avons la possibilité de décidé le degré d'augmentation de chacun dans le budget qui a été imparti au département. Il serait possible d'imaginer un classement basé sur le développement des compétences de chacun en faisant l'hypothèse que ce qui est acquis n'est pas remis en question. Une poid est donné pour le passage d'un niveau à un autre dans une compétence donnée et à l'acquisiation de nouvel compétence. Il est possible de donné des points losqu'on a développé les compétences requises pour un profil cohérent. Danc ce cas ceux-ci devrons être définit avec soin et faire l'objet de l'appobation de toute l'équipe. Un denier élément à prendre en compte pour la construction de la grille d'augmentation est l'existant. Il faut rester un maximum cohérent avec l'existent et donc avant d'établir une nouvelle politique de rémunération il faudra avoir fait un état des lieux précis des compétence et de la rémunération de chacun.
    
    \end{description}
    
    Le référentiel et les outils qui en découlent ont une nature dynamique. Odoo étant un editeur de logiciel de gestion, il serait envisageable de concevoir un support informatisé et dynamique pour ce référentiel dans une base de donnée basée sur notre framework. Cette base de donnée permettrait de s'affranchir d'une représentation particulière, pourrait être construite avec la contribution de tous (par exemple lorsque le besoin d'une nouvelle compétence se fait sentir l'ajouter au référentiel). Et elle permettrait de convevoir touts les outils que nous avons cité basé sur le référentiel, lorsqu'on rajoute une compétence celle-ci se trouverait automatiquement ajouté dans le formulaire d'évaluation, dans l'inventaire des compétences etc. L'intérêt de concevoir son propre outil est de pouvoir commencer très simplement et d'évoluer avec les besoins qui évolueront. 



\section{Piloter la mise en usage}
La mise en usage ne posera pas de problème dans un premier temps puisque les utilisateurs seront les concepteurs. Néanmoins, il sera utile de garder des métriques des utilisations. Comparer les objectifs fixés pour le PS et leur réalisation. Mesurer le développement de chacun, mesurer le succès de l'organisation du travail: est-ce que les tâches assignées ont été remplie correctement et est-ce que celle-ci on permis le développement des compétences de la personnes ? Ces données pourront être utilisées pour convaincre le reste de la société du bien fondé de la démarche. 

\section{Assurer la maintenance du système}
La maintenance du système devra ce faire non seulement par ses concepteurs mais aussi par tout les utilisateurs, surtout si cette base d'utilisateur grandit en dehors du PS. L'outil informatique accompagné d'un processus d'entretien pourra permettre à chacun de contribuer et d'entretenir le processus. Celui-ci pourra se faire suite à l'apparition de nouveau besoin ou à la disparition d'autre, suite aux entretiens d'évaluations. Dans l'idéal, il faudra pouvoir conserver le contexte et la date des modifications pour conserver l'historique.
 
 
 \section{Conclusion}
 Dans ce chapitre nous avons survolé rapidement la méthodologie pour la mise en place d'un référentiel et des outils qui en dérivent. La méthodologie nécessitera surement des adaptations lorsqu'elle se confrontera à la réalité, mais elle laisse entrevoir la possibilité d'atteindre les objectifs fixé pour la gestion des compétences au sein d'Odoo. A savoir une rationalisation de l'organisation du travail, avoir un meilleur pilotage des formations et du développement des employés, rendre son sens au évaluation et objectivé la rémunération. Pour cela, il faudra implémenté la méthodologie définie dans ce chapitre.





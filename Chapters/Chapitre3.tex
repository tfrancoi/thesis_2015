\chapter{Élaboration d'un référentiel de compétence}
L'élaboration d'un référentiel de compétence amène à prendre toutes une série de décisions. Cette élaboration peut être vu comme un processus qui va figé dans le marbre la définition et l'organisation du travail mais il n'en est rien, un référentiel est un outil qui évolue dans le temps et qui est conçu pour apréhender le futur. "S'il s'ancre dans le travail d'aujourd'hui, il vise essentiellement le travail de demain."\citep[pp.19]{refcompetence} Sachant cela nous ne resterons pas pétrifié par l'ampleur de la tâche et le nombre de mauvaises direction qu'il est possible de prendre. Après tout si il faut construire un réfétentiel de compétence pour Odoo, sa construction dans ce faire dans l'esprit de l'entreprise\footnote{Nous faisons ici référence aux deux dernières valeurs présentent dans le formulaire d'évaluation en annexe: "I want to move forward" et "I prefer to make things evolve than to not makes mistakes"} 

\paragraph{}L'introduction du livre "Élaborer des référentiels de compétences"\citep{refcompetence} propose une méthodologie en neuf étapes pour la mise en place et l'adoption du référentiel et son usage dans l'entreprise. Le processus est représenté à l'annexe C. 

\paragraph{} Les deux premières étapes du processus: "Se doter d'une définition de la compétence" et "Clarifier la finalité" ont été explicité dans le chapitre précédent. Il est interressant de noté que nous avons suivit une approche légèrement différente. Suite à l'article\citep{delobbe}, nous sommes parti de la finalité pour se doter de la définition appropriée de la compétence. Les étapes suivantes seront élaboré dans la suite de ce chapitre. Il est interressant de nous attarder un peu sur les étapes "soumettre à validation" et "organiser l'approbation par les acteurs" car ces étapes risquent de présenter des problèmes assez différents de ce à quoi on pourrait s'attendre dans une mise en place "classique" par le département des ressources humaines ou par le sommet hiérarchique.

\paragraph{} Ces étapes sont nécessaire pour assoire la légimité du référentiel. Le problème se pose généralement pour le département des ressources humaines et pour la hiérarchie de le faire accepter par les employés opérationels. Dans le cas présent, nous avons des éléments assez bas dans la hiérarchie qui définisse le référentiel. Et comme nous l'avons déjà expliqué dans le chapitre précédent, une partie du référentiel devra être construit avec les membres de l'équipe du PS. Il faudra bien sûr les faire valider par ceux-ci. Ceci ne devrait pas trop poser de problème mais il faudrait dans l'idéal le faire accepter par les ressources humaines et le sommet de la hiérarchie. C'est une problématique hautement politique et il est fort probable que non dans un premier temps. Il faut pluôt se poser la question de savoir si la gestion des compétences qui sera mis en place va permettre d'atteindre les objectifs fixés sans leur soutien. Il est fort probable que oui et une fois mis en place et fonctionelle, celle-ci sera beaucoup plus facile à promouvoir au sommet de la hiérarchie. Sachant cela, n'ayons pas peur d'avancer, nous allons donc maintenant décrire ce que nous allons faire lors de chacune des étapes. 

\section{Préciser le format}
Partir des tâches à effectuer

Définir le socle de base

Définir des profils cohérent pour permettre à un minimum de personne de travailler sur une tâche donnée mais aussi parce les compétences sont liées entre elle. 
\section{Recueillir les données}

\section{Traiter les données}


\section{Piloter la mise en usage}
Le référentiel n'est pas un outil qui permet de supporter les processus RH en sois. Le référentiel est plutôt un outil de base duquel dérive toutes sortes d'outils utiles à la sélection, à l'oraganisation du travail, à la formation, à l'évaluation, à la rémunération et à la gestion de carrière.\citep[pp.29]{refcompetence}. Pour chaque objectif que fixés, il faudra construire un outil approprié. 

\begin{description}
  \item[Planification des besoins] 
  Les besoins ne sont hélas pas conduit par la stratégie à long terme, ils sont régis par les tâches journalière et les besoins des clients et la stratégie à cours terme. Le référentiel listera toutes les compétences nécessaires au journalier. Il suffira de faire l'inventaire du niveau de compétence nécessaire et le nombre de personnes qui doivent les posséder. Il y a une autre source de besoin, comme nous sommes face à des personnes hautement qualifié, il faut qu'ils aient l'opportunité de se développer au sein de l'entreprise. Donc même si tout les besoins serton comblé il faudra toujours garder à l'esprit qu'en plus des besoins de l'entreprise il faut garder une trace des besoins de chacuns qui doivent bien entendu rester compatible avec ceux de l'entreprise. 
%TODO exemple de matrice

  \item[Organisation du travail, Développement et formation]
  Pour pouvoir rationnalisé la répartition du travail, il faudra trois outils. Le premier devra permettre de lister les compétences requises et leur niveau pour effectuer la tâche. Le second sera l'inventaire des compétences disponible pour chacun. Et le dernier devra lister les objectifs de chacun en terme de compétence. Ces trois outils permettrons d'assigner aux personnes qui ensemble possède toutes les compétences requis pour effectué là tâche, le plus souvent il n'y aura qu'une personne, mais il faudra aussi tenir compte des objectifs de chacun et si le contexte et la difficulté de la tâche le permet, assigné des personnes qui auront l'opportunité de développer leur compétence dans l'accomplissement de la tâche. %TODO exemple 
  
  \item[Evaluation]
  Dans le référentiel nous avons listé toutes les compétences nécessaires à l'heure actuelle et à venir. Nous avons définis plus haut la matrice des besoins. L'évaluation doit permettre de faire l'inventaire avec l'employé des compétences acquises et leur degré d'acquisition. Il devra aussi permettre de lister les compétences à acquerir ou à perfectionner. Le formulaire d'évaluation se devra de lister les compétences et leur niveau respectifs pour qu'il ne soit pas possible d'en oublier lors de l'entretien d'évaluation mais il devra permettre des laisser des compétences vides sans que cela ne pose problème car personne ne peut maitriser tout les aspects. Les objectifs en termes d'acquisition de compétence devront poussés à la construction de profils cohérents et une fois un profils suffisant maitrisé, l'employé pourra s'étendre à d'autres en fonction des besoins de l'entreprise et de ses désirs. 

\paragraph{}Les niveaux d'acquisition de chaque compétence ont été définit dans le référentiel mais rien n'indique dans le référentiel comment déterminer ce niveau. Cette évaluation devrait faire l'objet d'un chapitre entier si pas d'un travail annexe, il diffère très fortement entre les compétences techniques et fonctionelles et entre les compétences génériques dites transversales. Pour ces dernière, la littérature est abondante et nous n'avons hélas pas le temps de nous y pencher dans ce travail. Pour les compétences techniques et fonctionnelles, nous pouvons explorer une piste. Il serait possible d'utiliser une méthode dérivée de la méthode des incidents critiques.\citep[pp.272]{gestionressourceshumaine2002}. Si on part du principe qu'une compétence ne se manifeste qu'à travers les actions\citep[pp.171]{gestionressourceshumaine2007} et du fait que pour chaque tâche nous avons établit les compétences et leur degré de maitrise nécessaire au bon accomplissement de la tâche, il serait en théorie possible d'inférer les compétences acquises sur base des tâches effectuées durant la période évaluées. Avec cette méthode, quid des compétences acquises et démontré lors de la période précédente mais pas lors de cette période. Les compétences s'accumulent-elles d'années en années ? Y-a-t-il une date de péremption ? Probablement car Odoo se situe dans un contexte qui évolue très vite. Certaine compétence seront suplanté par d'autre plus importantes, d'autres nécessiteront un entretien régulier. 

  \item[Rémunération]
\end{description}


\section{Assurer la maintenance du système}

 





\chapter*{Introduction}
\paragraph{}Les pays industrialisés font face à un vieillissement de leurs populations. Pour l’Europe, ce phénomène est sans doute l’un des plus grands défis auquel elle devra faire face ces prochaines années. En effet, celui-ci risque de remettre en cause le fondement même du système social européen qui reste actuellement l’un des plus développés au monde. Les rapports sur la question sont unanimes et leurs conclusions sont plus alarmantes les uns que les autres. Les réponses à ces questions semblent toutes trouvées. Il ne se passe pas un jour sans que l’on nous parle de la nécessité de la réforme du système des pensions et de notre système sociale dans son ensemble. Il nous semble que ces types de propositions sont issus de visions court-termistes qui ne prennent pas en compte l’ensemble du phénomène et sa complexité. Nous pensons qu’il est nécessaire d’adopter une stratégie plus globale, c’est la raison pour laquelle nous avons décidé de développer dans cette étude des pistes de solutions que les pays européens pourraient adopter pour apporter une réponse durable à ce phénomène.
 
\paragraph{}Pour comprendre ce défi, il est indispensable d’analyser l’ampleur du phénomène de vieillissement des populations.  Pour ce faire, dans un premier temps nous définirons ce que l’on entend par vieillissement des populations et nous tenterons de mesurer son ampleur et d’évaluer son évolution. Ensuite, nous détaillerons les impacts du vieillissement sur nos sociétés. Ceux-ci comme nous le verrons sont multiples. En effet, outre son impact sur le financement des pensions et le financement des soins de santé, nous constaterons que  le vieillissement des populations impact également les forces de travail disponibles sur le territoire européen et affecte ainsi l’ensemble de l’économie.
 
\paragraph{}Après avoir posé le problème et ses différents paramètres, nous explorerons différentes pistes de solutions stratégiques. D’abord l’augmentation de la natalité qui permettra de garantir à long terme une base de travailleur mais qui nécessite des investissements importants à court terme. Ensuite, nous envisagerons différentes politiques  d’immigration qui pourraient venir pallier à court terme au manque d’actif. Finalement, nous nous attarderons sur la croissance économique. Dans ce domaine, au vu du sujet de note étude, il nous a semblé opportun de développer plus particulièrement l’économie argentée ou Silver Economy . En effet, une telle stratégie permettrait de transformer le vieillissement qui est présenté jusqu’ici comme une fatalité en opportunité économique. Chacune de ces solutions prisent séparément ne peuvent régler le problème mais leur combinaison permettrait sans doute de relever le défi.  

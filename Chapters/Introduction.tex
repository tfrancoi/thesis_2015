\chapter*{Introduction}
\paragraph{}
Odoo est un éditeur de progiciels libres qui vient de souffler ses dix premières bougies. L'augmentation des effectifs y constitue l'un des plus grand défis. Une fois son software publié, le modèle économique repose sur la qualité, la vélocité et le degré d’industrialisation des services de la société. Ainsi, les ressources humaines, et plus spécialement leurs compétences, doivent assurer sa pérennité et sa croissance vers la première place du marché de l'ERP\footnote{Entreprise Ressource Planning ou en francais, progiciel de gestion intégré}. Toutefois, à l'heure actuelle, la gestion des ressources humaines n'a pas évolué à la hauteur de ces ambitions et la gestion des compétences y est absente. 

\paragraph{}Le présent document se focalise sur la détermination d'un modèle de gestion des compétences qui doit assurer la prise en charge des défits de planification, de formation, d'évaluation et de rémunération des individus. La mise en place d'un tel système nécessite le concours du top management et la mise en pratique de la ligne stratégique décidée. Au jour d'aujourd'hui, il apparait qu'Odoo ne possède pas de plan opérationel découlant de la ligne stratégique. La volonté de mise en place d'une gestion des compétences vient directement du département des opérations. Cette gestion ne pourra donc, hélas, que partir des constats faits par celui-ci. 

\paragraph{}Sachant cela, nous allons, pour commencer, faire un état des lieux et constater les manques présents dans la gestion des ressources humaines actuelle. Nous définirons ainsi les objectifs à atteindre pour la mise en place de la gestion des compétences. Il faudra ensuite déterminer quel modèle mettre en place pour que celle-ci soit une réussite. Nous verrons que cela dépend non seulement des objectifs mais aussi de la structure de l'organisation. Finalement, la mise en place de la gestion des compétences passe par la définition d'un référentiel. Si nous ne prétendons pas créer un référentiel exhaustif pour Odoo dans ce travail, nous espérons pouvoir définir une méthodologie, qui permettra la mise en place de celui-ci, ainsi que des outils connexes, qui permettront d'atteindre les objectifs fixés pour les différents aspects de la gestion des ressources humaines. 

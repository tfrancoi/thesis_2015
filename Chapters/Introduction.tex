\chapter*{Introduction}
\paragraph{}L’augmentation continue de l’effectif au sein d’Odoo, éditeur de progiciels libres, qui vient de souffler ses 10 bougies seulement, constitue des défis importants. Une fois son software publié, son modèle économique repose sur la qualité, la vélocité et le degré d’industrialisation de ses services. Ainsi, ses ressources humaines et plus spécialement leurs compétences doivent assurer la pérennité et la croissance vers la première place du marché de l'ERP. Mais à l'heure actuelle, la gestion des ressources humaines n'a pas évolué à la hauteur des ambitions et la gestion des compétences en est absente. 

\paragraph{}Ce présent document focalise sur la détermination d'un modèle de gestion des compétences qui doit assurer la prise en charge de ces défits en terme de planification, de formation, d'évaluation et de rémunétation des individus. La mise en place d'un tel système nécessite le concours du top management et la mise en pratique de la ligne stratégique décidé. Néamoins, il apparait très vite qu'Odoo ne possède pas de plan opérationel qui découle de la ligne stratégique. Le moteur de la mise en place d'une gestion des compétence vient directement des opérations. Cette gestion ne pourra hélas partir que des constats fait par celles-ci. 

\paragraph{}Sachant cela, nous allons dans un premier temps tirer les constats et les manques dans la gestion des ressources humaines actuelle et définir les objectifs à atteindre pour la mise en place de la gestion des compétences. Il faudra ensuite déterminer quelle modèle mettre en place pour que celui-ci soit une réussite. Nous verrons que celui-ci dépend non seulement des objectifs mais aussi de la structure de l'organisation. Finalement, le mise en place de la gestion des compétences passe par la définition d'un référentiel. Si nous ne prétendons pas vouloir définir le référentiel complet pour Odoo dans ce travail, nous espérons pouvoir définir une méthodologie qui permettra la mise en place de celui-ci et de ses outils connexe qui permettrons d'atteindre les objectifs fixé pour les différents aspect de la gestion des ressources humaines. 

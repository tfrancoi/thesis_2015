\chapter*{Conclusion}
Le travail accomplit ici pourrait facilement être répliqué pour le département de recherche et développement qui nécessite principalement des connaissance techniques pointues. Avec un travail plus poussé sur les compétences génériques, il pourrait aussi s'appliquer à l'équipe des consultants fonctionnels et l'ambition est bien là au sein d'Odoo pour faire ce travail. 

Il faut noté qu'il y a un biais important dans la défintion des objectifs et la finalité de l'outils. L'initiative vient des acteurs de terrain et ceux-ci ont leur objectifs propres. L'outils devra surement évoluer si celui-ci veut s'adapter au besoin des ressources humaines ou à la hiérarchie. 

Dans le troisième chapitre nous n'avons fait qu'effleuré la mise au point d'un référentiel.

Ce travail ne suffit pas à mettre en place une gestion des compétences au sein d'Odoo mais il porte en lui l'impulsion qui va le permettre.  

Porté un regard critique sur la gestion des compétences, Faut-il brûler la gestion des compétences. Pas de cadre normatif, il faudra tester empiriquement la méthodologie basée sur les besoins et l'amélioré en continue sur base du vécu sur le terrain.\citep[pp.252-253]{competencesbruler2006}

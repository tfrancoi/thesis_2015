\chapter*{Conclusion}
\paragraph{}Au terme de cette étude, nous pouvons tirer les conclusions suivantes. Le premier chapitre définit la notion de vieillissement de la population et mesurer ce phénomène. Il est indéniable et il s'accentuera jusqu'en 2050. Il se stabilisera en 2070 un peu en dessous de 2050 mais bien au-dessus de 2015. Les défis à relever pour les pays de l'Union européenne sont donc importants. Le second chapitre détaille les impacts de ce vieillissement,  le financement des pensions basé sur un système de répartition ne peut être tenable sous sa forme actuelle. Le besoin d’une réforme du système de soin de santé est moins évident car même si les dépenses augmentent par rapport au PIB, la part du vieillissement dans cette augmentation est négligeable. L'évolution des coûts des soins de santé est imputable aux progrès technologiques qui génèrent l’augmentation de la consommation dans toutes les tranches de la population.
 
\paragraph{}Pour répondre de manière structurelle à ces défis nous avons développé trois solutions que nous estimons stratégiques. Les premières parties de cette étude nous ont permis d’identifier les causes du vieillissement de la population. La première d’entre elles est le faible taux de natalité qui induit un vieillissement par le bas. Dès lors, il nous a semblé évident qu’une solution à cette situation consiste à l’augmentation de ce taux. Les politiques à mettre en place consistent à minimiser les coûts directs et indirects de la natalité, permettre aux parents de concilier leur vie de famille et leur travail et donner une stabilité économique aux femmes.
 
\paragraph{}Au terme de cette première solution nous avons constaté que les effets de celles-ci ne peuvent être escomptés qu’à long terme. En effet, il faudrait attendre minimum 20 ans pour qu’un nouveau-né puisse entrer sur le marché du travail. Dès lors, l’immigration nous a paru une solution qui permet de combler à court terme ce manque de main-d’œuvre annoncé par les projections démographiques. Cette solution a un double avantage.  D’une part, elle permet de remédier au vieillissement par le haut en injectant directement dans l’économie des forces de travail. D’autre part, nous avons constaté que les immigrés ont un taux de natalité supérieur aux autochtones ce qui permet également de relever rapidement ce taux. Cette solution comporte quelques risques que nous avons développés. Pour que l’immigration soit une solution optimale il est indispensable d’intégrer ces nouveaux venus dans le pays d’accueil et de supprimer les barrières à l’embauche pour les activer le plus rapidement possible. Nous avons pu constater qu’un politique d’immigration économique choisie permet de cibler les migrants nécessaires au développement du pays et d’optimiser cette solution.
 
 
\paragraph{}Avoir une force de travail disponible immédiatement grâce à l’immigration n’a pas de sens si les pays européens ne peuvent pas offrir à ces nouveaux venus des emplois à occuper. La dernière solution que nous avons donc développée porte sur la croissance économique et plus particulièrement sur la Silver Economy qui transforme le vieillissement en une belle opportunité. Nous avons cependant constaté qu’il s’agit d’un marché très particulier. En effet, les changements démographiques mènent  à des changements de marchés et de produits. Cette évolution devrait pousser à développer de nouveaux articles ou à adapter les produits existant pour conquérir  ce nouveau marché que constituent les seniors.
 
\paragraph{}Cette dernière solution que nous avons exploré reste dans le courant capitaliste qui se base sur l'utopie de la croissance infinie. On considère que la croissance de la population permettra une croissance économique. Nous savons aussi maintenant que la croissance infinie n'est plus possible sur une planète finie et faute de pouvoir coloniser d'autres planètes, il semble légitime de se demander si cette course à la croissance est soutenable. Les politiques nous proposent des solutions à court terme, nous avons exploré des solutions à plus long terme mais ne serait-il pas nécessaire de regarder encore plus loin ? 
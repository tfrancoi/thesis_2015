\chapter*{Conclusion Générale}
\paragraph{}Ce document avait comme objectifs de déterminer quelle gestion des compétences Odoo devait adopter et d'en proposer une méthodologie de mise en place. Mettre en place un modèle de gestion des compétences et le faire évoluer dépend normalement en grande partie de la capacité du sommet hiérarchique à traduire la stratégie de l’entreprise en opérations assurant qualité et profitabilité dans un contexte de marché à géométrie variable, tout en tenant compte des spécificités de l’individu. Hélas, le contexte particulier d'Odoo fait que le moteur de changement ne se trouve pas au niveau du top management. Nous sommes donc partis de constats factuels. 

\paragraph{}Il est apparu très vite qu'avant de choisir une méthode, il fallait déterminer non seulement les objectifs attendus, mais aussi la structure organisationelle. En faisant cette analyse, nous avons constaté qu'un retard par rapport à la gestion des ressources humaines vis-à-vis de la croissance et du changement de structure organisationelle. La structure a évolué de l'entreprenariat à l'adhocratie mais le modèle de gestion des ressources humaines est resté arbitraire. De là, les besoins sont devenus très clairs : mettre en place une gestion par les compétences pour accompagner le changement vers un modèle plus individualisant. Pour ce faire, nous avons choisi un modèle doté d'une faible prescription à la fois générique, pour la gestion des talents, et spécifique, pour la gestion de l'expertise, indispensable pour garantir la qualité et la valeur ajoutée des services au sein du département de services.

\paragraph{} Finalement, nous n'avons pu qu'effleurer la mise au point d'un référentiel. Celui-ci est basé sur les processus et les tâches réalisées tout au long de l'année en l'absence de vision stratégique claire. Il se veut basé sur le principe suivant: la compétence ne peut s'exprimer qu'à travers des actions dans un contexte précis. Ce document ne va pas assez loin dans la définition du référentiel et des outils qui en découlent, mais il porte en lui l'impulsion qui va permettre de mettre en place une gestion des compétences et une gestion par les compétences au sein d'Odoo, en tout cas au PS. 



\paragraph{} Dans ce travail, nous avons tout de suite décidé de mettre en place une gestion des compétences pour faire face aux défis que constitue la croissance. Mais ne faudrait-il pas plutôt brûler cette gestion des compétences ? Le livre \citep[pp.253]{competencesbruler2006}, qui pose cette question, pose un regard critique mais favorable à celle-ci, en prévenant qu'il ne faut pas s'attendre à une nouveauté radicale mais qu'il s'agit plutôt d'un outil pour une adaptation permanante. Il met en garde contre le décalage qui pourrait se produire entre les discours et la pratique journalière. Cette pratique devra évoluer au fil du temps et la méthodologie prescrite au dernier chapitre n'est qu'une impulsion qu'il faudra concrétiser et adapter avec la pratique.   

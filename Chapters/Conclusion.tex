\chapter*{Conclusion Générale et discussions}
Ce document avait comme objectifs de déterminer quelle gestion des compétences Odoo se devait d'adopter et proposer une méthodologie de mise en place.
Mettre en place un modèle de gestion des compétences et le faire évoluer dépend normalement en grande partie de la capacité du sommet hiérarchique à traduire la stratégie de l’entreprise en opérations assurant qualité et profitabilité dans un contexte de marché à géométrie variable en tenant compte des spécificités de l’individu. Hélas, le contexte particulier d'Odoo fait que le moteur de changement ne se trouve pas au niveau du top management mais plus bas. Nous sommes donc parti de constats factuels 


Il est apparu très vite qu'avant de choisir une méthode, il fallait déterminé non seulement les objectifs attendu mais aussi la structure organisationelle. En faisant cette analyse, nous avons fait le constant qu'un certain retard avait été pris pour la gestion des ressources humaines vis-à-vis de la croissance et du changement de structure organisationelle. La structure a évolué de l'entreprenariat à l'adhocratie mais le modèle de gestion des ressources humaines est resté arbitraire. De là, les besoins sont devenus très clair : mettre en place une gestion par les compétences pour accompagner le changement vers un modèle plus individualisant. 






Le présent document ne focalise pas sur cette capacité, ni sur les spécificités individuelles. A partir des constats factuelles qui constitue la définition du problème, nous analysons les tangentes potentielles afin de retenir une méthodologie de base argumentée qui nous semble le plus appropriée.


Le travail accomplit ici pourrait facilement être répliqué pour le département de recherche et développement qui nécessite principalement des connaissance techniques pointues. Avec un travail plus poussé sur les compétences génériques, il pourrait aussi s'appliquer à l'équipe des consultants fonctionnels et l'ambition est bien là au sein d'Odoo pour faire ce travail. 

Il faut noté qu'il y a un biais important dans la défintion des objectifs et la finalité de l'outils. L'initiative vient des acteurs de terrain et ceux-ci ont leur objectifs propres. L'outils devra surement évoluer si celui-ci veut s'adapter au besoin des ressources humaines ou à la hiérarchie. 

Dans le troisième chapitre nous n'avons fait qu'effleuré la mise au point d'un référentiel.

Ce travail ne suffit pas à mettre en place une gestion des compétences au sein d'Odoo mais il porte en lui l'impulsion qui va le permettre.  

Porté un regard critique sur la gestion des compétences, Faut-il brûler la gestion des compétences. Pas de cadre normatif, il faudra tester empiriquement la méthodologie basée sur les besoins et l'amélioré en continue sur base du vécu sur le terrain.\citep[pp.252-253]{competencesbruler2006}

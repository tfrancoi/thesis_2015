\chapter*{Introduction}


Gestion des ressources humaines
===============================
Système de rémunération à la personne, il y a de grande disparité entre les salaires pour un même poste. D'ailleurs on à 3 grand postes: Consultant, Consultant Technique, Développeur. p238
Sentiment d'équité assez faible p239

Rémunération basée sur l'individu et ses compétences: Poste est trop restrictif p241
``La notion de compétence est ainsi apparue comme répondant à ces besoins'' p241

``Désormais, l'évolution de carrière d'un opérateur est liée à sa compétence et ne plus de l'existence de postes plus élevés ou de son accès à ces postes'' p 243
``Il revient à l'entreprise d'utiliser cette compétence en confiant au salarié les responsabilités correspondates.'' p243



%TODO
Besoin: Définition de la notion de compétence


compétence va vers une individualisation du rapport salarial, chez odoo c'est déjà le cas mais c'est vraiment arbitraire. Une gestion de compétence permettrait si pas de rajouter un peu d'objectivité au moins de pouvoir guidé les choix qui ont l'air arbitraire. 


Nul ne saurait se déclarer compétent lui-même. p249

p 257 comment décrire une compétence ? en décrivant sa situation de travail ? 

p 263 quelle type de compétence veut-on rémunérer
rémunère l'acquis ou à acquérir
Polyvalence ou expertise ? 

Chapitre sur l'évaluation

Evaluation perçue comme inutile, l'embarras de ceux qui la font passé, et de ceux qui la recoivent. p 369

p 383. Redéfinir les objectifs et les critères explicites de la démarche

=> But c'est que les évaluations servent à qqchose
=> Pouvoir définir un plan d'amélioration et en faire le suivit.


Différentes tâches qui nécessite des compétences plus avancées
-- Basique
- Module backend
- SA
- Module Web
- Support

-- Spécialisée
- Customization web 
- 


-- Humaine et bonne connaissance
-Donnée formation technique
-Formation technique avancées

- Responsable technique
- Review

- Pre sales : Démo et Estimation


Tout est lié => Stratégie de l'entreprise => Tâches à effectuée => Compétences nécessaire => Rémunération


1 Lister les compétences nécessaires
2 Définir les compétences 
3 Méthode d'évaluation des compétences
4 Formation pour augmenté les compétences
5 Prévisionelles quelle compétence vont être nécessaire dans le futur
4 Valorisation salariales en fonction des compétences et de leur évaluation

\chapter*{Introduction}


Gestion des ressources humaines
===============================
Système de rémunération à la personne, il y a de grande disparité entre les salaires pour un même poste. D'ailleurs on à 3 grand postes: Consultant, Consultant Technique, Développeur. p238
Sentiment d'équité assez faible p239

Rémunération basée sur l'individu et ses compétences: Poste est trop restrictif p241
``La notion de compétence est ainsi apparue comme répondant à ces besoins'' p241

``Désormais, l'évolution de carrière d'un opérateur est liée à sa compétence et ne plus de l'existence de postes plus élevés ou de son accès à ces postes'' p 243
``Il revient à l'entreprise d'utiliser cette compétence en confiant au salarié les responsabilités correspondates.'' p243



%TODO
Besoin: Définition de la notion de compétence


compétence va vers une individualisation du rapport salarial, chez odoo c'est déjà le cas mais c'est vraiment arbitraire. Une gestion de compétence permettrait si pas de rajouter un peu d'objectivité au moins de pouvoir guidé les choix qui ont l'air arbitraire. 


Nul ne saurait se déclarer compétent lui-même. p249

p 257 comment décrire une compétence ? en décrivant sa situation de travail ? 

p 263 quelle type de compétence veut-on rémunérer
rémunère l'acquis ou à acquérir
Polyvalence ou expertise ? 

Chapitre sur l'évaluation

Evaluation perçue comme inutile, l'embarras de ceux qui la font passé, et de ceux qui la recoivent. p 369

p 383. Redéfinir les objectifs et les critères explicites de la démarche

=> But c'est que les évaluations servent à qqchose
=> Pouvoir définir un plan d'amélioration et en faire le suivit.


Différentes tâches qui nécessite des compétences plus avancées
-- Basique
- Module backend
- SA
- Module Web
- Support

-- Spécialisée
- Customization web 
- 


-- Humaine et bonne connaissance
-Donnée formation technique
-Formation technique avancées

- Responsable technique
- Review

- Pre sales : Démo et Estimation


Tout est lié => Stratégie de l'entreprise => Tâches à effectuée => Compétences nécessaire => Rémunération


1 Lister les compétences nécessaires
2 Définir les compétences 
3 Méthode d'évaluation des compétences
4 Formation pour augmenté les compétences
5 Prévisionelles quelle compétence vont être nécessaire dans le futur
4 Valorisation salariales en fonction des compétences et de leur évaluation






\section{Gestion des ressources humaines: Le développement de la gestion des compétences}

- Définir la compétences
- S'en servir comme la base des décisions concernant le personnel: Evaluation, Formation, Gestion de carière, Affectation au emploi

Definition de compétence  P171
La compétence ne prend son sens quand dans l'action. On ne peut pas observer une compétence en soi, mais on peut observer son expression dans l'action. On ne peut parler d'une compétence que dans le cadre précis d'une situation de travail.  

Elle combine de façon dynamique les différents éléments qui la constituent (savoirs, savoir-faire, raisonements) 

Il faut identifié les situations de travail dans lesquels s'exprimes les compétences. 
Pour pouvoir mesurer la compétence, il faut analyser le travail. Découper celui-ci en activités secondaire qui mène au résultat.
Mais pour des activités complexe ce n'est pas si facile de découper celle-ci en sous-tâches, parfois l'expert lui-même n'est pas capable de décrire son raisonement.

Attention il ne faut pas oublé le contexte dans lequel la compétence s'exerce. 

\paragraph{Méthode de repérage et de codification des compétences}



=> Listes de compétencs en fonction des activités de l'emploi

L'élaboration d'un référentiel de compétences soulèvent des problèmes. 
=> Problème de définition des compétences 
=> Problème de l'usage du référentiel de compétence: besoin du formateur, évaluateur et recruteur est différent
=> Problème de jugement : La reconnaissance de la compétence n'est dû qu'au jugement public. On ne peut pas se déclarer compétence sois-même. 



\paragraph{Les compétences un outil}
- Pour rationaliser le travail de l'équipe
- Repenser la contribution du salarié et de sa performance: ce n'est plus implement le diplôme. 
- Conformer le comportement des employés à de nouvelle norme d'action 
- Permet de définir des normes de coopération et d'échange. 

\paragraph{C'est une réponse à de nombreux problèmes}
Mécontentement des nouveaux entrants en termes d'évolution professionnelle
Pallier à l'insufissance de polyvalence

=> La gestion des compétences serait paré de toutes les vertus et la solutions de toutes sortes de problèmes

\begin{itemize}
    \item Favorise une redistribution des tâches et des responsabilités
    \item Elle accroit la mobilité professionnelle
\end{itemize}



\section{définition du modèle de l'entreprise}
\begin{table}
    \caption{}
    \label{tab: page 185}

    \begin{center}
        \begin{tabular}{ll}
             Entrepreneuriale & Arbitraire\\
             Bureaucratique & Objectivant \\
             Adhocratique & Individualisant\\
             Professionnelle & Conventionnaliste \\
             Missionaire & Valoriel \\
        \end{tabular}
    \end{center}
\end{table}



La mise en place de la gestion des compétences devraient s'inscrire directement dans la démarch stratégique de l'entrerpise, alors que ca reste souvent déconnecté. 

Voir de la gestion des compétences à la gestion par les compétences. 


\section{Grille d'analyse de la gestion des compétences}

Page 192
\begin{enumerate}
    \item Quel est la stratégie de l'entreprise visée par le projet ?
    \item Quel problème l'entreprise essaye de régler par la gestion des compétences ?
    \item Lien entre compétence et stragégie 
    \item Comment la compétence est-elle définie
    \item Quel champ des ressources humaines: Recrutement, classification, rémunération, formaton, gestion des carières, etc.)
    \item Quels changement introduit la gestion des coméptences par rapport à l'existant
    \item Quels sont les outils utilisés
    \item Quels moyens sont-ils prévus  
\end{enumerate}
%TODO
% Demander à Johan les objectifs à atteindre à travers la gestion de compétence
% Demander à mon equipe de décrire les situations de travail et les compétences nécessaire. 

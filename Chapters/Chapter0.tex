\chapter*{Introduction}
L’augmentation continue de l’effectif au sein d’Odoo, éditeur de progiciels libres, qui vient de souffler ses 10 bougies seulement, constitue des défis importants. Une fois son software publié, son modèle économique repose sur la qualité, la vélocité et le degré d’industrialisation de ses services. Ainsi, ses ressources humaines et plus spécialement leurs compétences doivent assurer la pérennité et la croissance vers la première place du marché de l'ERP. Mais à l'heure actuelle, la gestion des ressources humaines n'a pas évolué à la hauteur des ambitions. 
Ce présent document focalise sur l’aspect cruciale qui est la détermination d'un modèle de gestion des compétences qui doit assurer que l’individu puisse 
-> se forger son identité d’acteur économique, 
-> assumer ses responsabilités dans des processus recurrent et des projets, 
-> assurer une satisfaction ‘client' aux dimensions multiple
->  augmenter sa contribution à l’entreprise et aux collègues en se spécialisant, et 
-> adhérer aux valeurs de l’entreprise et participer à la construction de l’environnement humain


Mettre en place un modèle de gestion des compétences dépend largement de la capacité du top management à traduire la stratégie de l’entreprise en modus operandi assurant qualité et profitabilité dans un contexte de marché à géométrie variable en tenant compte des spécificités de l’individu. Le présent document ne focalise pas sur cette capacité, ni sur les spécificités individuelles. A partir des constats factuelles qui constitue la définition du problème, nous analysons les tangentes potentielles afin de retenir une méthodologie de base argumentée qui nous semble le plus appropriée.


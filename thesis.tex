\documentclass[a4paper,10pt]{report}
\usepackage[utf8]{inputenc}
\usepackage[T1]{fontenc}
\usepackage{fancyhdr}
\usepackage[french]{babel}
\usepackage[authoryear, numbers]{natbib}
\usepackage[final]{pdfpages} 
\usepackage[unicode=true,hidelinks]{hyperref}
\usepackage[babel=true]{csquotes}
\lhead[\rm\thepage]{\fancyplain{}{\sl{\rightmark}}}
\rhead[\fancyplain{}{\sl{\leftmark}}]{\rm\thepage}





\begin{document}
\title  {Titre du papier}
\author  {Thibault François }
\maketitle
%\lhead{\emph{Table des matières}}  % Set the left side page header to "Contents"
\tableofcontents  % Write out the Table of Contents

\chapter*{Introduction}


Gestion des ressources humaines
===============================
Système de rémunération à la personne, il y a de grande disparité entre les salaires pour un même poste. D'ailleurs on à 3 grand postes: Consultant, Consultant Technique, Développeur. p238
Sentiment d'équité assez faible p239

Rémunération basée sur l'individu et ses compétences: Poste est trop restrictif p241
``La notion de compétence est ainsi apparue comme répondant à ces besoins'' p241

``Désormais, l'évolution de carrière d'un opérateur est liée à sa compétence et ne plus de l'existence de postes plus élevés ou de son accès à ces postes'' p 243
``Il revient à l'entreprise d'utiliser cette compétence en confiant au salarié les responsabilités correspondates.'' p243



%TODO
Besoin: Définition de la notion de compétence


compétence va vers une individualisation du rapport salarial, chez odoo c'est déjà le cas mais c'est vraiment arbitraire. Une gestion de compétence permettrait si pas de rajouter un peu d'objectivité au moins de pouvoir guidé les choix qui ont l'air arbitraire. 


Nul ne saurait se déclarer compétent lui-même. p249

p 257 comment décrire une compétence ? en décrivant sa situation de travail ? 

p 263 quelle type de compétence veut-on rémunérer
rémunère l'acquis ou à acquérir
Polyvalence ou expertise ? 

Chapitre sur l'évaluation

Evaluation perçue comme inutile, l'embarras de ceux qui la font passé, et de ceux qui la recoivent. p 369

p 383. Redéfinir les objectifs et les critères explicites de la démarche

=> But c'est que les évaluations servent à qqchose
=> Pouvoir définir un plan d'amélioration et en faire le suivit.


Différentes tâches qui nécessite des compétences plus avancées
-- Basique
- Module backend
- SA
- Module Web
- Support

-- Spécialisée
- Customization web 
- 


-- Humaine et bonne connaissance
-Donnée formation technique
-Formation technique avancées

- Responsable technique
- Review

- Pre sales : Démo et Estimation


Tout est lié => Stratégie de l'entreprise => Tâches à effectuée => Compétences nécessaire => Rémunération


1 Lister les compétences nécessaires
2 Définir les compétences 
3 Méthode d'évaluation des compétences
4 Formation pour augmenté les compétences
5 Prévisionelles quelle compétence vont être nécessaire dans le futur
4 Valorisation salariales en fonction des compétences et de leur évaluation






\section{Gestion des ressources humaines: Le développement de la gestion des compétences}

- Définir la compétences
- S'en servir comme la base des décisions concernant le personnel: Evaluation, Formation, Gestion de carière, Affectation au emploi

Definition de compétence  P171
La compétence ne prend son sens quand dans l'action. On ne peut pas observer une compétence en soi, mais on peut observer son expression dans l'action. On ne peut parler d'une compétence que dans le cadre précis d'une situation de travail.  

Elle combine de façon dynamique les différents éléments qui la constituent (savoirs, savoir-faire, raisonements) 

Il faut identifié les situations de travail dans lesquels s'exprimes les compétences. 
Pour pouvoir mesurer la compétence, il faut analyser le travail. Découper celui-ci en activités secondaire qui mène au résultat.
Mais pour des activités complexe ce n'est pas si facile de découper celle-ci en sous-tâches, parfois l'expert lui-même n'est pas capable de décrire son raisonement.

Attention il ne faut pas oublé le contexte dans lequel la compétence s'exerce. 

\paragraph{Méthode de repérage et de codification des compétences}



=> Listes de compétencs en fonction des activités de l'emploi

L'élaboration d'un référentiel de compétences soulèvent des problèmes. 
=> Problème de définition des compétences 
=> Problème de l'usage du référentiel de compétence: besoin du formateur, évaluateur et recruteur est différent
=> Problème de jugement : La reconnaissance de la compétence n'est dû qu'au jugement public. On ne peut pas se déclarer compétence sois-même. 



\paragraph{Les compétences un outil}
- Pour rationaliser le travail de l'équipe
- Repenser la contribution du salarié et de sa performance: ce n'est plus implement le diplôme. 
- Conformer le comportement des employés à de nouvelle norme d'action 
- Permet de définir des normes de coopération et d'échange. 

\paragraph{C'est une réponse à de nombreux problèmes}
Mécontentement des nouveaux entrants en termes d'évolution professionnelle
Pallier à l'insufissance de polyvalence

=> La gestion des compétences serait paré de toutes les vertus et la solutions de toutes sortes de problèmes

\begin{itemize}
    \item Favorise une redistribution des tâches et des responsabilités
    \item Elle accroit la mobilité professionnelle
\end{itemize}



\section{définition du modèle de l'entreprise}
\begin{table}
    \caption{}
    \label{tab: page 185}

    \begin{center}
        \begin{tabular}{ll}
             Entrepreneuriale & Arbitraire\\
             Bureaucratique & Objectivant \\
             Adhocratique & Individualisant\\
             Professionnelle & Conventionnaliste \\
             Missionaire & Valoriel \\
        \end{tabular}
    \end{center}
\end{table}



La mise en place de la gestion des compétences devraient s'inscrire directement dans la démarch stratégique de l'entrerpise, alors que ca reste souvent déconnecté. 

Voir de la gestion des compétences à la gestion par les compétences. 


\section{Grille d'analyse de la gestion des compétences}

Page 192
\begin{enumerate}
    \item Quel est la stratégie de l'entreprise visée par le projet ?
    \item Quel problème l'entreprise essaye de régler par la gestion des compétences ?
    \item Lien entre compétence et stragégie 
    \item Comment la compétence est-elle définie
    \item Quel champ des ressources humaines: Recrutement, classification, rémunération, formaton, gestion des carières, etc.)
    \item Quels changement introduit la gestion des coméptences par rapport à l'existant
    \item Quels sont les outils utilisés
    \item Quels moyens sont-ils prévus  
\end{enumerate}
%TODO
% Demander à Johan les objectifs à atteindre à travers la gestion de compétence
% Demander à mon equipe de décrire les situations de travail et les compétences nécessaire. 
 
\chapter{État des lieux et définition des objectifs}
\paragraph{}Odoo S.A, située en Brabant Wallon, est une société d’édition d'applications de gestion d'enterprises. Depuis sa fondation en 2005, elle jouit d'une croissance hors pair : elle est passée de 2 à 285 employés en dix ans. En cinq ans, le nombre d'employés en Belgique est passé de 18 à 125\footnote{Cette information est tirée du logiciel de gestion des ressource humaines interne consulté le 8 juillet}. Cette croissance est d'abord dûe à l’aspect innovant des logiciels édités: ils sont Open Source, sous licence AGPL\footnote{\enquote{GNU Affero General Public License (ou AGPL, signifiant licence publique générale Affero) est une licence libre dérivée de la licence publique générale GNU avec une partie supplémentaire couvrant les logiciels utilisés sur le réseau. Elle a été écrite par Affero pour autoriser les droits garantis par la GPL à couvrir les interactions avec des produits à travers un réseau comme Internet, ce que la GPL ne fait pas.}\cite{agpl}}, intégrés entre eux à l'image des ERP\footnote{Enterprise Resource Planning~\citep{wikierp}} et permettent, contrairement à la concurrence, une adaptation rapide et compétitive aux besoins des clients. Elle est également encouragée par une approche commerciale centrée sur les PME sans contrainte géographique. Ces atouts ont permis deux levées de fonds, en 2010 et en 2014, desquelles ont découlé deux augmentations de personnel (de 18 à 34 employés, puis de 64 à 111 employés). 

\paragraph{}Les applications ne comportant pas de couts de licence, Odoo génère ses revenus des services d’implémentation des applications auprès des clients, ainsi que des services de maintenance, de support et de migrations vers des versions supérieures. Les applications sont rendues disponibles sur une platforme mutualisée ou SAAS\footnote{Software as a Service}. Odoo délivre aussi du support et de la formation à un réseau de partenaires qui s'occupe du marché des grandes entreprises. 


\paragraph{}Les produits et activités cités ci-dessus sont assurés par trois départements principaux: Le développement du logiciel et de la platforme mutualisée par le département de recherche et développpement\footnote{Par la suite, il sera abrégé par R\&D}; le département de vente et marketing qui assure la stratégie commerciale et la vente de services; et finalement le département de services ou PS\footnote{Abréviation de Professional Services} qui assure les services aux clients (services avant-vente, services d’analyse des besoins et de faisabilité, de mise en oeuvre à travers des projets, de maintenance, de support). 

\paragraph{}Le PS comprend deux équipes. D'une part, il y a l'équipe fonctionnelle, composée de consultants fonctionnels pour la plupart ingénieurs de gestion et d'un responsable, qui s'occupe des missions d'analyse, d'implémentation, de gestion de projet et de support. D'autre part, il y a l'équipe technique, composée de consultants techniques et d'un responsable d'équipe,tous avec une solide formation en informatique, qui s'occupe de l'analyse de faisabilité, de l'implémentation des adaptations du logiciel aux besoins des clients, du support technique et de la formation. Depuis 2010, le département de services est passé de 5 personnes à 36 personnes. La croissance a été chaotique avec de nombreux départs; les deux équipes ont été crées en cours de route; la séparation entre poste fonctionel et technique également; et finalement la délégation par le directeur du département d'une partie de ses responsabilités à deux "team leaders" date de moins d'un an. Mais en dehors de cela, rien n'a été fait pour accompagner la croissance de ce département. En tant que responsable d’équipe technique au PS , j’ai le privilège de pouvoir accéder à une partie des informations permettant de faire des constats ainsi que de formuler des solutions et de les mettre en oeuvre. Pour cette raison, le présent travail se focalisera principalement sur ce département.


\paragraph*{}Une croissance aussi rapide engendre de nombreux problèmes et défits dont l'intégration des nouveaux employés dans la culture de l'entreprise et leur formation. Il faut veiller à garder une qualité de service constante malgré les disparités grandissantes entre les individus, offrir à ceux-ci des possibilités d'évolution dans l'entreprise malgré le nombre de postes très restreint dont elle dispose et finalement envisager une meilleure organisation du travail (actuellement basée principalement sur la disponibilité), afin d'assurer un résultat optimal et le développement de chacun. Pour répondre à ces nombreux défits nous avons choisi de nous aider de la gestion des compétences.

\section{Objectifs visés par la gestion des compétences}
D'après le livre de Guérin~\citep{gestionressourceshumaine2007} et l'article de Delobbe~\citep{delobbe}, la gestion des compétences a plusieurs objectifs dont voici une liste non exhaustive.
\begin{itemize}
    \item \enquote{Elle a pour effet d'assurer l'adhésion des salariés à des normes de comportements considérées comme acceptables et valorisées au sein d'une organisation}~\citep[p.40]{delobbe}
    \item Elle permet d'adapter les comportements des salariés aux attentes des clients dans une logique de service et de qualité. ~\citep[182]{gestionressourceshumaine2007}
    \item \enquote{Gérer la polyvalence et la flexibilité... Elle vise à dépasser une allocation rigide des effectifs à des postes de travail délimités}~\citep[p.41]{delobbe}
    \item Elle favorise une redistribution des tâches et des responsabilités en cas de réduction d'effectifs ~\citep[182]{gestionressourceshumaine2007}
    \item  \enquote{L'objectif de la gestion des compétences est donc d'identifier, de sélectionner, de promouvoir et de récompenser les plus talentueux..}~\citep[p.43]{delobbe}
    \item  \enquote{La gestion des compétences a ici pour objectif premier d'affirmer et de développer l'expertise technique interne indispensable à la réalisation des missions de l'entreprise.} ~\citep[p.45]{delobbe}

\end{itemize}

\paragraph{}La gestion des compétences peut être utilisée à tous les niveaux des ressources humaines: les évaluations et la rémunération du personnel; la gestion des performances; la classification des emplois; l'allocation des effectifs; la sélection des candidats; la formation des employés et la gestion des carrières. ~\citep[p.32]{delobbe}. Normalement, la gestion des compétences devrait s'inscrire dans une démarche stratégique. Malheureusement, cette mise en place de la gestion des compétences ne venant pas du sommet hiérarchique, il ne sera pas possible de l'inscrire dans la stratégie de l'entreprise. Un autre problème est le manque de stratégie de la société à moyen et long terme.  La gestion des compétences s'inscrira donc dans la limite du département de services et commencera avec l'équipe des consultants techniques. 

\paragraph{}Avant de se lancer tête baissée dans la définition de la gestion des compétences qui va être mise en place chez Odoo, arrêtons-nous un instant. La conclusion de l'article~\citep{delobbe} suggère un lien entre le modèle de gestion des compétences à appliquer et le type de structure organisationelle de l'entreprise qui la met en place. Attardons-nous donc un instant sur la structure organisationnelle d'Odoo, sur son évolution et sur l'évolution du modèle de gestion des ressources humaines. Une fois cet état des lieux fait, nous pourrons déterminer quelle gestion des compétences est la plus appropriée.
        



\section{Structure organisationnelle et politique de gestion des ressources humaines au sein d'Odoo}
\subsection{La configuration en 2010}
Il est intérressant de revenir à la configuration d'Odoo en 2010, juste après la première levée de fond. Comme on peut le voir dans le tableau \ref{nb_employe} en annexe, au début de l'année 2010, il y avait 18 employés et à la fin de l'année, il y en avait déjà 34. Ce nombre reste faible comparé au 125 travailleurs employés en belgique actuellement. À cette époque, le sommet hierarchique était composé d'un CEO-fondateur, d'un CSO, d'un COO et d'un CTO\footnote{Respectivement Chief Executive Officer, Chief Sales Officer, Chief Operating Officer, Chief Technical Officer}. Il y avait déjà trois départements, présents sur le même site: le département de recherche et développement géré par le CTO; le département de vente géré par le CSO; et le département de services, en théorie géré par le COO. En dehors du sommet hiérchique, il n'y avait pas de responsable d'équipe. Les employés sont depuis le début très qualifiés: des ingénieurs en informatique (en R\&D et au département de services), et des ingénieurs des gestions (au département de services et à la vente). Au sein de chaque département, le travail était intercheangable entre chaque membre d'un département. En R\&D et au PS\footnote{Abréviation pour les départements de service: Professional Services}, chacun travaillait sur son projet et il était difficile de changer l'assignation en cours de route, mais toute nouvelle tâche était suceptible d'être assignée à quiconque. Le CEO passait presque quotidiennement voir qui faisait quoi, pour faire des ajustements ou débloquer une situation. Toutes les décisions étaient prises par le comité de direction composé des quatres membres exécutifs mais bien entendu le CEO avait toujours le dernier mot. C'est à cette époque que le beau-frère de celui-ci fut envoyé aux États-Unis pour ouvrir un nouveau bureau. Au sein du PS, l'organisation du travail se faisait de manière très simple: on choisissait la personne la plus compétente parmis les personnes disponibles, et sachant que personne n'était jamais vraiment disponible, on choisissait simplement la plus disponible. Le paragraphe suivant résume la situation au niveau de la gestion ressources humaines chez Odoo en 2010:

\begin{description}
    \item[Planification] Il nous manque des informations pour pouvoir évaluer la planification au niveau RH. Il semble que dans un contexte de croissance, le recrutement était ouvert pour les trois départements.
    \item[Sélection] La sélection se faisait via une interview avec le directeur concerné ou alors directement avec le CEO. Dans un premier temps, les interviews étaient informelles. Pour les postes techniques, des exercices de programmation ont été mis en place à la fin de l'année 2010.
    \item[Formation] Tous les employés recevaient une semaine de formation à leur arrivée. Une semaine supplémentaire était octroyée aux profils techniques. Ces deux semaines de formation étaient données car elles étaient vendues et prestées pour nos partenaires. Les formations n'étaient pas spécifiques en fonction des différents postes. Il fallait se former sur le tas. De manière générale, la méthode de formation consistait à jeter les nouveaux employés tout habillés dans la piscine.
    \item[Evaluation] La notion d'évaluation n'était pas du tout formalisée. Elle se faisait principalement à la demande de l'employé. À cette époque, les seules personnes congédiées étaient celles du département de vente lorsqu'elles avaient de mauvais chiffres, et elles ne prestaient généralement pas leur préavis. Les personnes qui démissionnaient ne prestaient également que très rarement leur préavis. La démission étant parfois perçue comme une trahison de la part de la direction.
    \item[Rémunération] La rémunération dépendait de l'humeur du CEO lors de l'entretien de sélection. Les augmentations se faisaient ensuite essentiellement à la demande de l'employé: il n'y avait donc aucune règle en la matière. 
    \item[Promotion] Il n'existait, par ailleurs, aucune possibilité de promotion: seule la croissance donnait l'espoir d'en avoir une un jour. Cela n'empêchait par contre en rien de voir sa rémunération augmentée. 
\end{description}

 

\paragraph{}Si l'on met en parallèle les particularités des ressources humaines d'Odoo en 2010 et la théorie de Pichault et Nizet ~\citep[pp. 48-49]{pichault}, on peut tirer comme conclusion que la configuration était principalement entrepreneuriale. L'autorité du fondeur était élevée. Il y avait une grande division du travail au niveau vertical mais faible au niveau horizontal, en tout cas au sein d'un même département. La coordinatation du travail se faisait via les supervisions directes du CEO et du directeur. On pouvait aussi noter l'implication des membres de la famille. Cependant, on pouvait déjà percevoir des signes d'une configuration adhocratique: employé très qualifié, organisation principalement par projet en R\&D et au PS. 

\paragraph{}Par contre, le modèle de GRH\footnote{Gestion des ressources humaines} était clairement arbitraire~\citep[pp. 115-119]{pichault}, preuve en est le licenciement instantané des employés à congédier. Par ailleurs, le petit nombre d'employé favorisait "l'esprit maison"" avec l'organisaton de nombreux verres entre collègues. En R\&D, il n'était pas rare de se réveiller chez un collègue les lendemains de veille avec le CTO dans le canapé d'à coté. La sélection et les évaluations se faisaient intuitivement.  La formation se faisait sur le tas et les promotions étaient presque inexistantes. 

\subsection{La configuration à l'heure actuelle}
\paragraph{} Ce modèle de GRH arbitraire fonctionnait assez bien avec la configuration entrepreneuriale d'une petite société de 20 employés. La configuration a cependant bien évolué ces cinq dernières années, durant lesquelles l'entreprise est passée de 18 employés à 125 employés. 

\paragraph{} Comme évoqué antérieurement, les départements se sont progressivement structurés en équipes. Le COO a été remplacé par le directeur du PS. Deux départements, marketing et financier, se sont greffés à l'ensemble. L'autonomie de chaque équipe a grandi, avec, parfois, un contournement de la ligne hiérarchique de la part du CEO. Par contre, les décisions stratégiques restent toujours entre les mains du comité de direction, emputé de son COO, mais comprenant deux nouveaux membres: le responsable marketing et le responsable financier. Les communications entre les équipes se sont structurées via un système de tickets. Au département de services, il y a deux types de profils: fonctionnel et technique. Les équipes constituées des deux profils se forment et se déforment au fil des projets. Elles sont assez autonomes. Des tensions sont apparues entre le département de ventes et le département de service, le premier ayant des objectifs de chiffre d'affaire et le second des objectifs de qualité de service et de rentabilité.

\paragraph{} Avec la croissance du nombre de clients, le support des utilisateurs a pris une place stratégique au sein du PS, mais il n'est pas géré par une équipe en particulier. Les employés s'en occupent à tour de rôle. Cela présente deux avantages. Premièrement, le support étant perçu comme une tâche ingrate, on évite la frustration que pourraient ressentir les personnes si elles y étaient allouées sans interruption. Deuxièmement, le support touche à tous les aspects opérationnel d'Odoo: il a donc un grand pouvoir formateur dont tout le monde doit bénéficier. Cependant, cette configuration pose problème au niveau de la standardisation de la qualité et des procédures. Des processus plus standardisés apparaissent par ailleurs au niveau de la vente avec l'appui du logiciel Odoo. 

\paragraph{} Au niveau de l'organisation du travail au sein du PS, les choses n'ont pas beaucoup changé, on choisit toujours la personne la plus compétente parmi les personnes disponibles, c'est-à-dire, celle qui est un peu moins occupé que les autres et la notion de personne compétente reste très floue. 

\paragraph{} Nous pouvons observer que la configuration entrepreneuriale a cédé la place à une configuration adhocratique~\citep[pp. 53-54]{pichault} avec une forte décentralisation du pouvoir pour les questions opérationnelles, mais toujours une forte centralisation pour les décisions stratégiques. Il y a aussi une petite tendance à la bureaucracie pour les tâches plus répétitives comme le support ou la création de contrats. 

\paragraph{}Il y a cependant encore de nombreux problèmes au niveau de l'organisation du travail au sein du PS. Il y a, d'abord, celui de la qualité: une activité telle que le support nécessite de fournir un niveau relativement constant de qualité, qui n'est actuellement pas garanti. Ensuite, la spécialisation pose problème. Lorsqu'un nouveau type de tâche se présente, la personne qui est disponible s'en charge. Par la suite, si d'autres tâches de la même nature surviennent, la tendance est de l'assigner toujours à la même personne car elle connait déjà le problème. Ce comportement génère une situation problèmatique car il n'existe pas de backup si la personne en charge tombe malade ou démissionne. 
Finalement, l'assignation des tâches est principalement basée sur la disponibilité des ressources. Cela pose un problème au niveau de la formation de l'équipe. Les compétences de chacun sont développées au hasard, sans tenir compte des besoins ou de la volonté de l'employé. Il n'y a pas de politique cohérente de développement. Faisons maintenant le point sur l'évolution de la politique des ressources humaines depuis 2010. 

\begin{description}
  \item[Planification] La planification des ressources se fait toujours département par département. À l'heure actuelle, un gel total du recrutement est opéré sans tenir compte des besoins des départements, dans le but d'atteindre un seuil de rentabilité. Les départs sont maintenant mieux gérés, de façon à ce que l'employé qui part puisse confier ses responsabilités à un collègue.
  \item[Sélection] Le processus de recrutement vise l’acquisition de profils génériques : des individus possédant des savoir académiques appropriés, une capacité intellectuelle d’apprentissage continue ainsi qu'un potentiel d’autonomie élevé. Les tests adaptés pour chaque famille de postes sont évalués, avant l'entretien, avec le responsable du service. Les compétences spécifiques ne sont pas recherchées: elles s'acquerront avec l'expérience. 
  
  \item[Formation] Concernant la formation, rien n'a bougé. Après les deux semaines de formation, chaque équipe coache ses nouveaux venus. Ce modèle a été maintenu car le sommet hiérarchique est convaincu que la formation externe contient très peu de valeur et que l’exposition immédiate et fréquente aux problématiques des clients est un accélérateur pour l’obtention et le renforcement des compétences. Un système d’accompagnement personnalisé a donc été mis en place pour chaque nouvel employé au sein de l'équipe fonctionnelle. 
  \item[Evaluation] L'évaluation reste est un des points noirs de la gestion de ressources humaines. Elle consiste simplement en un entretien d'évaluation. Celui-ci est l'occasion pour le responsable d'avoir une discussion sur l'année écoulée avec chaque membre de son équipe, mais c'est surtout l'occasion pour l'employé de demander une augmentation si l'évaluation s'est bien passée. Rien ne permet objectivement de savoir si l'évaluation est bonne ou mauvaise. 
  
\paragraph{}Le formulaire d'évaluation actuel (Annexe B) comporte plusieurs subdivisions. La première et la dernière abordent les objectifs de l'année écoulée et de celle à venir. La seconde partie contient une série de points à coter. Il s'agit d'un mélange de compétences, de savoirs, de savoir-faire qui seraient pertinants pour mesurer la performance. Cependant, le glossaire\footnote{A la première page du formulaie on peut lire \enquote{Each item is defined in the glossary (page 7).}. Le formulaire ne compte que 6 pages} n'est jamais communiqué, ni à l'évaluateur, ni à l'évalué. Les différents points doivent être notés, mais l'échelle de référence à utiliser dans la notation n'est pas définie. La troisième subdivision traite de la conformité du comportement de la personne aux valeurs de l'entreprise, qui sont au nombre de huit. Elle comporte également une notation de l'appréciation globale. Ici encore, l'évaluation doit se faire selon une échelle non définie. Finalement, deux questions permettent d'aborder avec l'employé ses aspirations professionnelles. Ce formulaire rassemble de nombreux aspects de la gestion des ressources humaines, mais sans leur donner de cadre. Il a été conçu de manière très générique pour convenir à tous les départements sans qu'aucun ne puisse s'y reconnaître.

\paragraph{}Dans le formulaire, des objectifs globaux sont définis. Malheureusement, à l'heure actuelle, dans le département de service, il n'y a aucune politique cohérente pour définir des objectifs personnels. D'ailleurs, les objectifs qui sont définis ne sont que très rarement suivis. Dans la pratique, le reste du questionnaire n'est souvent pas abordé, sauf si l'employé l'a rempli et qu'il souhaite discuter de points précis. La faute en incombe au manque de définition des éléments à évaluer et à l'absence de méthode et de références pour évaluer ces éléments. Tout cela donne la sensation d'une politique d'évaluation et de rémunération complètement arbitraire. Il y a également un manque important d'information et formation pour les personnes qui évaluent.
  \item[Rémunération] Le système est basé sur les précédents, il n'y toujours pas de politique de rémunération. Les salaires des anciens et leurs évolutions servent comme base comparative pour les salaires des nouveaux venus et leurs évolutions. Seule la liste des salaires actuels est maintenue.
  
  \item[Promotion] Il y a eu très peu de promotions, et la croissance de la société reste toujours le meilleur espoir pour en obtenir une. Ellles ne sont toutefois pas automatiques. La R\&D est restée très longtemps avec un seul responsable, le CTO, malgré ses 40 membres. Ce n'est que très récemment qu'elle en a eu davantage.  
\end{description}

\paragraph{} On retrouve toujours beaucoup de caractéristiques du modèle arbitraire, tout en rencontrant déjà quelques éléments du modèle individualisant. Chez les commerciaux, la rémunération est variable. La sélection se fait sur base de compétences vérifiées par des tests lors de l'embauche. L'évaluation détermine des objectifs qui devraient être suivis et évalués à la fin de la période.


\paragraph{} La gestion des compétences chez Odoo doit donc permettre d'achever le passage d'un modèle arbitraire, qui ne convient plus à la taille de l'entreprise, à un modèle individualisant, plus adapté à la nouvelle structure adhocratique. Les aspects les plus importants que cette gestion des compétences devra permettre sont la planification, la formation et l'évaluation. L'objectif est de pouvoir planifier de manière plus structurée les compétences nécessaires au sein de chaque équipe. Pour pouvoir planifier les besoins, nous devrons faire un état des lieux de ce qui existe et ensuite décider comment acquérir les compétences manquantes via la formation. Les évaluations devront permettre d'établir l'état des lieux des compétences existantes mais aussi de pousser les employés à acquérir celles manquant au sein de leur équipe. Une fois cette gestion mise en place, il sera plus facile d'objectiver une rémunération basée sur les compétences de chacun. Il sera également plus facile d'envisager une mobilité horizontale et verticale.

\paragraph{} Maintenant que les attentes par rapport à la mise en place d'une gestion des compétences ont été définies, chercons le modèle qui convient le mieux à la situation et aux objectifs du PS au sein d'Odoo.





 
\chapter{Vers quelle gestion des compétences}
Essayons de trouver quel type de gestion des compétences serait le plus adapté pour atteindre les objectifs définis, puis définissons une méthodologie et un plan d'action pour sa mise en place. Mais, avant tout, il faut s'accordons-nous sur la définition de "compétence". 

\section{Définition générale de la compétence}
Regardons ce que la littérature dit à propos de la notion de compétence. Commençons par ce qui en est dit dans le livre de Guérin, Cadin, Pigeyre\citep[pp.170-171]{gestionressourceshumaine2007}
\begin{quotation}
\textit{ On observe une grande variété dans les définitions adoptées, mais toutes retiennent, d'une manière ou d'une autre, les mêmes éléments essentiels:
\begin{itemize}
 \item La compétence prend sens par rapport à l'action: on ne peut parler de compétence que dans le cadre précis d'une situation de travail.
 \item Elle combine de façon dynamique les différents éléments qui la constituent (savoirs, savoir-faire pratiques, raisonnements) pour répondre à des exigences d'adaptation.
\end{itemize}}
\end{quotation}

Plus loin, nous retrouvons encore ce lien entre compétence et action.\citep[pp.172]{gestionressourceshumaine2007}
\begin{quotation}
 \textit{"La compétence est une notion abstraite et hypothétique."}\footnote{Leplat J., ``Compétence et ergonomie'', Modèle en analyse du travail, Bruxelles, Mardaga, 1991, pp. 263-278}
 \textit{On ne peut en observer que les manifestations. [...], c'est à partir de la situation de travail et de la manière dont celle-ci est assumée qu'il est possible d'inférer la compétence.}
\end{quotation}

Et finalement, n'oublions pas le caractère socialement construit de la compétence. \citep[pp.249]{gestionressourceshumaine2007}
\begin{quotation}
\textit{[...] c'est le fait de reconnaître une compétence qui la fait exister. Autrement dit, la compétence n'existe pas sans le jugement d'autrui: nul ne saurait se déclarer compétent lui-même.}
\end{quotation}

Nous observons un point de vue assez similaire dans Aubret, Gilbert\citep[pp.7]{evalcompetence} où là encore il ne s'accorde pas sur une définition unique de la compétence:

\begin{quotation}
\textit{La notion de compétence se présente souvent comme une notion insaisissable au regard de la diversité des usages. [...]
Le terme de compétence fait partie de ces mots à multiples facettes que personne n'a véritablement le pouvoir de réduire à une seule [définition] non équivoque et de l'imposer à tous.
Aussi, nous voyons apparaître dans la littérature sur les compétences de nombreuses définitions qui prennent ce mot, soit comme un terme, soit comme une notion, un concept ou un construit social. [...] R. Zemke (1995), [...], en arrive à la conclusion que la compétence, les compétences, les modèles de compétences et la formation axée sur la compétence sont des mots valises qui signifient seulement ce que l'auteur veut leur faire dire.}

\textit{Ce que disent les chercheurs et praticiens:}
\textit{\begin{itemize}
 \item Compétence: c'est la capacité à résoudre un problème dans un contexte donné (Michel \& Ledru, 1991);
 \item Les compétences sont des ensembles de connaissances, de capacités d'actions et de comportements structurés en fonction d'un but et dans un type de situations données (Gilbert \& Parlier, 1992);
 \item [...]
 \item La compétence est la prise d'initiative et de responsabilité de l'individu sur des situations professionnelles auxquelles il est confronté (Zarifian, 1999)
\end{itemize}}

\end{quotation}\citep[pp.7-8]{evalcompetence}

Finalement l'article de Delobbe, Gilbert, Le Boudelaire\citep[pp.30]{delobbe} résume assez bien la complexité de la situation.
\begin{quotation}
\textit{Les définitions de la compétence ont été marquées par des divergences idéologiques qui se sont traduites dans la façon concrète de formuler les référentiels. Entre le savoir-faire opérationnel validé de l'accord ACAP 2000 et la compétence définie comme l'intelligence des situations par Bortef, entre les approches ergonomiques et sociologique francophones dans lesquelles la compétence est directement ancrée dans les activités des opérateurs et l'approche psychologique surtout nord-américaine, il y a des nuances importantes. }\citep[pp.30]{delobbe}
\end{quotation}


\paragraph{}Il ressort que la compétence n'a pas de définition unique: de nombreux chercheurs s'accordent pour dire que la compétence ne peut s'exprimer que dans l'action. Sans action, il n'y a pas de compétence. La compétence est le plus souvent un mélange de savoirs, savoir-faire et de capacités à raisoner. Notre définition dépendra donc du contexte dans lequel nous voulons l'utiliser. L'article Delobbe\citep[pp.31]{delobbe} lie celle-ci aux caractéristiques de l'organisation qui va l'employer. Nous avons conclu au chapitre précédent qu'Odoo tend vers une structure adhocratique et que la transition vers un modèle individualisant doit s'opérer. Voyons donc quel modèle de gestion de compétence convient le mieux à cette situation.


\section{Recherche des modèles appropriés}
Quatre modèles sont définis dans l'article Dellobe\citep[pp.39-49]{delobbe}: Le modèle de normalisation, le modèle de polyvalence, le modèle du talent individuel et le modèle de l'expertise. Nous allons maintenant mettre en perspective chaque modèle avec les besoins que nous avons définit plus haut pour déterminer lequel apporterait le plus à Odoo.
\begin{description}
  \item[Le modèle de normalisation]
  Ce modèle favorise l'application de normes qui sont acceptées et valorisées dans l'entreprise. Ce modèle est utile dans le cadre d'une société à croissance rapide ou à la suite de fusion-acquisition. Nous ne cherchons pas du tout à faire appliquer des normes au sein d'Odoo. Ce modèle ne semble donc pas d'application. 
  \item[Le modèle de la polyvalence]
  Un des élements recherché par la gestion des compétences est une meilleur organisation du travail au sein d'une même famille de postes. Hélas, ce modèle ne peut pas convenir. Il est surtout utile pour des métiers peu qualifiés et pour permettre une évolution d'une organisation rigide du travail vers une plus de flexibe. Ici ce n'est pas le cas, nous avons affaire à des travailleurs hautement qualifiés et la flexibilité est déjà présente. Ce modèle n'est pas optimal pour permettre un bon développement de l'équipe. 
  \item[Le modèle du talent individuel]
  Ce modèle convient bien à une structure adhocratique. \textit{"En termes de stratégie, cette approche convient aux entreprises [...] dans lesquel[le]s l'adaptabilité, la réactivité à saisir les opportunités, la capacité à gérer des situations neuves et complexes et à proposer des solutions innovantes sont décisives"}\citep[pp.44]{delobbe} Cette situation décrit bien une certaine réalité du travail de consultant technique. Les équipes sont constituées en fonction des besoins du projet et des compétences de chacun. Ce modèle définit des compétences décontextualisées qui sont soit des aptitudes intellectuelles soit des aptitudes humaines. Le référentiel ne définit pas les tâches à réaliser ni la manière de les réaliser, ce qui correspond assez bien au contexte de travail chez Odoo. Il y a malgré tout un bémol à ce modèle: il définit le plus souvent des compétences très génériques et décontextualisées. Or, ici, nous avons affaire à une expertise qui est nécessaire pour mener à bien les tâches du poste de consultant. Cette expertise ne peut ni être générique, ni être décontextualisée. 
  \item[Le modèle de l'expertise]
  \textit{"La gestion des compétences a ici pour objectif premier d'affirmer et de développer l'expertise technique interne indispensable à la réalisation des missions de l'entreprise. [...] Les organisations qui visent la prestation d'un service à haute valeur ajoutée [...] basent leur avantage concurrentiel sur la détention de compétences et de connaissances internes pointues, rares, et difficilement imitables"}\citep[pp.45]{delobbe} En tant qu'éditeur, les services de Odoo S.A. se doivent d'être à la pointe en ce qui concerne le logiciel Odoo. La gestion des compétences et des connaissances doit donc permettre de délivrer ce service de pointe et de le maintenir à niveau tout au long de l'évolution du logiciel. 
\end{description}


\paragraph{}Le modèle qui conviendrait donc au département de service d'Odoo serait un mélange du \textit{modèle du talent individuel} et du \textit{modèle de l'expertise}. La table \ref{model_comp} reprend les caractéristiques des deux modèles retenus. Comme on peut le voir, ces deux modèles possèdent des caractéristiques opposées, l'un se veut générique et décontextualisé, l'autre est très spécifique, adapté au contexte de chaque famille de métier. 

\paragraph{}Le modèle d'Odoo devra intégrer ces deux dimensions quelque peu contradictoires.
La performance du département de service passera bien entendu par le développement des compétences techniques et fonctionnelles, qui doivent sans cesse être renouvelées, et par celui des compétences humaines, nécessaires à la bonne marche des équipes de projet, à la relation client, ainsi qu'à la gestion des équipes grandissantes.



\begin{table}[H]
  \caption{Résumé des caractéristiques des référentiels de compétences selon leur modèle}
  \label{model_comp}

  \begin{center}
    \begin{tabular}{p{0.25\textwidth}p{0.34\textwidth}p{0.33\textwidth}}
      & \textbf{Talent individuel} & \textbf{Expertise} \\
      \hline
      \textbf{Formulation de la compétence} & Aptitutes intellectuelles et comportementales. Laisse une large marge de manoeuvre & Savoir et Savoir-faire techniques et fonctionnels. Des savoir-agir complexes\\
      \textbf{Maille du référentiel} & Assez large, générique et décontextualisé & Etroite, Spécifique\\
      \textbf{Acteurs impliqués}  & Département RH, Cabinet externe &  Les équipes concernées\\
  
    \end{tabular}
  \end{center}
\end{table}

\section{Définition du modèle de gestion de compétence chez Odoo}
Dans les sections précédentes, nous avons déterminé quels étaient les objectifs de la gestion des compétences et quels modèles de gestion allaient nous permettre d'atteindre ces objectifs. Nous allons maintenant synthétiser ces éléments et décrire les caractéristiques du modèle "idéal" de gestion de compétences pour Odoo dans le contexte actuel.\footnote{Compte tenu de la rapide croissance et du changement continu de stratégie, il est impossible de prédire la durée de la validité de cette analyse.} 

\subsection{Définition de la compétence adapté à Odoo}
Nous avons vu précédement que deux modèles assez différents étaient utiles pour atteindre les objectifs fixés. La compétence sera un savoir-agir complexe basée sur des savoirs et savoir-faire techniques et fonctionnels mais aussi des aptitudes comportementales et intellectuelles. 

\subsection{Porté du référentiel ou maille}
Pour le premier groupe de compétence, les savoir-agir complexes, la maille sera assez petite. Cette partie du référentiel sera différente pour chaque famille de rôle: consultant technique, consultant fonctionnel, développeur R\&D. Ce sont les postes actuels. Une définition des compétences techniques et fonctionnelles permettra une meilleur granularité au sein de chaque poste pour définir des profils. Dans le cadre de ce travail, on ne détaillera que les compétences nécessaires au rôle de consultant technique. Il faudra définir un référentiel différent pour chaque famille de postes, même si certains pourraient se recouper. Cette partie devra rendre la complexité de chaque poste. 

\paragraph{}Le problème avec une maille serrée est le nombre de référentiels à mettre à jour au fur et à mesure.
 Toutefois, dans le cas d'Odoo, avec 3 référentiels qui auraient beaucoup en commun, on couvre déjà 76 employés sur les 125. Bien entendu, à l'intérieur de chaque référentiel, il pourra y avoir de nombreux profils mais ceux-ci ne complexifieront pas beaucoup la maintenance. La seconde partie du référentiel sera par contre très large et décontextualisée. Il pourra être commun à tous les rôles et ne nécessitera que peu de maintenance. 

\subsection{Les acteurs impliqués dans la construction}
Pour la partie spécifique à chaque rôle, il faudra impliquer les membres des opérations directement. Ceux-ci seront les plus à même de décrire leur travail, les savoirs et les savoir-faire. Cette situation est pratique vu que l'initiative d'une meilleure gestion des compétences vient de la partie opérationnelle. Pour la partie générique, les choses se compliquent. Généralement, les acteurs impliqués ne sont pas conscients des aptitutes comportementales et intellectuelles qu'ils utilisent et qui sont nécessaires pour effectuer au mieux leur travail. Le référentiel ne pourra pas venir d'eux. Il vient le plus souvent de spécialistes et du département de ressources humaines mais il ne sera pas facile d'impliquer celui-ci. Cependant, il faut remarquer que cette partie est déjà présente dans le formulaire d'évaluation mis au point par le département de ressources humaines. Si l'on met de coté l'absence de référentiel s'y rapportant, il s'agit déjà un point positif.



\subsection{Les objectifs et les usages du référentiel}
Le référentiel aura plusieurs usages. Il permettra d'abord de formaliser la planification du besoin en compétences et d'aider à une meilleur répartition du travail, afin de favoriser le développement de chacun en fonction des besoins. Ensuite, il sera utilisé lors des évaluations qui serviront à faire l'état des lieux des compétences de chacun ainsi qu'à planifier leur développement au sein des équipes du PS. Finalement, cet état des lieux permettra d'objectiver une rémunération individualisée qui, pour le moment, semble arbitraire.



\section{Conclusion}
Dans le premier chapitre, nous avions mis en exergue les problèmes auxquels le département de services au sein d'Odoo faisait face ainsi que les objectifs à atteindre. Dans ce chapitre, nous avons défini les caractéristiques du modèle et du référentiel. Il nous reste à déterminer comment construire ce référentiel et ce qu'il va contenir.
 










 


\chapter{Élaboration d'un référentiel de compétence}
L'élaboration d'un référentiel de compétence amène à prendre toutes une série de décisions. Cette élaboration peut être vu comme un processus qui va figé dans le marbre la définition et l'organisation du travail mais il n'en est rien, un référentiel est un outil qui évolue dans le temps et qui est conçu pour apréhender le futur. "S'il s'ancre dans le travail d'aujourd'hui, il vise essentiellement le travail de demain."\citep[pp.19]{refcompetence} Sachant cela nous ne resterons pas pétrifié par l'ampleur de la tâche et le nombre de mauvaises direction qu'il est possible de prendre. Après tout si il faut construire un réfétentiel de compétence pour Odoo, sa construction dans ce faire dans l'esprit de l'entreprise\footnote{Nous faisons ici référence aux deux dernières valeurs présentent dans le formulaire d'évaluation en annexe: "I want to move forward" et "I prefer to make things evolve than to not makes mistakes"} 

\paragraph{}L'introduction du livre "Élaborer des référentiels de compétences"\citep{refcompetence} propose une méthodologie en neuf étapes pour la mise en place et l'adoption du référentiel et son usage dans l'entreprise. Le processus est représenté à l'annexe C. 

\paragraph{} Les deux premières étapes du processus: "Se doter d'une définition de la compétence" et "Clarifier la finalité" ont été explicité dans le chapitre précédent. Il est interressant de noté que nous avons suivit une approche légèrement différente. Suite à l'article\citep{delobbe}, nous sommes parti de la finalité pour se doter de la définition appropriée de la compétence. Les étapes suivantes seront élaboré dans la suite de ce chapitre. Il est interressant de nous attarder un peu sur les étapes "soumettre à validation" et "organiser l'approbation par les acteurs" car ces étapes risquent de présenter des problèmes assez différents de ce à quoi on pourrait s'attendre dans une mise en place "classique" par le département des ressources humaines ou par le sommet hiérarchique.

\paragraph{} Ces étapes sont nécessaire pour assoire la légimité du référentiel. Le problème se pose généralement pour le département des ressources humaines et pour la hiérarchie de le faire accepter par les employés opérationels. Dans le cas présent, nous avons des éléments assez bas dans la hiérarchie qui définisse le référentiel. Et comme nous l'avons déjà expliqué dans le chapitre précédent, une partie du référentiel devra être construit avec les membres de l'équipe du PS. Il faudra bien sûr les faire valider par ceux-ci. Ceci ne devrait pas trop poser de problème mais il faudrait dans l'idéal le faire accepter par les ressources humaines et le sommet de la hiérarchie. C'est une problématique hautement politique et il est fort probable que non dans un premier temps. Il faut pluôt se poser la question de savoir si la gestion des compétences qui sera mis en place va permettre d'atteindre les objectifs fixés sans leur soutien. Il est fort probable que oui et une fois mis en place et fonctionelle, celle-ci sera beaucoup plus facile à promouvoir au sommet de la hiérarchie. Sachant cela, n'ayons pas peur d'avancer, nous allons donc maintenant décrire ce que nous allons faire lors de chacune des étapes. 

\section{Préciser le format}
Partir des tâches à effectuer

Définir le socle de base

Définir des profils cohérent pour permettre à un minimum de personne de travailler sur une tâche donnée mais aussi parce les compétences sont liées entre elle. 
\section{Recueillir les données}

\section{Traiter les données}


\section{Piloter la mise en usage}
Le référentiel n'est pas un outil qui permet de supporter les processus RH en sois. Le référentiel est plutôt un outil de base duquel dérive toutes sortes d'outils utiles à la sélection, à l'oraganisation du travail, à la formation, à l'évaluation, à la rémunération et à la gestion de carrière.\citep[pp.29]{refcompetence}. Pour chaque objectif que fixés, il faudra construire un outil approprié. 

\begin{description}
  \item[Planification des besoins] 
  Les besoins ne sont hélas pas conduit par la stratégie à long terme, ils sont régis par les tâches journalière et les besoins des clients et la stratégie à cours terme. Le référentiel listera toutes les compétences nécessaires au journalier. Il suffira de faire l'inventaire du niveau de compétence nécessaire et le nombre de personnes qui doivent les posséder. Il y a une autre source de besoin, comme nous sommes face à des personnes hautement qualifié, il faut qu'ils aient l'opportunité de se développer au sein de l'entreprise. Donc même si tout les besoins serton comblé il faudra toujours garder à l'esprit qu'en plus des besoins de l'entreprise il faut garder une trace des besoins de chacuns qui doivent bien entendu rester compatible avec ceux de l'entreprise. 
%TODO exemple de matrice

  \item[Organisation du travail, Développement et formation]
  Pour pouvoir rationnalisé la répartition du travail, il faudra trois outils. Le premier devra permettre de lister les compétences requises et leur niveau pour effectuer la tâche. Le second sera l'inventaire des compétences disponible pour chacun. Et le dernier devra lister les objectifs de chacun en terme de compétence. Ces trois outils permettrons d'assigner aux personnes qui ensemble possède toutes les compétences requis pour effectué là tâche, le plus souvent il n'y aura qu'une personne, mais il faudra aussi tenir compte des objectifs de chacun et si le contexte et la difficulté de la tâche le permet, assigné des personnes qui auront l'opportunité de développer leur compétence dans l'accomplissement de la tâche. %TODO exemple 
  
  \item[Evaluation]
  Dans le référentiel nous avons listé toutes les compétences nécessaires à l'heure actuelle et à venir. Nous avons définis plus haut la matrice des besoins. L'évaluation doit permettre de faire l'inventaire avec l'employé des compétences acquises et leur degré d'acquisition. Il devra aussi permettre de lister les compétences à acquerir ou à perfectionner. Le formulaire d'évaluation se devra de lister les compétences et leur niveau respectifs pour qu'il ne soit pas possible d'en oublier lors de l'entretien d'évaluation mais il devra permettre des laisser des compétences vides sans que cela ne pose problème car personne ne peut maitriser tout les aspects. Les objectifs en termes d'acquisition de compétence devront poussés à la construction de profils cohérents et une fois un profils suffisant maitrisé, l'employé pourra s'étendre à d'autres en fonction des besoins de l'entreprise et de ses désirs. 

\paragraph{}Les niveaux d'acquisition de chaque compétence ont été définit dans le référentiel mais rien n'indique dans le référentiel comment déterminer ce niveau. Cette évaluation devrait faire l'objet d'un chapitre entier si pas d'un travail annexe, il diffère très fortement entre les compétences techniques et fonctionelles et entre les compétences génériques dites transversales. Pour ces dernière, la littérature est abondante et nous n'avons hélas pas le temps de nous y pencher dans ce travail. Pour les compétences techniques et fonctionnelles, nous pouvons explorer une piste. Il serait possible d'utiliser une méthode dérivée de la méthode des incidents critiques.\citep[pp.272]{gestionressourceshumaine2002}. Si on part du principe qu'une compétence ne se manifeste qu'à travers les actions\citep[pp.171]{gestionressourceshumaine2007} et du fait que pour chaque tâche nous avons établit les compétences et leur degré de maitrise nécessaire au bon accomplissement de la tâche, il serait en théorie possible d'inférer les compétences acquises sur base des tâches effectuées durant la période évaluées. Avec cette méthode, quid des compétences acquises et démontré lors de la période précédente mais pas lors de cette période. Les compétences s'accumulent-elles d'années en années ? Y-a-t-il une date de péremption ? Probablement car Odoo se situe dans un contexte qui évolue très vite. Certaine compétence seront suplanté par d'autre plus importantes, d'autres nécessiteront un entretien régulier. 

  \item[Rémunération]
\end{description}


\section{Assurer la maintenance du système}

 




 
\chapter{Comment évaluer les compétences}
L'évaluation des compétences est un domaine très vaste, qui implique la psychologie. Ce document ce contete d'effleurer la problèmatique. Il sera pourtant nécessaire d'explorer ce point pour une mise en pratique de la gestion des compétences. 
\chapter{Mise en oeuvre}
\chapter{Conclusion}
Le travail accomplit ici pourrait facilement être répliqué pour le département de recherche et développement qui nécessite principalement des connaissance techniques pointues. Avec un travail plus poussé sur les compétences génériques, il pourrait aussi s'appliquer à l'équipe des consultants fonctionnels et l'ambition est bien là au sein d'Odoo pour faire ce travail. 
\section{Critique de la gestion des compétences et }






%\lhead{\emph{Conclusion}}


\label{Bibliographie}
\lhead{\nouppercase{\leftmark}}
\bibliographystyle{plain}  % Use the "unsrtnat" BibTeX style for formatting the Bibliography
\bibliography{Bibliography} 



%% ----------------------------------------------------------------
% Now begin the Appendices, including them as separate files
\part*{Annexes}
\lhead{\nouppercase{\leftmark}}
\appendix % Cue to tell LaTeX that the following 'chapters' are Appendices
\chapter{Evolution du nombre d'employés}
\begin{figure}[h!]
    \begin{center}
        \includegraphics[scale=0.6]{document/evolution.png}
        \caption{}
        \label{}
    \end{center}
\end{figure}
\begin{table}[ht!]
    \caption{ Évolution du nombre d'employés chez Odoo en Belgique. Source: Bilan déposé à la BNB de 2005 à 2015\cite{bnb}}
    \label{nb_employe}

    \begin{center}
        \begin{tabular}{|cccc|}
             \hline
             Année & Entrées & Sortie & Total \\
             \hline
             2005 & 2  & 1  & 2   \\
             2006 & 5  & 3  & 4   \\
             2007 & 4  & 2  & 6   \\
             2008 & 7  & 5  & 7   \\
             2009 & 9  & 3  & 18  \\
             2010 & 25 & 9  & 34  \\
             2011 & 24 & 16 & 42  \\
             2012 & 29 & 10 & 61  \\
             2013 & 19 & 16 & 64  \\
             2014 & 66 & 19 & 111 \\
             \hline
        \end{tabular}
    \end{center}
\end{table}





\chapter{Odoo Appraisal Form}
\includepdf[pages=1-6]{document/Odoo_appraisal.pdf}


\addtocontents{toc}{\vspace{2em}}  % Add a gap in the Contents, for aesthetics



\end{document}

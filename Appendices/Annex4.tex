\chapter{Ébauche du référentiel de compétences techniques}

\section{Processus à analyser}
\begin{description}
  \item[Projet]
  Gestion de projet du point de vue de l'équipes techniques, de l'analyse technique à la maintenance en passant par les développement et les tests. L'un des processus les plus complexe à gérer mais aussi l'un des mieux maitriser et mieux documenter
  \item[Support Technique]
  Le processus est en soit assez simple mais relève de cas le plus souvent complexe et inédit. Il nécessite beaucoup d'ingéniosité mais aussi beaucoup de tact.
  \item[Formation]
  Le contenu de la formation est simple et relativement stable. Mais ce processus nécessite des compétences pédagogiques importantes
  \item[Consultance]
  C'est un processus fourre-tout qui nécessite une bonne dose d'improvisation parfois et un bon contact avec le client
  \item[Développement Saas] 
  Ce processus est le plus simple et est assez bien maitrisé. Mais les possibilités limités de l'environnement nécessite pas mal de créativité.
  \item[Avant-vente]
  Malgré ces aspects l'avant-vente est un des processus les plus avancés, il nécessite généralement de bien maitrisé tout les autres mais il faut en plus maitrisé des apsects humains et tenir compte de considération économique.
  \item[Administration de l'infrastructure]
  Processus clairement définit mais qui nécessite des compétences très différent de tout les autres. 
\end{description}


\section{Analyse des tâches du processus: Projet}
\begin{description}
    \item[Développement Backend]
    Cette tâche requiert beaucoup de compétence. D'abord des compétences analytiques. Il faut comprendre la spécification fonctionnelle, l'analyser pour pouvoir la modéliser. Il faut ensuit pouvoir écrire le code, le tester et le delivrer. 
    L'écriture du code demande des connaissance du langage python, du framework odoo, du langage xml et qweb
    \item[Analyse Technique, estimation et faisabilité]
    Cette tâche nécessite des connaissance technique assez générale pour identifier la partie spécifique qui sera impacter par les demandes des clients. Esnuites en fonction du domaine, cette tâche requiert les même compétences que le développement backend ou frond end. Il faut aussi connaitre toutes les phases nécessaire à la production de la fonctionnalité 
    \item[Test fonctionnel]
    La tâche de test fonctionnelle nécessite de connaitre et de comprendre les spécifications fonctionnelles de la fonctionnalité implémentée. Il faut être capabable de concevoir des scénarios généraux mais aussi les scénarios limites à tester. Finalement, il faut être capable de mettre en pratique les scénarios de tests. 
    \item[Test automatique backend]
    Cette tâche est fort similaire à la précédente, elle nécessite de comprendre les spécifications mais aussi le code, de concevoir des scénarios de test et ensuite de les implémenter dans le framework de test Odoo. 
    \item[Revue de la qualité du code]
    L'objectif de cette tâche est de ce prononcer sur la qualité du code écrit. Est-il lisible, répond-il à la spécification, est-il performant et robuste et ensuite de donner un feedback constructif à la personne qui a écrit le code. Cette tâche est l'une des plus complexe et les plus complètes. Elle nécessite des très bonne connaissance des langages qui sont impliqués dans le code à reviewer, du framework odoo pour pouvoir proposer de meilleur solution que celles utilisées. Il faut pouvoir se construire une image globale du code en lisant ligne par ligne. Finalement, il faut une certain tact pour exprimer son feedback. 
    \item[Odoo Deployement] Cette tâche consiste à déployer odoo sur un serveur avec tout le code nécessaire 
\end{description}

\subsection{Tâches non détaillées dans cette annexe}
\begin{itemize}
 \item Test automatique frontend
 \item Développement Front end
 \item Web Design
 \item Administration Postgresql
 \item Performance optimisation
 \item Méthode agile
 \item Gestion des sources
 \item Documentation Technique
 \item Documentation Fonctionnelle
 \item Support
 \item Version Migration
 \item Data Migration
 \item Intégration
\end{itemize}

\section{Description de compétence}
\subsubsection{Savoir programmer en python}
La connaissance du parfait d'un langage est un travail de plusieurs année, mais ce n'est pas ce qui est demandé. Il faut pouvoir manipulé les principales structures de donnée, comprendre la documentation et être capable d'utiliser les bibliothèques nécessaires au développement après en avoir lu la documentation. 
\subsubsection{Connaitre et appliquer les guidelines odoo en programmation}



\subsubsection{Connaître et pouvoir utiliser le framework Odoo pourla crétation/modification de module}
\begin{itemize}
 \item Les vues les plus courantes: Form, Tree, Search, Kanban, Graph, Pivot
 \item Connaître la syntaxe de qweb
\end{itemize}
\subsubsection{Savoir implémenter des tests automatisés pour le backend par rapport au périmètre de son propre code}
Cette compétence nécessite de connaître le python et le framework Odoo en plus de cela il faut connaître le frameword unittest2 ainsi que sont implémentation dans Odoo. Il faut être capable de détecter les cas limites qui valent la peine d'être testés. L'utilisation de l'outils coverage est un plus. 

\subsubsection{Savoir écrire des requête SQL}
Pouvoir écrite des requête de sélection de donnée, de création, de mise à jours et de suppression. Pouvoir effectuer des jointure entre les tables. 
\subsubsection{Compréhension fonctionnelle d'un module métier}
Avoir une compréhension fonctionnelle d'un module métier consiste à connaître son but, avoir fait l'expérience d'un cas concret d'implémentation du module, de pouvoir le configurer et de connaitre ses principales fonctionnalités. On peut citer les principaux modules:
\begin{itemize}
 \item CRM
 \item Vente
 \item Gestion de stock
 \item Gestion de la production
 \item Achat
 \item Comptabilité 
 \item Comptabilité analytique
 \item Marketing
 \item Point de vente
 \item Gestion de projet
 \item Gestion des ressources humaines
 \item Website
 \item E-commerce
\end{itemize}
\subsubsection{Comprendre et analyser les besoins métier du client}
\subsubsection{Pouvoir modéliser les besoins du client en termes d'objets et d'interaction entre les objets.}
 - Traduie les demandes des clients en spécification techniques
\subsubsection{Être capable de découper un problème en plusieurs sous problème}


\subsubsection{Être capabable de déployer et d'administrer Odoo}

\subsection{Compétence non détaillées dans cette annexe}
\begin{itemize}
 \item Savoir Programmer en javascript
 \item Savoir Programmer en bash
 \item Savoir écrire des fichier xml odoo et des vues backend
 \item Savoir écrire des vues qweb
 \item Être capabable d'administrer un serveur Postgresql
 \item Être capabable d'administrer un serveur linux
 \item Pouvoir adapter son discours en fonction de son interlocuteur
 
\end{itemize}


\section{Exemple de socle de base}
\begin{itemize}
 \item Être capable de découper un problème en plusieurs sous problème
 \item Savoir programmer en python
 \item Être capabable de déployer et d'administrer Odoo
 \item Connaître et pouvoir utiliser le framework Odoo pourla crétation/modification de module
 \item Connaitre et appliquer les guidelines odoo en programmation
 \item Savoir implémenter des tests automatisés pour le backend par rapport au périmètre de son propre code
 \item Compréhension fonctionnelle de 2 modules
\end{itemize}


\section{Exemple de profil cohérent}
\subsection{Profil : Développeur Backend}
\begin{itemize}
 \item Connaitre le framework Odoo
 \item Être capable de modifier des modules Odoo existant
 \item Être capable d'étendre des modules Odoo existant
 \item Être capable créer des nouveaux modules
 \item Savoir programmer en python
 \item Savoir écrire des requête SQL
 \item Savoir écrire des fichier xml odoo et des vues backend
 \item Connaitre les outils de débogage python
 \item Savoir implémenter des tests automatisés pour le backend par rapport au périmètre de son propre code
 \item Connaitre les guideline de développement Odoo
 \item Connaissance de git: branche, tag, merge, diff, stash, checkout, clone
 \item Connaissance technique des modules de base: Base, Email

\end{itemize}

